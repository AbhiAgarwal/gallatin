\documentclass[11pt, oneside]{article}   	% use "amsart" instead of "article" for AMSLaTeX format
\usepackage{geometry}                		% See geometry.pdf to learn the layout options. There are lots.
\geometry{letterpaper}                   		% ... or a4paper or a5paper or ... 
%\geometry{landscape}                		% Activate for for rotated page geometry
%\usepackage[parfill]{parskip}    		% Activate to begin paragraphs with an empty line rather than an indent
\usepackage{graphicx}				% Use pdf, png, jpg, or eps§ with pdflatex; use eps in DVI mode
								% TeX will automatically convert eps --> pdf in pdflatex		
\usepackage{amssymb}

\title{Intellectual Autobiography and Plan for Concentration}
\author{Abhi Agarwal}
\date{} % hides the date

\begin{document}
\maketitle
\section{Intellectual Autobiography}

\par I choose Gallatin because I didn't want to go the traditional route in studying Engineering, or Computer Science, but have more freedom and develop skills that are directly relevant to the things I want to pursue in the future, and also things I am interested in. A computer science degree usually doesn't allow you to explore the things you want to do in depth, but allows you to understand a broad range of concepts. I enjoy learning and building upon the ideas of Robotics, Artificial Intelligence, and the Engineering Principles behind them, but also to explore the philosophical aspects of these things and so I believe Gallatin was the perfect fit for me. Moreover, Gallatin's culture of concentrating and focusing on learning and explore newer ideas rather than having classes to fulfill requirements is something I have thoroughly enjoyed.

\par In the past I have taken two internships, but also have worked on a large amount of personal projects, and worked on projects with a team. The two internships I took were at MedConsultAsia, and at Aptech. The two internships gave me experience in different styles of programming, and different use cases as I was working in at an education-based environment at Aptech, while at MedConsultAsia I was working at a medical research company. Through these internships, apart from the programming experience, I understood how to shape different products and user experiences for the different projects, which was fundamentally important for me to arriving at my concentration. Previous to my two internships, I started experimenting with web development. Through middle-school and junior-high, I worked over over 10 different projects, but spent time concentrating on one, which gave my most of my understanding of programming.

\par I started getting more interested in Computer Science theory in high-school when I took the International General Certification of Secondary Education (IGCSE), and the International Baccalaureate (IB). I developed a decent understanding for the basics of Computer Science through the Computer Science curriculum, but also Mathematics, and Physics, which were a requirement. The mixture of these led me to discovering Robotics, and Artificial Intelligence, which increased my interest in hardware. I wasn't able to start tinkering with hardware until I got to college as I was never motivated to test the things I learn in Physics in a real world environment. However, coming to college I been fortunate to take Physical Computing, which has helped me to improve my understanding of prototyping and the mixture of Physics and Computer Science, and how I am able to use simple material to manipulate complex components.

\par The classes that have inspired me to move towards my concentration at New York University have been a mixture of courses at Courant and at Gallatin. At Courant I took a total of 9 courses in Computer Science, but the ones that have inspired me the most have been: Machine Learning, Artificial Intelligence, Simulations, Physical Computing, and Basic Algorithms. These courses have shaped my fundamental understanding of what I thought Artificial Intelligence is, and how I am able to apply my intuition to solve problems rather than focus on my algorithmic knowledge. In addition, the course Algorithmic Problem Solving has allowed me to test and train my knowledge of algorithms, but in a practical way.

\par Moreover, the classes Utopia, and the Digital Commons have shaped the way I fundamentally perceive communities, and behavior in general. In the Digital Commons we looked at communities both offline and online, and how people engage in conversations and how people use discourse to share their subjective opinions, which allowed me to understand human behavior in depth and increased my interest in artificial communities (discourse between intelligent agents). Utopia helped me shape my understanding of how I want to see the world develop as we thoroughly debated the differences in quality of life in each Utopia, and how changing different things in a Utopia would led to different changes.

\par Through the last 2 years I have also been fortunate enough to work on projects outside of class with other individuals. I have worked on libraries (a set of programming tools to simplify development), startup ideas, and hackathons to improve my programming skills, but more importantly they have allowed me to learn how to consider edge-cases and improved my though process when I have a new idea. I am now able to think through a project from developing it analytically to actually completing a prototype quickly, and I think this is a skill that is extremely important in the Computer Science field.

\section{Concentration}

\par I am going to title my concentration `Engineering Creation'. Engineering Creation is learning the process of thinking, designing, and engineering tools and products and how it leads to creation. It's observing the union of both engineering and creation, and looking at the intersection of the two in detail. In addition, I will also be taking a look at those concepts while considering their links to Artificial Intelligence and Robotics.

\section{Plan for Concentration}

\par In the next two years of my study I am going to keep pursuing the majority of my subjects in Computer Science to maximize my knowledge of the subject area, but specifically target graduate courses in Computer Science. By the end of my spring semester (2014) I will have fulfilled the Computer Science major requirements, which also include the four electives - so for my next set of Computer Science courses I will look towards graduate courses (more advanced in Artificial Intelligence) or more Computer Science electives. Specifically I will be looking to study the following courses in depth:

\begin{itemize}
\item Natural Language Processing
\item Introduction to Machine Learning
\item Deep Learning
\item Parallel Computing
\item Realtime analysis
\item Advanced Algorithms/Heuristic Problem Solving 
\item Compiler Construction
\item Big data analysis
\item Distributed Systems
\end{itemize}

\par All these courses will play within my advantage for my field of study in the future. Next semester I will also be moving towards doing some research in Artificial Communities. It's something I thought about as there are more intelligent agents, and more Artificial Intelligence how communities will develop amongst them, and how the communities will form.

\par Apart from Computer Science I will be moving towards taking more Philosophy, Psychology, Circuitry, Logic, Law, Knowledge, and Open Access classes. I believe I have to touch these subjects as I have always been fascinated by each of these and some of them will help me develop my understanding outside just Computer Science. For my Gallatin courses I?m aiming to take a few classes in Religion \& Philosophy as these are subjects I haven?t read on extensively in the best, and so I?m thinking of taking classes along the lines of Buddhism and Psychology, Spirituality, and Law and Legal Thought (offered Spring 2014). I am also fairly interested in city-planning, and I would also like to explore classes such as Designing the Future City if I ever have the opportunity to do so (In addition to my Utopia class).

\par In the future after graduation I am interested in working in the consumer software industry before starting a project of my own. I am fairly interested in introducing social search, which is a project I worked on this spring, and working on a few other ideas I have been developing. I believe Gallatin has been the perfect way for me to explore my passions while also being pushed to take classes that would benefit my future.

\end{document}  