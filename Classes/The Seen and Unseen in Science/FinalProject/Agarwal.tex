\documentclass[11pt, oneside]{article}
\usepackage[a4paper]{geometry}
\geometry{letterpaper}
\usepackage{amssymb}
\linespread{2.3}
\usepackage[affil-it]{authblk} 
\usepackage{etoolbox}
\usepackage{lmodern}
\usepackage{ifpdf}
\usepackage{mla}
\usepackage{blindtext}
\usepackage[english]{babel}
\usepackage{fancyhdr}

\renewcommand\Authfont{\fontsize{12}{14.4}\selectfont}
\renewcommand\Affilfont{\fontsize{9}{10.8}\itshape}

\setlength{\voffset}{-0.7in}
\setlength{\headsep}{6pt}

\title{Exploring historically how intelligence has been defined, and why people have attempted to define it\vspace{-0.4cm}}
\author{Abhi Agarwal\vspace{-1cm}}
\date{}

\begin{document}

\maketitle

% Is there one trait intelligence boils down to?
% Should there be one trait? Should we be looking for that one trait?

\par Most of the history of the study of intelligence is dominated by psychologists. From Francis Galton to Alfred Binet to Cyril L. Burt to Howard Gardner. Most of the history we explore and read about of intelligence testing, definitions of intelligence, or the measurement of intelligence has been attributed and built by psychologists. This paper explores how intelligence has been defined, measured, and the aim of defining intelligence from the perspective of individuals in different fields.

% Talk about historical foundations of: Plato, Aristotle, Thomas Aquinas, Juan Huarte, Jean-Marc Gaspard Itard.
% http://www.intelltheory.com/map.shtml
% http://www.intelltheory.com/darwin.shtml 
% http://www.gsjournal.net/old/babin/intro1.htm

\par 

% Perhaps have a labeled diagram of the most influential intelligence thinkers and label them by their fields. I want to explore the intentions and measures of what other people in other fields talked about intelligence. 

% Charles Darwin (1809) (Naturalist)
% http://www.intelltheory.com/darwin.shtml

\par The early debates in understanding intelligence were to determine if hereditary or not. Charles Darwin's evolutionary theories were central to a lot of arguments presented in the early intelligence theories. Darwin's and Wallace's theories provided an argument to begin the nurture vs nature debate. Barlow, Darwin's granddaughter, in her additions to `The Autobiography of Charles Darwin', quotes Darwin, in one his letters, as saying ``I do not think that I owe much to him intellectually-nor to my four sisters... I am inclined to agree with Francis Galton in believing that education and environment produce only a small effect on the mind of any one, and that most of our qualities are innate" (Darwin, `The autobiography of Charles Darwin', 43). Darwin's work inspired Francis Galton who after the publication of Darwin's theories went on to coin the terms eugenics, and `nature versus nurture'. Darwin supported Galton's theories and in his letter began to define what attributed intelligence. Darwin, as a naturalist, defined the attributes of intelligence to show that ``there is no fundamental difference between man and the higher mammals in their mental faculties" (Darwin, `The Descent of Man', 66). 

% Jean-Martin Charcot (French Neurologist)
% https://en.wikipedia.org/wiki/Jean-Martin_Charcot
% Teacher of Freud, Binet

\par Jean-Martin Charcot was the father of neurology. His theories attempted to link physiological and mental processes together, and were the foundations to show correlations between genetics, neurophysiology and intelligence. Charcot, similar to Darwin, also contributed to understanding the hereditary nature of intelligence by advocating and furthering the diathesis-stress model. The diathesis-stress model is one of the widely credited models that is used to explain the influence the environment has on genetics. His research on hysteric patients showed that most of the patients had a genetic predisposition to the disease, but the patient was more likely to get the disease after being exposed to certain environmental factors. This was slightly different to Darwin's point of view, but the model still contributed to the idea that genetics were a contributing factor to diseases, and other bodily developments. Charcot's research stressed the importance of environmental factors (in conjunction to hereditary factors) more than his predecessors did, which was important since he showed it through conclusions made from his patients rather than from his past experiences. In addition, Charcot is also celebrated as the teacher for both Binet and Freud who also both went on to contribute to intelligence theories directly or indirectly. 

\par Intelligence theory during that period revolved a lot around understanding how intelligence was passed on rather than the measurement of it. Darwin, Charcot, and Galton, directly or indirectly, established the modern foundations for intelligence theory. The theories presented by each of them contributed to the nature versus nurture debate, and the debate was central to the understanding of intelligence was passed on. The definition and measurement of intelligence would be very different if intelligence was determined more by nurture than nature or if intelligence was not hereditary. At the time, it was incredibly hard to define intelligence since there were mixed theories on what contributed to intelligence, and the arguments presented by Darwin, Charcot, and Galton helped formulate a clearer understanding of its source. 

\par Galton's contribution were incredibly important as he measured the inheritance of intelligence through data in his book `Hereditary Genius'. In the book Galton showed that ``human mental abilities and personality traits, no less than the plant and animal traits described by Darwin, were essentially inherited" (Seligman, 54), and in his book he describes genius as ``an ability that was exceptionally high and at the same time inborn" (Galton, Hereditary Genius, 6). The basis of his argument followed Darwin's argument where he claimed that nature always prevailed against nurture. In the book he examines data on ``wrestlers of the North Country" (Galton, Hereditary Genius, 312), literary men, commanders, men of science, poets, musicians and uses their heritages and accomplishments to show how they correlated to intelligence (statistical correlation being a concept he created). His work became important in the foundations of intelligence research, as it linked logic and deduction with statistical evidence. This was a shift in paradigm for the community around intelligence thinking since it was the first time someone had used numerical data to compare the intelligence of groups of people, and also based their theory on it.

\par In addition, it is also important to discuss the aims that Galton had to define intelligence. Through his work on looking at ``wrestlers of the North Country", Peerages of England, and more, the last two chapters of his book were called ``The comparative worth of different races", and "Influences that affect the natural ability of nations". A lot of his work is significant in intelligence theories, but his aims were not well-founded. Through his data he makes claims such as, ``the number among the negroes of those whom we should call half-witted men, is very large" (Galton, Hereditary Genius, 339), and ``that the average ability of the Athenian race is, on the lowest possible estimate, very nearly two grades higher than our own (Galton, Hereditary Genius, 342). Throughout his book he develops this grading framework to be able to quickly pinpoint which civilization matches which grade depending on their accomplishments, and intelligence levels. The framework he was developing for the measurement of intelligence was to comment and develop a theory about race superiority.

\par Similarly, about a hundred years before Galton, Samuel Morton made similar conclusions about race and their intellectual ability. Morton, a physician, in his work `Crania Americana', claimed that you could measure the intellectual ability of a race by their skull capacity. Morton, similar to Galton, described the Caucasian as ``distinguished by the facility with which it attains the highest intellectual endowments" (Morton, 2), and ``the Negro" as ``joyous, flexible, and indolent; while the many nations which compose this race present a singular diversity of intellectual character, of which the far extreme is the lowest grade of humanity" (Morton, 3). They were expressing their biased opinions on other races through `science', but their use of statistics and intelligence theories made it more harmful as people perceived that they had data to back up their views. Gould writes about Morton in `The Mismeasure of Man' and describes Morton as having an unconscious bias (among other reasons) that supposed his prejudicial views.

\par Through the research it can be seen that early intelligence research was biased as the theoretical framework of the researchers lead them to use intelligence theories to show race superiority. After Darwin's `On the Origin of Species' a lot of  Galton went on to write % Eugenics stuff

The field of intelligence theory was formally established by three key individuals: Wilhelm Wundt, James McKeen Cattell, and Alfred Binet. 

\par Wilhelm Wundt

\par James McKeen Cattell

% Alfred Binet (1857) (Psychologist)
% https://en.wikipedia.org/wiki/Alfred_Binet

\par Alfred Binet was a psychologist who was the inventor of the first used intelligence test. 
\par ``It seems to us that in intelligence there is a fundamental faculty, the alteration or the lack of which, is of the utmost importance for practical life. This faculty is judgment, otherwise called good sense, practical sense, initiative, the faculty of adapting one's self to circumstances" (Binet, 42).

% Charles Spearman (1863) (Psychologist)
% https://en.wikipedia.org/wiki/Charles_Spearman

\par Charles Spearman

% Jean Piaget (maybe)
% https://en.wikipedia.org/wiki/Jean_Piaget#Artificial_intelligence

\par Jean Piaget

% Vernon Mountcastle (1918) paper in 1957 (Neuroscientist)
% https://en.wikipedia.org/wiki/Vernon_Benjamin_Mountcastle

\par Vernon Mountcastle 
\par His work was not directly in the field of intelligence research, but in cognition and in the cerebral cortex. These are 

% Howard Gardner (1943) paper in 1983 (developmental psychologist)
% https://en.wikipedia.org/wiki/Howard_Gardner
% https://en.wikipedia.org/wiki/Theory_of_multiple_intelligences

\par Howard Gardner, in his book `Frames of Mind: The Theory of Multiple Intelligences', formulates eight particular behavior that must be observed in order to be considered intelligent. 

% Alan Turing
% Computer Scientist

\par Alan Turing

% Jeff Hawkins (2005)
% https://en.wikipedia.org/wiki/Jeff_Hawkins
% https://en.wikipedia.org/wiki/On_Intelligence

\par Jeff Hawkins, a well known technologist, develops his theory, from the background of a Electrical Engineer turned Biophysicist turned self-taught Neuroscientist, in his book `On Intelligence'. Hawkins' aim to measure and quantify intelligence is to be able to re-build it in the form of an Artificially Intelligent agent. This is fundamentally different from the aims of many other researcher seeking to define and measure intelligence. The distinction is important since most researchers seek to explore or quantify intelligence for philosophical or experimentation purposes, but Hawkins seeks to go a step further and make attempts to build it. In addition, the theoretical framework or worldview of being able to build an intelligent machine only became possible after the formation of machines and the fields of Artificial Intelligence and Computer Science.

\par Hawkins' theory is solidified on the idea that intelligence is just learning to see repetitions in patterns, and does not constitute of the ability to do any specific tasks. Hawkins writes ``the intelligent machine must learn via observation of its world, including input from an instructor when necessary. Once our intelligent machine has created a model of its world, it can then see analogies to past experiences, make predictions of future events, propose solutions to new problems, and make this knowledge available to us" (Hawkins, 209). In addition, Hawkins is also highly influenced in his work by Mountcastle's research, and some of his theories are grounded in the correctness of Mountcastle's theories. Hawkins believes that the different areas of your brain that learn (i.e. eyes for learning to see), learn in the exact same way. This means that there is potential to find an algorithm that's able to represent the foundations of learning, and by extension his view and definition of intelligence boils down to this one-learning algorithm.

\par In conclusion, a lot of different individuals have studied intelligence in a lot of different aspects, and for a lot of different applications. Understanding human intelligence, what it means, and how to measure it has been a pretty central problem in the past, and there is still active research on understanding it now.

\noindent 

\begin{workscited}
\bibent \\
\bibent Hawkins, Jeff, and Sandra Blakeslee, On Intelligence. New York: Henry Holt, 2005. Print. \\
\bibent Darwin, Charles. and Barlow, Nora, The autobiography of Charles Darwin. London: Collins, 1958. Print. \\
\bibent Binet. Alfred, and Simon, Theodore, The development of intelligence in children. Baltimore, Williams \& Wilkins, 1916. \\
\bibent Seligman, D. (2002). Good breeding. National Review, 1916.
Print. \\
\end{workscited}
\end{document}