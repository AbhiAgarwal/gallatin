\documentclass[11pt, oneside]{article}
\usepackage[a4paper]{geometry}
\geometry{letterpaper}
\usepackage{amssymb}
\linespread{2.3}
\usepackage[affil-it]{authblk} 
\usepackage{etoolbox}
\usepackage{lmodern}
\usepackage{ifpdf}
\usepackage{mla}
\usepackage{blindtext}
\usepackage[english]{babel}
\usepackage{fancyhdr}

\renewcommand\Authfont{\fontsize{12}{14.4}\selectfont}
\renewcommand\Affilfont{\fontsize{9}{10.8}\itshape}

\setlength{\voffset}{-0.7in}
\setlength{\headsep}{6pt}

\title{Exploring historically how intelligence has been defined, and why people have attempted to define it\vspace{-0.4cm}}
\author{Abhi Agarwal\vspace{-1cm}}
\date{}

\begin{document}

\maketitle

% Is there one trait intelligence boils down to?
% Should there be one trait? Should we be looking for that one trait?

\noindent 

\begin{workscited}
\bibent \\
\bibent Hawkins, Jeff, and Sandra Blakeslee, On Intelligence. New York: Henry Holt, 2005.
Print. \\
\end{workscited}
\end{document}