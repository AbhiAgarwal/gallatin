\documentclass[11pt, oneside]{article}
\usepackage[a4paper]{geometry}
\geometry{letterpaper}
\usepackage{amssymb}
\linespread{2.3}
\usepackage[affil-it]{authblk} 
\usepackage{etoolbox}
\usepackage{lmodern}
\usepackage{ifpdf}
\usepackage{mla}
\usepackage{blindtext}
\usepackage[english]{babel}
\usepackage{fancyhdr}

\renewcommand\Authfont{\fontsize{12}{14.4}\selectfont}
\renewcommand\Affilfont{\fontsize{9}{10.8}\itshape}

\setlength{\voffset}{-0.7in}
\setlength{\headsep}{6pt}

\title{Exploring historically how intelligence has been defined, and why people have attempted to define it\vspace{-0.4cm}}
\author{Abhi Agarwal\vspace{-1cm}}
\date{}

\begin{document}

\maketitle

% Is there one trait intelligence boils down to?
% Should there be one trait? Should we be looking for that one trait?

\par Most of the history of the study of intelligence is dominated by psychologists. From Francis Galton to Alfred Binet to Cyril L. Burt to Howard Gardner. Most of the history we explore and read about of intelligence testing, definitions of intelligence, or the measurement of intelligence has been attributed and built by psychologists. This paper explores how intelligence has been defined, measured, and the aim of defining intelligence from the perspective of individuals in different fields.

% Talk about historical foundations of: Plato, Aristotle, Thomas Aquinas, Juan Huarte, Jean-Marc Gaspard Itard.
% http://www.intelltheory.com/map.shtml
% http://www.intelltheory.com/darwin.shtml 
% http://www.gsjournal.net/old/babin/intro1.htm

% Not done plato (etc.) yet.

\par The early debates in understanding intelligence were to determine if hereditary or not. Charles Darwin's evolutionary theories were central to a lot of arguments presented in the early intelligence theories. Darwin's and Wallace's theories provided an argument to begin the nurture vs nature debate. Barlow, Darwin's granddaughter, in her additions to `The Autobiography of Charles Darwin', quotes Darwin, in one his letters, as saying ``I do not think that I owe much to him intellectually-nor to my four sisters... I am inclined to agree with Francis Galton in believing that education and environment produce only a small effect on the mind of any one, and that most of our qualities are innate" (Darwin, `The autobiography of Charles Darwin', 43). Darwin's work inspired Francis Galton who after the publication of Darwin's theories went on to coin the terms eugenics, and `nature versus nurture'. Darwin supported Galton's theories and in his letter began to define what attributed intelligence. Darwin, as a naturalist, defined the attributes of intelligence to show that ``there is no fundamental difference between man and the higher mammals in their mental faculties" (Darwin, `The Descent of Man', 66). 

\par Jean-Martin Charcot was the father of neurology. His theories attempted to link physiological and mental processes together, and were the foundations to show correlations between genetics, neurophysiology and intelligence. Charcot, similar to Darwin, also contributed to understanding the hereditary nature of intelligence by advocating and furthering the diathesis-stress model. The diathesis-stress model is one of the widely credited models that is used to explain the influence the environment has on genetics. His research on hysteric patients showed that most of the patients had a genetic predisposition to the disease, but the patient was more likely to get the disease after being exposed to certain environmental factors. This was slightly different to Darwin's point of view, but the model still contributed to the idea that genetics were a contributing factor to diseases, and other bodily developments. Charcot's research stressed the importance of environmental factors (in conjunction to hereditary factors) more than his predecessors did, which was important since he showed it through conclusions made from his patients rather than from his past experiences. In addition, Charcot is also celebrated as the teacher for both Binet and Freud who also both went on to contribute to intelligence theories directly or indirectly. 

\par Intelligence theory during that period revolved a lot around understanding how intelligence was passed on rather than the measurement of it. Darwin, Charcot, and Galton, directly or indirectly, established the modern foundations for intelligence theory. The theories presented by each of them contributed to the nature versus nurture debate, and the debate was central to the understanding of intelligence was passed on. The definition and measurement of intelligence would be very different if intelligence was determined more by nurture than nature or if intelligence was not hereditary. At the time, it was incredibly hard to define intelligence since there were mixed theories on what contributed to intelligence, and the arguments presented by Darwin, Charcot, and Galton helped formulate a clearer understanding of its source. 

\par Galton's contribution were incredibly important as he measured the inheritance of intelligence through data in his book `Hereditary Genius'. In the book Galton showed that ``human mental abilities and personality traits, no less than the plant and animal traits described by Darwin, were essentially inherited" (Seligman, 54), and in his book he describes genius as ``an ability that was exceptionally high and at the same time inborn" (Galton, Hereditary Genius, 6). The basis of his argument followed Darwin's argument where he claimed that nature always prevailed against nurture. In the book he examines data on ``wrestlers of the North Country" (Galton, Hereditary Genius, 312), literary men, commanders, men of science, poets, musicians and uses their heritages and accomplishments to show how they correlated to intelligence (statistical correlation being a concept he created). His work became important in the foundations of intelligence research, as it linked logic and deduction with statistical evidence. This was a shift in paradigm for the community around intelligence thinking since it was the first time someone had used numerical data to compare the intelligence of groups of people, and also based their theory on it.

\par In addition, it is also important to discuss the aims that Galton had to define intelligence. Through his work on looking at ``wrestlers of the North Country", Peerages of England, and more, the last two chapters of his book were called ``The comparative worth of different races", and "Influences that affect the natural ability of nations". A lot of his work is significant in intelligence theories, but his aims were not well-founded. Through his data he makes claims such as, ``the number among the negroes of those whom we should call half-witted men, is very large" (Galton, Hereditary Genius, 339), and ``that the average ability of the Athenian race is, on the lowest possible estimate, very nearly two grades higher than our own (Galton, Hereditary Genius, 342). Throughout his book he develops this grading framework to be able to quickly pinpoint which civilization matches which grade depending on their accomplishments, and intelligence levels. The framework he was developing for the measurement of intelligence was to comment and develop a theory about race superiority.

\par Similarly, about a hundred years before Galton, Samuel Morton made similar conclusions about race and their intellectual ability. Morton, a physician, in his work `Crania Americana', claimed that you could measure the intellectual ability of a race by their skull capacity. Morton, similar to Galton, described the Caucasian as ``distinguished by the facility with which it attains the highest intellectual endowments" (Morton, 2), and ``the Negro" as ``joyous, flexible, and indolent; while the many nations which compose this race present a singular diversity of intellectual character, of which the far extreme is the lowest grade of humanity" (Morton, 3). They were expressing their biased opinions on other races through `science', but their use of statistics and intelligence theories made it more harmful as people perceived that they had data to back up their views. Gould writes about Morton in `The Mismeasure of Man' and describes Morton as having an unconscious bias (among other reasons) that supposed his prejudicial views. Among other reasons, the general acceptance of theories Darwin published in `On the Origin of Species' disproved Morton's work, and marked it as an origin of scientific racism.

\par Through the research it can be seen that early intelligence research was biased as the theoretical framework of the researchers lead them to use intelligence theories to show race superiority. Galton went on to write `Inquiries into Human Faculty and its Development', which gave gave support to the eugenics movement. Arthur Jensen, in his book `Galton's legacy to research on intelligence', writes ``it seemed obvious and even unarguable to Galton that, from a eugenic viewpoint, superior mental and behavioral capacities, as well as physical health, are advantageous, not only to an individual but for the well-being of society as a whole" (Jensen, 145), and ``Galton's pioneer contributions to the science of mental ability became so amalgamated with his enthusiasm for eugenics as to have also contributed to the disfavor in which Galtonian research on human intelligence has been held in the latter half of the twentieth century" (Jensen, 145). Galton and Morton both thought they were scientifically correct, but ``what seems not to have been sufficiently clear, even to Galton himself, yet needs to be emphasized, is this: prescriptive eugenics falls not in the purview of science, but in the province of moral philosophy" (Jensen, 150). The distinction that Jensen attempts to create is that there is a difference with what is, and what Galton and Morton thought should be. The eugenics movement clearly portrayed the aims of the early researchers in trying to define and measure intelligence. The study of intelligence gave them a lot of power and influence since they were able to manipulate data to show put forth their views, however biased they were.

\par In addition, on the first page of Galton's book `Hereditary Genius' he writes ``I conclude that each generation has enormous power over the natural gifts of those that follow, and maintain that it is a duty that we owe to humanity to investigate the range of that power, and to exercise it in a way that, without being unwise towards ourselves, shall be most advantageous to future inhabitants of the earth" (Galton, Hereditary Genius, 1). This is a very powerful and encompassing statement that was influential to the views that a lot of intelligence researchers had in that century. Furthermore, at this point in time, intelligence theories were mostly consisted of psychologists, biologists, neurologists looking at exploring differences in humans and comparing their intellectual abilities.

\par Influenced by the early research of Galton, Charcot, Darwin, and others, the field of intelligence theory was formally established by three key individuals: Wilhelm Wundt, James McKeen Cattell, and Alfred Binet. Wilhelm Wundt was a Philosopher and is known to be the father of Experimental Psychology. Wundt begun using psychological techniques in the laboratory, and through this theoretical framework and background in philosophical he viewed the mind as an activity rather than a solid mass as biologists did, which in a way was similar to Freud. Wundt is indirectly influential in intelligence theories since he developed his theories in his book `Principles of Physiological Psychology', which became the basis of a lot of psychology labs, but was also instrumental in the establishment of the first intelligence test. His theory establishes the principle of recording mental activity based on reactions and measurable activity of the patients and objective knowledge. The development of modern psychology and experimental psychology was important for research around intelligence since intelligence was not something that can be directly measured (even now).

\par James McKeen Cattell, a student of Wundt, worked along side of Wundt to help establish these correct practices, and helped establish the field of psychology. During the development of Wundt's framework, Cattell started to experiment with measuring reaction time rates of individuals based on different activities, and began to study the underlying differences they had based on their reaction times. Cattell is basically measuring different mental processes, such as reaction times of their memory, movement, and he also measured their sense of weights and if they could observe noticeable differences between different masses. In a lot of his research he believed that his experimentations were leading him to measure intelligence as he explored mental aptitude. His research on measuring intelligence was not successful, as we know, because he spent time doing experimentations on intelligence was linked to people's bodily features rather than doing written tests. In addition, Cattell was also a big supporter of eugenics. He went so far as to proposition his son/daughter with money if they were to marry a suitor of a high achieving background (such as a professor or an academic). His aims on his research are not clear, but he was known to be pleased about his `inherited ability', which for him meant that he had influence and could make a difference.

\par Alfred Binet was a psychologist who was the inventor of the first used intelligence test. Binet began researching alongside Charcot in Charcot's laboratory. Binet was able to learn a lot from the experimental nature of the laboratory, and a lot from the views that Charcot had on hypnotism. Later, it was shown that Charcot's hypothesis were incorrect, and Binet began to shift his work to differential, experimental, and many other types of psychology theories. His view of intelligence theories shifted from attempting to understand intellectual ability to understanding intellectual development and its relations to attention span and a person's receptiveness. His view started shifted after his daughters were born as he begun doing intelligence testing on them, and started giving them tests. This was a large shift as it concentrated on understanding how intelligence in a person developed rather than examining how intellectual a person was in a short period of time.

\par Furthermore, Binet began working with the French government on the education of retarded children. A part of this project was to develop methodology to identify the particular student who needed to be a part of this group. Over a couple years, Theodore Simon, a colleague of Binet's, and Binet started to devise questions that separated the children. They asked the children to name body parts or count coins, solve patterns, and the questions got progressively harder. They devised 30 questions, and the first set of questions were questions that all children could answer (regardless of however challenged they were). He found that intellectual ability developer in children in variable rates, which had to be accounted for, and he stressed the study of intelligence to be qualitative rather than through quantitate measures. Binet and Simon in regards to intelligence write, ``it seems to us that in intelligence there is a fundamental faculty, the alteration or the lack of which, is of the utmost importance for practical life. This faculty is judgment, otherwise called good sense, practical sense, initiative, the faculty of adapting one's self to circumstances. A person may be a moron or an imbecile if he is lacking in judgment; but with good judgment he can never be either. Indeed the rest of the intellectual faculties seem of little importance in comparison with judgment" (Binet and Simon, 42). 

\par Simon and Binet developed this measure of intelligence for social good. Their intentions have seemed to be very positive and they both disagreed with a lot of the research that came out of their experimentation and research. He condemned the research that happened at Stanford and the development of his research that viewed intelligence as a being one-dimensional or having one aspect to it. Meanwhile, in the U.S., in 1908, H.H. Goddard translated Simon and Binet's intelligence test into English, and began distributing it to the academic audience. Goddard also went on to further it in his books `Standard method for giving the Binet test', `Feeble-Mindedness: Its Causes and Consequences', `School Training of Defective Children', and many more. Historically, this is an important contribution as it further accelerated and captured the interests of psychologists, mathematicians, and a lot more people around the world. Goddard, pre-1908, was becoming a huge advocate for institutions (such as a hospital, school, legal system, etc.) to begin using intelligence testing. One of his largest accomplishments was requiring deaf, blind, and mentally retarded students to get special education, and also introducing into law that criminals with subpar intelligence should limit their criminal responsibility. Even though he also advocated for eugenics, he was able to utilize this measure to bring some social good for the years to follow.

% Terman (1916) and Stern (1912) and Yerkes and Charles Spearman

\par Lewis Terman and William Stern published their paper `Stanford Revision of the Binet-Simon Scale' in 1916 after being inspired by H.H. Goddard's translation.

\par Yerkes, Charles Spearman

% Jean Piaget (maybe)
% https://en.wikipedia.org/wiki/Jean_Piaget#Artificial_intelligence

\par Jean Piaget

% Vernon Mountcastle (1918) paper in 1957 (Neuroscientist)
% https://en.wikipedia.org/wiki/Vernon_Benjamin_Mountcastle

\par Vernon Mountcastle 
\par His work was not directly in the field of intelligence research, but in cognition and in the cerebral cortex. These are 

% Howard Gardner (1943) paper in 1983 (developmental psychologist)
% https://en.wikipedia.org/wiki/Howard_Gardner
% https://en.wikipedia.org/wiki/Theory_of_multiple_intelligences

\par Howard Gardner, in his book `Frames of Mind: The Theory of Multiple Intelligences', formulates eight particular behavior that must be observed in order to be considered intelligent. 

% Alan Turing
% Computer Scientist

\par Alan Turing

% Jeff Hawkins (2005)
% https://en.wikipedia.org/wiki/Jeff_Hawkins
% https://en.wikipedia.org/wiki/On_Intelligence

\par Jeff Hawkins, a well known technologist, develops his theory, from the background of a Electrical Engineer turned Biophysicist turned self-taught Neuroscientist, in his book `On Intelligence'. Hawkins' aim to measure and quantify intelligence is to be able to re-build it in the form of an Artificially Intelligent agent. This is fundamentally different from the aims of many other researcher seeking to define and measure intelligence. The distinction is important since most researchers seek to explore or quantify intelligence for philosophical or experimentation purposes, but Hawkins seeks to go a step further and make attempts to build it. In addition, the theoretical framework or worldview of being able to build an intelligent machine only became possible after the formation of machines and the fields of Artificial Intelligence and Computer Science.

\par Hawkins' theory is solidified on the idea that intelligence is just learning to see repetitions in patterns, and does not constitute of the ability to do any specific tasks. Hawkins writes ``the intelligent machine must learn via observation of its world, including input from an instructor when necessary. Once our intelligent machine has created a model of its world, it can then see analogies to past experiences, make predictions of future events, propose solutions to new problems, and make this knowledge available to us" (Hawkins, 209). In addition, Hawkins is also highly influenced in his work by Mountcastle's research, and some of his theories are grounded in the correctness of Mountcastle's theories. Hawkins believes that the different areas of your brain that learn (i.e. eyes for learning to see), learn in the exact same way. This means that there is potential to find an algorithm that's able to represent the foundations of learning, and by extension his view and definition of intelligence boils down to this one-learning algorithm.

\par In conclusion, a lot of different individuals have studied intelligence in a lot of different aspects, and for a lot of different applications. Understanding human intelligence, what it means, and how to measure it has been a pretty central problem in the past, and there is still active research on understanding it now.

\noindent 

\begin{workscited}
\bibent \\
\bibent Hawkins, Jeff, and Sandra Blakeslee, On Intelligence. New York: Henry Holt, 2005. Print. \\
\bibent Darwin, Charles. and Barlow, Nora, The autobiography of Charles Darwin. London: Collins, 1958. Print. \\
\bibent Binet. Alfred, and Simon, Theodore, The development of intelligence in children. Baltimore, Williams \& Wilkins, 1916. \\
\bibent Seligman, D. (2002). Good breeding. National Review, 1916.
Print. \\
\end{workscited}
\end{document}