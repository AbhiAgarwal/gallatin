\documentclass[11pt, oneside]{article}
\usepackage[a4paper]{geometry}
\geometry{letterpaper}
\usepackage{amssymb}
\linespread{2.3}
\usepackage[affil-it]{authblk} 
\usepackage{etoolbox}
\usepackage{lmodern}
\usepackage{ifpdf}
\usepackage{mla}
\usepackage{blindtext}
\usepackage[english]{babel}
\usepackage{fancyhdr}

\renewcommand\Authfont{\fontsize{12}{14.4}\selectfont}
\renewcommand\Affilfont{\fontsize{9}{10.8}\itshape}

\setlength{\voffset}{-0.7in}
\setlength{\headsep}{6pt}

\title{Exploring historically how intelligence has been defined, and why people have attempted to define it\vspace{-0.4cm}}
\author{Abhi Agarwal\vspace{-1cm}}
\date{}

\begin{document}

\maketitle

% Is there one trait intelligence boils down to?
% Should there be one trait? Should we be looking for that one trait?

\par Most of the history of the study of intelligence is dominated by psychologists. From Francis Galton to Alfred Binet to Cyril L. Burt to Howard Gardner.

% Talk about historical foundations of: Plato, Aristotle, Thomas Aquinas, Juan Huarte, Jean-Marc Gaspard Itard, Charles Darwin (very important for some reason).
% http://www.intelltheory.com/map.shtml
% http://www.intelltheory.com/darwin.shtml 
% http://www.gsjournal.net/old/babin/intro1.htm

% Perhaps have a labeled diagram of the most influential intelligence thinkers and label them by their fields. I want to explore the intentions and measures of what other people in other fields talked about intelligence. 

% Jean-Martin Charcot (French Neurologist)
% https://en.wikipedia.org/wiki/Jean-Martin_Charcot
% Teacher of Freud, Binet

% Alfred Binet (1857)
% https://en.wikipedia.org/wiki/Alfred_Binet

\par Alfred Binet

% Charles Spearman (1863)
% https://en.wikipedia.org/wiki/Charles_Spearman

\par Charles Spearman

% Jean Piaget (maybe)
% https://en.wikipedia.org/wiki/Jean_Piaget#Artificial_intelligence

% 

% Vernon Mountcastle (1918) paper in 1957
% https://en.wikipedia.org/wiki/Vernon_Benjamin_Mountcastle

\par Vernon Mountcastle 
\par His work was not directly in the field of intelligence research, but in cognition and in the cerebral cortex. These are 

% Howard Gardner (1943) paper in 1983
% https://en.wikipedia.org/wiki/Howard_Gardner
% https://en.wikipedia.org/wiki/Theory_of_multiple_intelligences

\par Howard Gardner, in his book `Frames of Mind: The Theory of Multiple Intelligences', formulates eight particular behavior that must be observed in order to be considered intelligent. 

% Jeff Hawkins (1957)
% https://en.wikipedia.org/wiki/Jeff_Hawkins
% https://en.wikipedia.org/wiki/On_Intelligence

\par Jeff Hawkins, a well known technologist, develops his theory, from the background of a Electrical Engineer turned Biophysicist turned self-taught Neuroscientist, in his book `On Intelligence'. Hawkins' aim to measure and quantify intelligence is to be able to re-build it in the form of an Artificially Intelligent agent. This is fundamentally different from the aims of many other researcher seeking to define and measure intelligence. The distinction is important since one seeks to explore or quantify intelligence for philosophical or experimentation purposes, and one seeks to go a step further and make attempts to build it. In addition, the theoretical framework or worldview of being able to build an intelligent machine only became possible after the formation of the fields of Artificial Intelligence and Computer Science.

\par Hawkins' theory is solidified on the idea that intelligence is just learning to see repetitions in patterns, and does not constitute of the ability to do any specific tasks. Hawkins writes ``the intelligent machine must learn via observation of its world, including input from an instructor when necessary. Once our intelligent machine has created a model of its world, it can then see analogies to past experiences, make predictions of future events, propose solutions to new problems, and make this knowledge available to us" (Hawkins, 209). In addition, Hawkins is also highly influenced in his work by Mountcastle's research, and some of his theories are grounded in the correctness of Mountcastle's theories.

\par ``Intelligence is measured by the predictive ability of a hierarchical memory, not by humanlike behavior" (Hawkins 210). ``Intelligent machines will have the equivalent of a cortex and a set of senses, but the rest is optional" (Hawkins, 208). 

\noindent 

\begin{workscited}
\bibent \\
\bibent Hawkins, Jeff, and Sandra Blakeslee, On Intelligence. New York: Henry Holt, 2005.
Print. \\
\end{workscited}
\end{document}