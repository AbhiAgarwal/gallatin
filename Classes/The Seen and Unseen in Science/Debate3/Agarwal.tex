\documentclass[11pt, oneside]{article}
\usepackage[a4paper]{geometry}
\geometry{letterpaper}
\usepackage{amssymb}
\linespread{2.2}
\usepackage[affil-it]{authblk} 
\usepackage{etoolbox}
\usepackage{lmodern}
\usepackage{ifpdf}
\usepackage{mla}
\usepackage{blindtext}
\usepackage[english]{babel}
\usepackage{fancyhdr}

\pagestyle{fancy}
\fancyhead[LE]{}
\fancyhead[RO]{}
\fancyhead[RE]{}
\fancyhead[LO]{}
\cfoot{\thepage}

\renewcommand\Authfont{\fontsize{12}{14.4}\selectfont}
\renewcommand\Affilfont{\fontsize{9}{10.8}\itshape}

\setlength{\voffset}{-0.7in}
\setlength{\headsep}{6pt}

\title{Debate: Scientific representation\vspace{-0.4cm}}
\author{Abhi Agarwal\vspace{-1cm}}
\date{}

\begin{document}

\maketitle

\noindent \textbf{Side A}: Trust the picture: the decision not to drop the bomb was made based on a picture produced by significant technological and scientific expertise, and is therefore trustworthy.

% Dumit, Tufte, Voss
% We don't have a lot to trust here. We have to trust in technology and scientific representation and knowledge in order for us to trust and make a decision. In the same way we trust PET scans to help us, we must trust technology. "New Seeing-Eye Machines ... look inside your body, can save your life" (161), and "medical imaging promised to provide early warnings of the onset of mental illness, one of the largest problems in its treatment and prevention" (Dumit, 147). There still exists doubt, though: "I am fascinated and horrified by the possibility posed here, of a world in which technology can tell me who I truly am". Since this object is unknown and foreign the only tools we have are the ones we know and have validated. We have to place our beliefs in things we know and utilize them to solve this issue. There's no other feasible way to do this.

% A lot of decisions we make everyday are now based on technology we trust, and reports we generate using that technology. Decisions such as if a particular country is developing nuclear technology. We use technology to know and be able to visualize those things. In addition, we also use technology to be able to understand if people are carrying guns into airports, etc. This is at a small scale, but technology builds upwards. Meaning that a piece of technology builds on different parts, and if one combines parts to build newer technology then you have to trust the reliability of the previous technologies. 

% Tufte would say that this is pure data, and the scientific representation of this picture is the best plotted data. This is since this has been generated purely by a pathogen and presented to us directly. 

\par An initial reason to trust the picture is the fact that dropping the bomb will actually accelerate the growth and spread of the alien pathogen, which could inevitability harm us and put is in a worse situation.  

% The fact that the alien virus will actually accelerate spread and grow if we drop the bomb. 

% this was their second attempt at making a decision on this. They have clearly thought about the different options, and this new knowledge was clearly a game changer.

\newpage

% Myers, Tucker

\noindent \textbf{Side B}: Drop the bomb: a scientific and technological picture is not sufficient grounds for making a decision of this magnitude.

% Introduction

\par An initial reason to doubt the reliability of their progress is the fact that they are indecisive, and they have changed their decision twice. They did not have certainty in their own hypothesis, which lead them to explore different options. This reduces the reliability because it implies that they might still not be definitive if they are correct or not, and the technology or representation they have used was not successful in giving them a correct conclusion initially. This could imply many different faults in the technology and the representation, and without exploring these faults they might not be able to fully validate their results.

% First time they are seeing something like this, and their personal judgement cannot be trusted

\par It is the first time that they have observed a pattern, or an alien pathogen, like this, and therefore their judgement based on their initial observation on the structure can not be trusted. Myers, in his paper `Illustrations in sociobiology', writes ``it is not till the same phenomena repeat themselves in the same, in the same place, a great number of times, that the observer learns to trust these impressions" (Myers, 50). The technology and the knowledge of the scientists is very limited for this scenario, and they do not have the theoretical framework to make a claim to not drop the bomb that could potentially kill millions of people. It is the first time that any of them have seen a pattern as such.

% repeated experimentation or further validation. They witnessed using the photography, but they haven't played detective yet.

\par In the scene they observed the pattern once, and make a decisive conclusion based on their one observation. They did not repeat the experiment to validate whether they would see the same kind of multiplication from the alien pathogen or if it always proved the same pattern. Tucker writes, ``a photographer must 'prove his photograph' by expounding the manner in which it had been made before the image could be admitted as a matter of fact" (Tucker, 379). The photograph in this case is the moment where the duplication occurs. Proof in scientific representation is important, and it is vital to have more data and more observations in order to prove this. They have not repeatedly observed this pattern `a great number of times'.

% The technology cannot be trusted as it might not be a reliable form for alien patogens

\par Moreover, they are attempting to use scientific representation and techniques they know to study and observe patterns on items that might not follow the same patterns. Since this is an alien pathogen we can not conclude and make decisions solely based on the technology we have. Tucker writes, ``like photographic practices in meteorology and bacteriology, spirit photography highlighted issues of witnessing, detecting, and deceiving" (Tucker, 402). Similar to spirit photography, this meteorite comes from an area we have not explored in the past or have significant experience in, and the contents of the meteorite are also unknown to us. Utilizing our instruments to help witness and detect what is in that material might not work, and becomes problematic if we trust it completely. The technology was created, tested, and validated to items found on Earth. In the scene they have not validated the use of their technology on these foreign objects. They also can not be completely sure since they have very limited ways of validating their technology against unknown objects.

% Conclusion

\par For the sake of argument we will not drop the bomb, and run more tests to validate if our understanding and the conclusion we made are accurate. 

\begin{workscited}
\bibent \\
\bibent Voss, Julia. ``Darwin's pictures."  \textit{Darwin's pictures}.  2010. Print. \\
\bibent Myers, Greg. ``Illustrations in sociobiology."  \textit{Illustrations in sociobiology}.  1988. Print. \\
\bibent Lane, Maria. ``Geographies of Mars."  \textit{Geographies of Mars}.  2010. Print. \\
\bibent Tucker, Jennifer. ``Photography as witness."  \textit{Photography as witness}.  1997. Print. \\
\bibent Dumit, Joe. ``Picturing personhood."  \textit{Picturing personhood}.  2004. Print. \\
\bibent Tufte, Edward. ``Visual Display."  \textit{Visual Display}.  2001. Print. \\
\end{workscited}
\end{document}