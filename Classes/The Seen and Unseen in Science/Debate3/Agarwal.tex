\documentclass[11pt, oneside]{article}
\usepackage[a4paper]{geometry}
\geometry{letterpaper}
\usepackage{amssymb}
\linespread{2.2}
\usepackage[affil-it]{authblk} 
\usepackage{etoolbox}
\usepackage{lmodern}
\usepackage{ifpdf}
\usepackage{mla}
\usepackage{blindtext}
\usepackage[english]{babel}
\usepackage{fancyhdr}

\pagestyle{fancy}
\fancyhead[LE]{}
\fancyhead[RO]{}
\fancyhead[RE]{}
\fancyhead[LO]{}
\cfoot{\thepage}

\renewcommand\Authfont{\fontsize{12}{14.4}\selectfont}
\renewcommand\Affilfont{\fontsize{9}{10.8}\itshape}

\setlength{\voffset}{-0.7in}
\setlength{\headsep}{6pt}

\title{Debate: Scientific representation\vspace{-0.4cm}}
\author{Abhi Agarwal\vspace{-1cm}}
\date{}

\begin{document}

\maketitle

\noindent \textbf{Side A}: Trust the picture: the decision not to drop the bomb was made based on a picture produced by significant technological and scientific expertise, and is therefore trustworthy.

% Dumit, Tufte, Voss

% The fact that the alien virus will actually accelerate spread and grow if we drop the bomb. 

% A lot of decisions we make everyday are now based on technology we trust, and reports we generate using that technology. Decisions such as if a particular country is developing nuclear technology. We use technology to know and be able to visualize those things. In addition, we also use technology to be able to understand if people are carrying guns into airports, etc. This is at a small scale, but technology builds upwards. Meaning that a piece of technology builds on different parts, and if one combines parts to build newer technology then you have to trust the reliability of the previous technologies. 

\newpage

\noindent \textbf{Side B}: Drop the bomb: a scientific and technological picture is not sufficient grounds for making a decision of this magnitude.

% Myers, Tucker
% "It must not be imagined that any drawing represents what the observer sees the moment he looks through the telescope. Instants of exceptional seeing flash out, here and there, at different spots on the planet. It is not till the same phenomena repeat themselves in the same, in the same place, a great number of times, that the observer learns to trust these impressions. One has to keep one's mind constantly at the highest pitch to catch and retain what the eye sees" (50).

\begin{workscited}
\bibent \\
\bibent Voss, Julia. ``Darwin's pictures."  \textit{Darwin's pictures}.  2010. Print. \\
\bibent Myers, Greg. ``Illustrations in sociobiology."  \textit{Illustrations in sociobiology}.  1988. Print. \\
\bibent Lane, Maria. ``Geographies of Mars."  \textit{Geographies of Mars}.  2010. Print. \\
\bibent Tucker, Jennifer. ``Photography as witness."  \textit{Photography as witness}.  1997. Print. \\
\bibent Dumit, Joe. ``Picturing personhood."  \textit{Picturing personhood}.  2004. Print. \\
\bibent Tufte, Edward. ``Visual Display."  \textit{Visual Display}.  2001. Print. \\
\end{workscited}
\end{document}