\documentclass[11pt, oneside]{article}
\usepackage[a4paper]{geometry}
\geometry{letterpaper}
\usepackage{amssymb}
\linespread{2.2}
\usepackage[affil-it]{authblk} 
\usepackage{etoolbox}
\usepackage{lmodern}
\usepackage{ifpdf}
\usepackage{mla}
\usepackage{blindtext}
\usepackage[english]{babel}
\usepackage{fancyhdr}

\pagestyle{fancy}
\fancyhead[LE]{}
\fancyhead[RO]{}
\fancyhead[RE]{}
\fancyhead[LO]{}
\cfoot{\thepage}

\renewcommand\Authfont{\fontsize{12}{14.4}\selectfont}
\renewcommand\Affilfont{\fontsize{9}{10.8}\itshape}

\setlength{\voffset}{-0.7in}
\setlength{\headsep}{6pt}

\title{Debate: Things unseen\vspace{-0.4cm}}
\author{Abhi Agarwal\vspace{-1cm}}
\date{}

\begin{document}

\maketitle

\noindent \textbf{Side A}: Obi-wan: The Force is an energy field generated by all living things, it surrounds us, penetrates us, it binds the galaxy together.

\par In `Star Wars: A New Hope', Obi-wan (and other characters) continually senses disturbances in the Force when catastrophic events happen, and these feelings always end up being valid. When living things die, there is a change in the Force, and to Obi-wan and the Force-sensitives, this change evokes some kind of feeling or emotion. This implies that there must exist some power that allows them to feel changes in life even across galaxies. Obi-wan calls this energy the Force, and it is what allows the Jedi Order to protect `The Galaxy' from the Sith Order. Moreover, in the Star Wars universe there are also Force Ghosts, who are deceased Force-sensitives who can still interact with the living. This solidifies the existence of the Force as it shows the true binding power of the Force (in this case to people who can control it). In addition, both the Jedi Order and the Sith Order gain from the existence of the Force, thereby reducing the motivation for either of the orders to falsely promote its existence to increase fear.

\par Han not having observed or experienced the Force does not imply that it does not exist. Bartusiak, in her book `Through a Universe Darkly', wrote ``Rubin realized that a huge reservoir of extra material, invisible to her telescope, must be tucked away somewhere to keep the stars from flying out of the galaxy" (Bartusiak, 212). Rubin, through her research, showed that there was invisible matter which contained enough mass to alter and shape the universe we live in. Dark matter, in a way, binds our galaxies together. In a similar way, the Force binds `The Galaxy' together. The Force described does not have to consist of matter or be an energy in the traditional sense, but can be an energy that binds life forms together in some other metaphysical way. The closest representation of Force to traditional forces would be to observe changes in entropy. In the Star Wars universe, Force was discovered and then used to keep order in `The Galaxy'. Bartusiak writes ``We now know that we were studying only the small fraction of it that is luminous" (Bartusiak, 216). Han only sees the luminous while Obi-wan sees beyond just the luminous.

\par Experiencing the Force by being a Force-sensitive is not necessary; simply an observation is enough to realize its existence. Rubin did not directly feel or experience the `extra material', but she observed it. She observed it through a medium, in this case the rotation of a galaxy, and in the same way an individual is able to observe the Force through a source. The individual can overlook while a member of the Jedi Order is using the physical power of the Force against an opponent. Han lacks this observation, and it is valid for him to question or deny the existence of the Force, simply because he has never encountered it in his past. In addition, it is plausible for Han to have never observed the force as the number of Force-sensitive beings is very low, and people are able to go through their lives without ever seeing one. But, his lack of observation should not rule out the existence of the Force.

\par Planck described this as a question in his essay `Unity of the physical world-picture'. Planck writes ``[the] constant element, independent of every human (and indeed of every intellectual) individuality, is what we call `the Real'. Or is there today a single physicist worthy of serious consideration who doubts the reality of the energy principle?" (Planck, 25). In this case, the `Real' is the Force, the physicist is the non-believer, and the principle is a field made up of living things. This brings up the question: is it worth discussing the existence of the Force with Han, who is neither a Jedi nor has ever observed the Force in action? Planck would agree that it is not worth discussing. It is easy to deny the existence of the Force. Planck writes ``a single glance into a precision laboratory shows us how many experiences and abstractions are comprised in one such measurement, simple as it may appear" (Planck, 24). There is a lot of abstraction in the idea of the Force, so one must observe it in order to believe in it.

\newpage 

\noindent \textbf{Side B}: Han Solo: There is no mystical energy field that controls my destiny. It is all a lot of simple tricks and nonsense. I call it luck. 

\par From a statistical standpoint, the number of people who have directly experienced, felt, or observed this mystical energy, or as they call it `the Force', is very small (as noted in `Star Wars: A New Hope'). This reduces the reliability of those who believe in the existence of the Force and it increases doubt. It questions the existence of the Force, and suggests that it could be an elaborate ruse by a small group of people. In addition, like Skywalker, in order to learn to harness the power of the Force one needs to be trained; this training could be to learn the tricks behind this mystical energy field.

\par In `The Science of Mechanics', Mach postulates the theory that ``there is no reality apart from our own impressions, and all natural science is in the last resort merely an economical adaptation of our thoughts to our impressions" (Mach, Principles, 23). From Mach's perspective it is valid for Han to not believe in the Force as there is no observable proof that the Force exists. If Han does not have his own impressions of Force then he should not have to believe in it. Furthermore, Mach would argue that it is not possible to observe a energy field generated by living things that somehow binds the galaxy together since it is not a `real element'. Mach writes ``the dividing line between the physical and the psychical is purely practical and conventional one; the only `real' elements in the world are the impressions" (Mach, Principles, 23). Everything we have observed about the Force has been psychical.

\par There has not been a clear explanation of how exactly the Force works, and how it is able to penetrate through individuals. The lack of detail reduces reliability that the Force exists, and allows for doubt. For something that binds the galaxy together, there must be science that is clearer on the power behind it, and greater knowledge of it. There are many unanswered questions about the Force, which lead to the questioning of its existence. In `Star Wars: A New Hope', Obi-Wan is able to use telekinesis to misguide the Stormtroopers. The Force has abilities such as hypnosis, levitation, and many more that not explained by existence of this one power. 

\par Moreover, Mach argues that ``we fill out the gaps in experience by the ideas that experience suggests" (Mach, Mechanics, 588). In the Star Wars universe the Force was, supposedly, discovered by a group of individuals, and then the knowledge was passed down to an exclusive society. Obi-wan (and members of the Jedi Order) are biased as the things they learnt about the Force were taught to them by their masters, and therefore their opinions and knowledge of the Force was shaped by them. For Obi-wan, the Force is an energy field that exists, and for him that is a constant that will forever exist. This discourages him from questioning the existence of the Force as it is the pinnacle to his world-view and, for him, his powers. However, the alternative to the theory of a single mystical energy field that binds the galaxy together is a world-view that suggests that the Force is actually a combination of different forces that act separately.

\par There does not have to be a singular mystical energy field called the Force; it can actually be attributed to many different individual theories. It is possible that the components that make up the Force each exist individually. Let us assume that the things that are attributed to the Force all exist. The Force then only exists as a heuristic hypothesis. Nye, in her work 'The Nineteenth Century Atomic Debates and the Dilemma of an `Indifferent Hypothesis'', describes heuristic hypothesis as an ``experiment which counts, where experiment is playing an on-going role in an open-ended pattern of research" (Nye, 250), and ``the heuristic hypothesis has no existential significance, but is only suggestive of further experiments, of new observations or information in the articulation of a programme of inquiry" (Nye, 250).   Applying Nye's reasoning allows us to explore the attributes that supposedly make up the Force, and experiment them individually. In addition, from Han's point of view ``the hypothesis is unconfirmed" (Nye, 250), which allows him to believe that the Force does not exist, but allows him to believe that each of its attributes can.

\par For the sake of argument we will consider the Force to exist. From the perspective of the Star Wars universe it is necessary for the Force to exist. It allows there to exist a power that helps govern the Galaxy, and keep order. However, as Mach would argue, there are a lot of attributes to the Force that are not explained by the current framework therefore we should not form a consensus until we postulate a better framework.

\begin{workscited}
\bibent \\
\bibent Mach, Ernst. ``The Science of Mechanics."  \textit{The Science of Mechanics}.  1919. Print. \\
\bibent Planck, Max. ``Unity of the physical world-picture."  \textit{Unity of the physical world-picture}.  1909. Print. \\
\bibent Mach, Ernst. ``Guiding principles of my scientific theory of knowledge."  \textit{Guiding principles of my scientific theory of knowledge}.  1910. Print. \\
\bibent Nye, Mary Jo. ``The Nineteenth Century Atomic Debates and the Dilemma of an `Indifferent Hypothesis'."  \textit{The Nineteenth Century Atomic Debates and the Dilemma of an `Indifferent Hypothesis'}.  1976. Print. \\
\bibent Adams, Henry.  ``The Dynamo and the Virgin."  \textit{The Education of Henry Adams}.  1900. Print. \\
\bibent Tyndall, John.  ``Scientific Use of the Imagination."  \textit{Scientific Use of the Imagination}.  1872. Print. \\
\bibent Gould, Stephen J.  ``Mismeasure of Man."  \textit{Mismeasure of Man}.  1981. Print. \\
\bibent Freud, Sigmund.  ``Introductory Lectures on Psycho-Analysis."  \textit{Introductory Lectures on Psycho-Analysis}.  1916. Print. \\
\bibent Bartusiak, Marcia.  ``Through a Universe Darkly."  \textit{Through a Universe Darkly}.  1993. Print. \\
\bibent Lucas, George. ``Star Wars: A New Hope."   \textit{Star Wars: A New Hope}. 20th Century Fox, 2004. DVD.
\end{workscited}

\end{document}