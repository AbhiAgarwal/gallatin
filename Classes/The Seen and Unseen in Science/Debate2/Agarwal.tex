\documentclass[11pt, oneside]{article}
\usepackage[a4paper]{geometry}
\geometry{letterpaper}
\usepackage{amssymb}
\linespread{2}
\usepackage[affil-it]{authblk} 
\usepackage{etoolbox}
\usepackage{lmodern}
\usepackage{ifpdf}
\usepackage{mla}
\usepackage{blindtext}

\makeatletter
\patchcmd{\@maketitle}{\LARGE \@title}{\fontsize{16}{19.2}\selectfont\@title}{}{}
\makeatother

\renewcommand\Authfont{\fontsize{12}{14.4}\selectfont}
\renewcommand\Affilfont{\fontsize{9}{10.8}\itshape}

\setlength{\voffset}{-1.1in}
\setlength{\headsep}{5pt}

\title{Debate: Things unseen\vspace{-0.4cm}}
\author{Abhi Agarwal\vspace{-1cm}}
\date{}

\begin{document}
\maketitle

\noindent \textbf{Side A}: Obi-wan: The Force is an energy field generated by all living things, it surrounds us, penetrates us, it binds the galaxy together.

\par In `Star Wars: A New Hope', Obi-wan (and other characters) continually senses disturbances in the Force when catastrophic events happen, and these feelings always end up being valid. When living things die, there is a change in the Force and to Obi-wan and the Force-sensitives, this change evokes some kind of feeling or emotion. This implies that there must exist some power that allows them to feel changes in life even across galaxies. Obi-wan calls this energy the Force, and it is what allows the Jedi Order to protect `The Galaxy' from the Sith Order. In addition, both the Jedi Order and the Sith Order gain from the existence of the Force so it reduces the motivation for either of the orders to falsely promote its existence to increase fear.

\par Han not having observed or experienced the Force does not imply that it does not exist. Bartusiak, in her book `Through a Universe Darkly', wrote ``Rubin realized that a huge reservoir of extra material, invisible to her telescope, must be tucked away somewhere to keep the stars from flying out of the galaxy" (Bartusiak, 212). Rubin, through her research, showed that there was matter that was invisible, contained mass, and contained enough mass to alter and shape the universe live in. Dark matter, in a way, binds our galaxies together. In a similar way, the Force binds `The Galaxy' together. The Force described does not have to consist of matter or be an energy in the traditional sense, but can be an energy that binds life forms together in some other metaphysical way. The closest representation of Force to traditional forces would be to observe changes in entropy. In the Star Wars universe, Force was discovered then used to keep order in `The Galaxy'. Bartusiak writes ``We now know that we were studying only the small fraction of it that is luminous" (Bartusiak, 216). Han only sees the luminous while Obi-wan sees beyond just the luminous.

\par Experiencing the Force, by being a Force-sensitive, is not necessary; simply an observation is enough to realize its existence. Rubin did not directly feel or experience the `extra material', but she observed it. She observed it through a medium, in this case the rotation of a galaxy, and in the same way an individual is able to observe the Force through a source. The individual can overlook while a member of the Jedi Order is using the physical power of the Force against an opponent. Han lacks this observation, and it is valid for him to believe that the Force does not exist simply because he has never encountered it in his past. But, his lack of observation should not determine the existence of the Force. In addition, it is plausible for Han to have never observed the force since the number of Force-sensitive beings is very low, and people are able to go through their lives without ever seeing one. 

\par 

\par Freud. what we see is only the surface and there exists a vast abyss within us or above us which we may never truly be able to see or explain. more to a galaxy than meets the eye

\par ``he could measure nothing except by chance collisions of movements imperceptible to his senses, perhaps even imperceptible to his instruments, but perceptible to each other, and so to some known ray at the end of the scale" (Adams, 2).

\newpage 

\noindent \textbf{Side B}: Han Solo: There is no mystical energy field that controls my destiny.  It is all a lot of simple tricks and nonsense. I call it luck. 

\par From a statistical standpoint the number of people who have directly experienced, felt, or observed the Force is a very small percentage. This reduces the reliability of the Force and it increases doubt. It is a trick played by a small group of people. In order to learn to harness the power of the Force you need to be trained; this training could be a training to learn the tricks behind the Force. In addition,

\par Creating a substance such as the Force could increase order and could be a way of understanding the universe.

\begin{workscited}
\bibent \\
\bibent Mach, Ernst. ``The Science of Mechanics."  \textit{The Science of Mechanics}.  1919. Print. \\
\bibent Planck, Max. ``Unity of the physical world-picture."  \textit{Unity of the physical world-picture}.  1909. Print. \\
\bibent Mach, Ernst. ``Guiding principles of my scientific theory of knowledge."  \textit{Guiding principles of my scientific theory of knowledge}.  1910. Print. \\
\bibent Nye, Mary Jo. ``The Nineteenth Century Atomic Debates and the Dilemma of an `Indifferent Hypothesis'."  \textit{The Nineteenth Century Atomic Debates and the Dilemma of an `Indifferent Hypothesis'}.  1976. Print. \\
\bibent Adams, Henry.  ``The Dynamo and the Virgin."  \textit{The Education of Henry Adams}.  1900. Print. \\
\bibent Tyndall, John.  ``Scientific Use of the Imagination."  \textit{Scientific Use of the Imagination}.  1872. Print. \\
\bibent Gould, Stephen J.  ``Mismeasure of Man."  \textit{Mismeasure of Man}.  1981. Print. \\
\bibent Freud, Sigmund.  ``Introductory Lectures on Psycho-Analysis."  \textit{Introductory Lectures on Psycho-Analysis}.  1916. Print. \\
\bibent Bartusiak, Marcia.  ``Through a Universe Darkly."  \textit{Through a Universe Darkly}.  1993. Print. \\
\bibent Lucas, George. ``Star Wars: A New Hope."   \textit{Star Wars: A New Hope}. 20th Century Fox, 2004. DVD.
\end{workscited}

\end{document}