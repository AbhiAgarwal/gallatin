\documentclass[11pt, oneside]{article}
\usepackage[a4paper]{geometry}
\geometry{letterpaper}
\usepackage{amssymb}
\linespread{2.4}
\usepackage[affil-it]{authblk} 
\usepackage{etoolbox}
\usepackage{lmodern}
\usepackage{ifpdf}
\usepackage{mla}
\usepackage{blindtext}
\usepackage[english]{babel}
\usepackage{fancyhdr}

\pagestyle{fancy}
\fancyhead[LE]{}
\fancyhead[RO]{}
\fancyhead[RE]{}
\fancyhead[LO]{}
\cfoot{\thepage}

\renewcommand\Authfont{\fontsize{12}{14.4}\selectfont}
\renewcommand\Affilfont{\fontsize{9}{10.8}\itshape}

\setlength{\voffset}{-0.7in}
\setlength{\headsep}{6pt}

\title{Debate: How a scientist sees\vspace{-0.4cm}}
\author{Abhi Agarwal\vspace{-1cm}}
\date{}

\begin{document}

\maketitle

\noindent \textbf{Side A}: Scientists see the world better than you.

\par In the movie, `Ghostbusters', there is a scene where in response to a question from the library administrator Dr. Peter Venkman says `Back off man, I'm a scientist' (Ghostbusters). The library administrator was unfairly questioning Venkman's scientific methodology. Since Venkman is a scientist he has expertise in the field, he should be allowed to question as he chooses since his knowledge of the supernatural is far greater than the library administrator's. Furthermore, this is validated by the premise that the library administrator has called upon Venkman and his team to solve this paranormal investigation that he could not have. The library administrator does not seem to have any knowledge about the supernatural, and seems like in shock that something similar to this is possible.

\par Pinch's theory-ladenness of observations, in his paper `Towards an analysis of scientific observation', suggests that externalities exist in Venkman's line of questioning that make it difficult to be understood by the untrained. Pinch writes ``the more external the [questioning], the more assumptions about the observational situation that must be included" (Pinch, 13). To the observer scientific methodology is a blackbox until it is unraveled and understood. In this particular circumstance the explanation behind Venkman's questions remains a blackbox for the library administrator. In addition, this could show that it was unfair for the library administrator to question him without knowing his intent.

\par Kuhn's views on perception, in his book `Structure of scientific revolutions', would add to Pinch's theory and suggest that understanding of something strongly depends on previous knowledge. Kuhn writes ``what a man sees depends both upon what he looks at and also upon what his previous visual-conceptual experience has taught him to see" (Kuhn, 113). In this case, Venkman has previously encountered similar scenarios while the library administrator has not, so his perception, or his theoretical framework, of the world is different to the library administrator's. In addition, Kuhn's theory on paradigms and the Duck-Rabbit example also suggests that the scientist and the library administrator could be seeing the world completely differently. The paradigms formed by the theoretical framework of the library administrator and Venkman are different, and therefore Kuhn suggests that their paradigms or theoretical frameworks would not be able to communicate. For the library administrator and Venkman to communicate, the library administrator would have to learn the scientific methodology and theories that Venkman utilizes to view the patient.

\par Moreover, trained judgment that the scientist has helps him be objective about his research. In the movie, the scientist also later utilizes technology to help be mechanically objective, and validate his hypothesis using technology. This validation helps him improve his understanding of the world and also improves his trained judgment, and allows him to better see the world in the future. Pinch writes ``science builds on what it already knows, even when its observational capabilities are concerned. It learns how to observe nature, and its ability to observe with increasing knowledge" (Pinch, 513). Trained judgment, mechanical objectivity, and truth-to-nature are three ways scientists have been objective in the past, and this sense of needing to be objective about their understanding of the world helps them see the world better since they have to continuously question their beliefs and validate them through some means. 

\par In addition, Shapere makes the statement that scientist `observe' the world whereas a normal person `sees' the world. Observation utilizes `the current state of physical knowledge' to help expand scientific theories. For example, no one has seen the inside of the sun, but scientists have been able to observe it using their trained judgment and observational skills and ``freed observation from? the unreliability of ordinary ones".

\par Scientists unravel blackboxes that most people abstract away in order to understand the world better. In the scene, the library administrator has no knowledge of the supernatural, but the scientist has unraveled the blackbox and has done significant research on the topic. The scientist has researched in many different topics that both the library administrator and the scientist know to exist, and have explored them in more depth. This implies that the scientist has a greater knowledge and a better perception of the world since he understands beneath the layer of abstraction while the library administrator does not. By unravelling these black boxes the scientist is able to see the world better.

\noindent \textbf{Side B}: Scientists do not see the world better than you.

% Lost their jobs at Columbia University. Could mean they're failed scientists. 
% What does it mean to see the world better than someone else.

\par No scientists do not see the world better, but they have developed their own system of scientific observation. Daston and Galison, in their book `Objectivity', write ``the trained expert (Doctor, Physicist, Astronomer) grounds his or her knowledge in guided experience, not special access to reality" (Daston and Galison, 359). They have increased experience in their own subject matter, but they do not have a special vision or better view of reality. They just have a different vision, and perhaps a more theoretical vision, of reality.

\par Furthermore, Pinch describes that scientists have to make assumptions in order to make claims. The validity of their claims depend on a lot of assumptions that have been created by other scientists in the past and by themselves. Kuhn adds ``the activity in which most scientists inevitably spend almost all their time, is predicated on the assumption that the scientific community knows what the world is like" (Kuhn, 5). This shows that scientists have a specialized skill set, which allows them to use knowledge from the past in order to make claims, but individually, like everyone else, they have to make assumptions and black box a lot of presumed knowledge. 

\par A lot of what shapes a scientist's world-view is assumptions they make and theories they understand. Similarly to a non-scientist, scientists can still make incorrect assumptions, and their theories can mislead and be incorrect. This implies that scientists could have a stricter vision of the world as their vision depends on past theories and observations made by the scientific community, but this is not a better view. In addition, the argument of a scientist seeing the world better by knowing scientific knowledge up to now is similar to the Whigs history paradigm. The Whigs party believed that their view of the world was better because history has built up to it. 

\par Culturally, scientists in their scientific work and research are sometimes influenced by external sources or their personal biases. Cartwright, in her book `Screening the body', uses the two videos `Highlights and Shadows' (155) and `Mass Radiography' (157) to portray this. In both of these promotional health videos healthiness is described as a young white women. Scientists used their medical gaze to further the stereotype of the healthy white women. Cartwright writes, ``I have argued that the press's representation of the X ray as sexualized spectacle and as a new mode of illicit looking was not at all a misreading of the work of science but a forgrounding of the fact that visual pleasure, sexual desire, and the thrill of mortality were not just incidental to the male radiologist's conquest of the inner body" (Cartwright, 154), which shows that scientists do not see the world better as, like us, they are perceptible to their own biases. This opposes to the purpose objectivity in science is supposed to serve; Galison and Daston define the concept of objectivity as supposed to minimize "intervention in hopes of achieving an image untainted by subjectivity" (Daston and Galison, 43). 

\par Finally, it can be said that scientists have a specialized way of seeing the world. They are trained to be objective with their research, and utilize past knowledge to support their hypothesis. It is difficult to say if that view is either better or worse than the view of non scientists. From Cartwright's readings it seems clear that gender and sexuality plays a role in science, and there are a lot of biases one does not know about that influence their ideas. In addition, it is also incredibly difficult to say what constitutes a world-view that is `better' since we all have our own subjective opinions and theoretical frameworks on what a world-view is. Historically, the media depended on scientists to have an objective understanding of the world, a `better' understanding, but scientists were still biased. A more informed view or a more specialized view does not mean that it is a better view. 

\begin{workscited}
\bibent \\
\bibent Reitman, Ivan, dir. Ghostbusters. Columbia Pictures, 1984. Film. \\
\bibent Pinch, Trevor. ``Towards an analysis of scientific observation."  \textit{Towards an analysis of scientific observation}.  1985. Print. \\
\bibent Shapere, Dudley. ``Concept of Observation in Science and Philosophy."  \textit{Concept of Observation in Science and Philosophy}.  1982. Print. \\
\bibent Kuhn, Thomas S. ``Structure of scientific revolutions."  \textit{Structure of scientific revolutions}.  1962. Print. \\
\bibent Cartwright, Lisa. ``Screening the body."  \textit{Screening the body}.  1995. Print. \\
\bibent Daston, Lorraine and Galison, Peter. ``Objectivity."  \textit{Objectivity}.  2007. Print. \\
\end{workscited}
\end{document}