\documentclass[11pt, oneside]{article}
\usepackage[a4paper]{geometry}
\geometry{letterpaper}
\usepackage{amssymb}
\linespread{2.3}
\usepackage[affil-it]{authblk} 
\usepackage{etoolbox}
\usepackage{lmodern}
\usepackage{ifpdf}
\usepackage{mla}
\usepackage{blindtext}
\usepackage[english]{babel}
\usepackage{fancyhdr}

\pagestyle{fancy}
\fancyhead[LE]{}
\fancyhead[RO]{}
\fancyhead[RE]{}
\fancyhead[LO]{}
\cfoot{\thepage}

\renewcommand\Authfont{\fontsize{12}{14.4}\selectfont}
\renewcommand\Affilfont{\fontsize{9}{10.8}\itshape}

\setlength{\voffset}{-0.7in}
\setlength{\headsep}{6pt}

\title{Debate: How a scientist sees\vspace{-0.4cm}}
\author{Abhi Agarwal\vspace{-1cm}}
\date{}

\begin{document}

\maketitle

% In video they are wearing scientific equipment: stethoscope.
% They are somewhat arrogant and dominating in the situation. Dropped the book on the table and created the disruption in the library. Makes it seem like they are important. 
% Talked about previous experience in undersea sponges?
% Those questions are quite standard when you are going into an examination with a doctor, though. 

\noindent \textbf{Side A}: Scientists see the world better than you.

\par In the movie, `Ghostbusters', there is a scene where in response to a question from the library administrator Dr. Peter Venkman says `Back off man, I'm a scientist' (Ghostbusters). The library administrator was unfairly questioning Venkman's scientific methodology. Since Venkman is a scientist he has expertise in the field, and should be allowed to question as he chooses since his knowledge of the supernatural is far greater than the library administrator's. Furthermore, this is validated by the premise that the library administrator has called upon Venkman and his team to solve this paranormal investigation that he could not have.

% Establish different perception/knowledge
% he has a phd in parapsychology and psychology

\par Pinch's theory-ladenness of observations, in his paper `Towards an analysis of scientific observation', suggests that externalities exist in Venkman's line of questioning that make it difficult to be understood by the untrained. Pinch writes ``the more external the [questioning], the more assumptions about the observational situation that must be included" (Pinch, 13). To the observer scientific methodology is a black box until it is unraveled and understood. In this particular circumstance the explanation behind Venkman's questions remains a blackbox for the library administrator, and could imply the library administrator's view of the world being more limited than Venkman's.

\par Kuhn's views on perception, in his book `Structure of scientific revolutions', would add to Pinch's theory and suggest that understanding of something strongly depends on previous knowledge. Kuhn writes ``what a man sees depends both upon what he looks at and also upon what his previous visual-conceptual experience has taught him to see" (Kuhn, 113). In this case, Venkman has previously encountered similar scenarios while the library administrator has not, so his perception of the world is different to the library administrator's. 

\par In addition, Kuhn's theory on paradigms and the Duck-Rabbit example suggests that the scientist and the library administrator could be seeing the world completely differently. 

% Explain why it is better

\par Two paradigms cannot communicate.

\par Scientists unravel blackboxes that most people abstract away in order to understand the world better.

\newpage

\noindent \textbf{Side B}: Scientists do not see the world better than you.

% Lost their jobs at Columbia University. Could mean they're failed scientists. 
% What does it mean to see the world better than someone else.

\par Kuhn's theory of paradigms suggests that the paradigm supported by the scientist and the paradigm supported by the library administrator are incommensurable, which means that we can not compare them.

% Two paradigms are incommensurable. Cannot compare them? They both see the world differently but doesn't mean one is better.

\begin{workscited}
\bibent \\
\bibent Reitman, Ivan, dir. Ghostbusters. Columbia Pictures, 1984. Film.
\bibent Pinch, Trevor. ``Towards an analysis of scientific observation."  \textit{Towards an analysis of scientific observation}.  1985. Print. \\
\bibent Shapere, Dudley. ``Concept of Observation in Science and Philosophy."  \textit{Concept of Observation in Science and Philosophy}.  1982. Print. \\
\bibent Kuhn, Thomas S. ``Structure of scientific revolutions."  \textit{Structure of scientific revolutions}.  1962. Print. \\
\bibent Cartwright, Lisa. ``Screening the body."  \textit{Screening the body}.  1995. Print. \\
\bibent Daston, Lorraine and Galison, Peter. ``Objectivity."  \textit{Objectivity}.  2007. Print. \\
\end{workscited}
\end{document}