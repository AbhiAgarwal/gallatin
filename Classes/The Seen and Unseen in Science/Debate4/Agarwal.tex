\documentclass[11pt, oneside]{article}
\usepackage[a4paper]{geometry}
\geometry{letterpaper}
\usepackage{amssymb}
\linespread{2.3}
\usepackage[affil-it]{authblk} 
\usepackage{etoolbox}
\usepackage{lmodern}
\usepackage{ifpdf}
\usepackage{mla}
\usepackage{blindtext}
\usepackage[english]{babel}
\usepackage{fancyhdr}

\pagestyle{fancy}
\fancyhead[LE]{}
\fancyhead[RO]{}
\fancyhead[RE]{}
\fancyhead[LO]{}
\cfoot{\thepage}

\renewcommand\Authfont{\fontsize{12}{14.4}\selectfont}
\renewcommand\Affilfont{\fontsize{9}{10.8}\itshape}

\setlength{\voffset}{-0.7in}
\setlength{\headsep}{6pt}

\title{Debate: How a scientist sees\vspace{-0.4cm}}
\author{Abhi Agarwal\vspace{-1cm}}
\date{}

\begin{document}

\maketitle

% https://en.wikipedia.org/wiki/Ghostbusters

% In video they are wearing scientific equipment: stethoscope.
% They are somewhat arrogant and dominating in the situation. Dropped the book on the table and created the disruption in the library. Makes it seem like they are important. 
% Talked about previous experience in undersea sponges?
% Those questions are quite standard when you are going into an examination with a doctor, though. 

\noindent \textbf{Side A}: Scientists see the world better than you.

\par In the movie, `Ghostbusters', there is a scene where in response to a question from the library administrator Dr. Peter Venkman says `Back off man, I'm a scientist' (Ghostbusters). The library administrator was unfairly questioning Venkman's scientific methodology. Since Venkman is a scientist he has expertise in the field, and thus should be allowed to question as he chooses since his knowledge of the supernatural is far greater than the library administrator's. Furthermore, this is validated by the premise that the library administrator has called upon Venkman and his team to solve this paranormal investigation that he could not have. The library administrator does not seem to have any knowledge about the supernatural, and seems like in shock that something similar to this is possible.

% Establish different perception/knowledge
% he has a phd in parapsychology and psychology

\par Pinch's theory-ladenness of observations, in his paper `Towards an analysis of scientific observation', suggests that externalities exist in Venkman's line of questioning that make it difficult to be understood by the untrained. Pinch writes ``the more external the [questioning], the more assumptions about the observational situation that must be included" (Pinch, 13). To the observer scientific methodology is a blackbox until it is unraveled and understood. In this particular circumstance the explanation behind Venkman's questions remains a blackbox for the library administrator. In addition, this could show that it was unfair for the library administrator to question him without knowing his intent.

\par Kuhn's views on perception, in his book `Structure of scientific revolutions', would add to Pinch's theory and suggest that understanding of something strongly depends on previous knowledge. Kuhn writes ``what a man sees depends both upon what he looks at and also upon what his previous visual-conceptual experience has taught him to see" (Kuhn, 113). In this case, Venkman has previously encountered similar scenarios while the library administrator has not, so his perception, or his theoretical framework, of the world is different to the library administrator's. In addition, Kuhn's theory on paradigms and the Duck-Rabbit example also suggests that the scientist and the library administrator could be seeing the world completely differently. The paradigms formed by the theoretical framework of the library administrator and Venkman are different, and therefore Kuhn suggests that their paradigms or theoretical frameworks would not be able to communicate. For the library administrator and Venkman to communicate, the library administrator would have to learn the scientific methodology and theories that Venkman utilizes to view the patient.

% and could imply the library administrator's view of the world being more limited than Venkman's

% Explain why it is better
% science builds on what it already knows, even when its observational capabilities are concerned. It learns how to observe nature, and its ability to observe with increasing knowledge? (513, Pinch).

\par Scientists unravel blackboxes that most people abstract away in order to understand the world better. In the scene, the library administrator has no knowledge of the supernatural, but the scientist has unraveled the blackbox and has done significant research on the topic.

% Scientists have been trained to explore and understand the world. They have been trained to unravel the blackboxes that they find, and expand human knowledge. 

\par Trained judgment that the scientist has helps him be objective about his research. He also later utilizes technology to help be mechanically objective. Trained judgment helps him see the world better since he 

\newpage

\noindent \textbf{Side B}: Scientists do not see the world better than you.

% Lost their jobs at Columbia University. Could mean they're failed scientists. 
% What does it mean to see the world better than someone else.

\par Kuhn's theory of paradigms suggests that the paradigm supported by the scientist and the paradigm supported by the library administrator are incommensurable, which means that we can not compare them. ``the activity in which most scientists inevitably spend almost all their time, is predicated on the assumption that the scientific community knows what the world is like" (Kuhn, 5)

\par No scientists do not see the world better, but they have developed their own system of scientific observation. Allows them to keep moral ethics. Abstract away details. etc.

\par "the trained expert (Doctor, Physicist, Astronomer) grounds his or her knowledge in guided experience, not special access to reality" (Daston Galison, 359). They have increased experience in their own subject matter, but they don't have a special vision or better view of reality.

\par Culturally, scientist are the same as people. Seen by the Cartwright reading `Screening the body'. Scientists made similar mistakes to normal people in terms of exploiting a gender? The medical gaze or the stereotypical healthy white women. Scientists are morally equivalent to people. The scientist could have explained his reasoning in order to help the person.


% Two paradigms are incommensurable. Cannot compare them? They both see the world differently but doesn't mean one is better.

\begin{workscited}
\bibent \\
\bibent Reitman, Ivan, dir. Ghostbusters. Columbia Pictures, 1984. Film.
\bibent Pinch, Trevor. ``Towards an analysis of scientific observation."  \textit{Towards an analysis of scientific observation}.  1985. Print. \\
\bibent Shapere, Dudley. ``Concept of Observation in Science and Philosophy."  \textit{Concept of Observation in Science and Philosophy}.  1982. Print. \\
\bibent Kuhn, Thomas S. ``Structure of scientific revolutions."  \textit{Structure of scientific revolutions}.  1962. Print. \\
\bibent Cartwright, Lisa. ``Screening the body."  \textit{Screening the body}.  1995. Print. \\
\bibent Daston, Lorraine and Galison, Peter. ``Objectivity."  \textit{Objectivity}.  2007. Print. \\
\end{workscited}
\end{document}