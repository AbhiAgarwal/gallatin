\documentclass[11pt, oneside]{article}
\usepackage[a4paper]{geometry}
\geometry{letterpaper}
\usepackage{amssymb}
\linespread{2.2}
\usepackage[affil-it]{authblk} 
\usepackage{etoolbox}
\usepackage{lmodern}
\usepackage{ifpdf}
\usepackage{mla}
\usepackage{blindtext}
\usepackage[english]{babel}
\usepackage{fancyhdr}

\pagestyle{fancy}
\fancyhead[LE]{}
\fancyhead[RO]{}
\fancyhead[RE]{}
\fancyhead[LO]{}
\cfoot{\thepage}

\renewcommand\Authfont{\fontsize{12}{14.4}\selectfont}
\renewcommand\Affilfont{\fontsize{9}{10.8}\itshape}

\setlength{\voffset}{-0.7in}
\setlength{\headsep}{6pt}

\title{Project Proposal\vspace{-0.4cm}}
\author{Abhi Agarwal\vspace{-1cm}}
\date{}

\begin{document}

\maketitle

\noindent \textbf{Title}: Exploring historically how intelligence has been defined, and why people have attempted to define it.

\par Over the last couple years I have taken my classes that have revolved around intelligence, and aspects around it. I have taken classes on quantifying intelligence, uses of intelligence to make intelligent machines, and a couple other. I have never done an in-depth study of how people have defined intelligence, and why they have attempted to define intelligence. I would like to study it from different perspectives: Biologists, Physicists, Psychologists, Social workers, Historians, Computer Scientists, and any other perspectives I can get my hands on.

\par The common thread among them is that they all try and define intelligence, but they all define it for different reasons and with different aims. They also have their own unique perspectives on how they should go about measuring intelligence. I would love to explore these different perspectives, and find similarities and differences between their aims, definitions, and how they will go about measuring intelligence. In addition, also to explore what different people think constitutes or `makes' people intelligence - the characteristics that build it. 

\par Some of the sources/authors I will use are: On Intelligence (Jeff Hawkins), Vernon Mountcastle, An Essay Concerning Human Understanding (Peter Nidditch), Mainstream Science on Intelligence (Wall Street Journal), Alfred Binet, The Bell Curve (Richard J. Herrnstein), Louis Leon Thurstone, Lee A. Thompson, Frames of Mind: The Theory of Multiple Intelligences (Howard Gardner), Charles Spearman, Ruth Benedict, Sandra Scarr, and a few more. These are just broader sources, but I'll narrow them down as I do the readings and research into these individuals or these sources. My idea is to find authors and sources from each of the different perspectives or areas of research.

\end{document}