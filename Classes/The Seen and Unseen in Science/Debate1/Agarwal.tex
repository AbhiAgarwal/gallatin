\documentclass[11pt, oneside]{article}
\usepackage{geometry}
\geometry{letterpaper}
\usepackage{graphicx}
\usepackage{amssymb}
\linespread{1.82311}

\title{Debate: Things newly seen}
\author{Abhi Agarwal}
\date{}

\begin{document}
\maketitle

\par Professor Farnsworth has had many successes in the past. In the show, he has managed to create a clone of himself, invent the planet express-ship (that runs at 300\% fuel capacity), invent a Paradox-Free Time Travel machine, and is also a tenured professor at the Mars University who won a Fields Medal. In addition, he has also managed to save the world through his inventions across the different episodes. All of his inventions seem to have been impossible and beyond the realm of what was imaginable before they were invented, and so his experience of building and changing the world in the past help validate the idea that it is possible that he actually managed to build a Smell-O-Scope.

\par However, judging by his appearance in the clip (as well as previous knowledge of his character), the professor does not seem to be in his right mind. He had forgotten that he built the device a whole year ago, and is known to suffer from dementia. Consequently, it would be quite difficult to trust a scientific breakthrough of that calibre if the individual was of that age and more importantly, in that mental condition.

\par It is also useful to take into account the other people who tried to use the Smell-O-Scope; both Fry and Leela tried to use it. Leela's personality is often skeptical and she makes decisions based on logic, thereby suggesting that if she believed in the professor's discovery, then the audience would be equally as likely to believe in it. In addition, judging by the speed that Fry, Leela, and Bender run towards him with, it would appear that they are usually excited to see and find out about his discoveries and inventions. However, only Fry (other than the professor himself) was able to confirm that each planetary body has its own scent; Leela was only able to help validate the smell of the giant ball of garbage. 

\par Moreover, the fact that they used a computer and can map the trajectory of the giant ball of garbage does help to validate the idea that the Smell-O-Scope is a reliable device. They used computational methods to validate their hypothesis, using satellites and other scientific radars, to guarantee that there is a moving object in space, and that it is coming towards either (regardless of its smell and its composition). As we rely on computers in our day-to-day lives, and have been relying on it for decades to map trajectories for us, it is possible to extend and increase the trust on an instrument (the Smell-O-Scope) when we are able to use machines that we trust to verify its results.

\par On the contrary, the Smell-O-Scope has only been tested once, and we only know how to operate it, but not how it works. Additionally, as they are able to go into space, they could have checked the smells for other planets and see if they match with the Smell-O-Scope, but did not do so. To this point we have yet to see the ?golden event', which Galison defines as ?the single picture of such clarity and distinctness that it commands acceptance" (Galison, 360), or any statistical evidence where we have valid evidence that the Smell-O-Scope works. Through Galison's framework, we cannot conclude or make a judgement, as we have incomplete data. This suggests that we should postpone alerting the city until we confirm our doubts, in order to prevent a mass panic. Furthermore, it is not clear how they were able to determine how far away the giant ball of garbage was, as the Smell-O-Scope seemed to find the scent of the closest object in its path, and did not take into consideration the distance.

\par However, just because it has not been done before does not mean it is impossible. A lot of the disbelief in an invention comes from the ideology that if it was not possible to do something earlier, it is extremely unlikely that it is going to be possible now. There hasn't been a way to smell things from large distances, and the professor's discovery allows for this. Hooke's use of microscopes for scientific observations made ?the world...not what it seemed. [His observations] documented that (1) many 'invisible' things actually existed, and that (2) seeing objects microscopically disclosed radically new appearances" (Hardwood, 136). The professor's discovery could still be reliable and trusted as it could be possible that a device that smells objects from large distances can be created, even if the chances are low. 

\par Cohen would argue that instruments do not give us the absolute truth. Our brains can still augment or interfere with what we smell, even if we use the Smell-O-Scope. Cohen argues that ?an explorer going into a new land tends to ?see? what he expected to find, filling his reports with comparisons made between the new land and the lands with which the explorer was already familiar, either from personal experience or from reading and conversations" (Cohen, 459). In this situation, because Leela and Fry understood the function of the Smell-O-Scope they had a disposition to acknowledge that each planet had its own scent. This could interfere with their judgement as they knew what they were looking for. 

\par In addition, instruments don?t necessarily make our senses more reliable. They help us interpret the world in a better way, but we are still prone to misinterpreting the world even when using an instrument like the Smell-O-Scope. For example, the instrument could have been misaligned or the smell might have been coming from a gas explosion at the International Space Station. Fry and Leela don't understand the mechanics of how the Smell-O-Scope works, and the professor cannot even remember how he created it. None of them are able to reliably explain what happens in the period between the smell coming into the device and when it enters their nose or even how the smell is traveling to the device itself. The smell can be altered, and Bender could potentially create a panic in New York because they did not realize their bias.

\par From this discussion, an interesting question arises: we have been able to extend our auditory and visual senses, so why not smelling? Similar to sight, smell is a sense and it too could potentially be extended. Galileo and Hooke were the first individuals to discover that we are able to use these instruments to extend our sight, and it stands to reason that the professor could be the first individual to extend our sense of smell. Cohen writes ?Science no longer [depends] on the primary sense-impressions of sight, feel, sound, taste, or smell, but rather on the output of a scientific instrument that [is] a mediator of the direct impressions of nature? (Cohen, 434). This quotation from Cohen becomes important as it allows us to open up to the possibility that there could be an object that allows us to smell things from far away.

\par In conclusion, exploring the different ways of knowing, it is hard to perceive right away that there is a possibility that such an object exists, but we can use reason to question our beliefs. In addition, given the professors track record and the scientific advances already made in that universe, it's definitely possible that the professor would have created a Smell-O-Scope.

\end{document}