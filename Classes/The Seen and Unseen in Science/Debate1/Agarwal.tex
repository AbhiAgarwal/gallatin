\documentclass[11pt, oneside]{article}
\usepackage{geometry}
\geometry{letterpaper}
\usepackage{graphicx}
\usepackage{amssymb}
\linespread{2.0}

\title{Debate: Things newly seen}
\author{Abhi Agarwal}
\date{}

\begin{document}
\maketitle

% Side A: Professor Farnsworth?s discovery of the giant ball of garbage is reliable, the city is doomed, and we should begin looting immediately.
% Side B: Professor Farnsworth?s discovery of the giant ball of garbage is not reliable, the city has nothing to fear, and looting should be postponed slightly. 

%% ASSUMPTIONS
% 1000 years into the future

% Establish who this person is first.
% Talk about his personality

\par Professor Farnsworth has had many successes in the past. In the show, he has managed to create a clone of himself, the planet express ship (that runs at 300\% fuel capacity), invent a Paradox-Free Time Travel machine, is a tenured professor at the Mars University, and won a Fields Medal. In addition, he has also managed to save the world through his inventions in many ways in many different episodes. All of his inventions seem to have been impossible and beyond the realm of what was imaginable before they were invented. His experiences of building and changing the world in the past help validate the idea that it's possible that he managed to build a Smell-O-Scope.

\par However, just by his appearance in the clip (and previous knowledge) the professor doesn't seem like he's in the right mind. The professor forgot that he built the device a whole year ago, and is known to suffer from dementia. Naturally, it would be fairly difficult to trust a scientific breakthrough of that caliber if the individual was of that age and medical condition.

% There were multiple people who were smelling at least "something different" every time they smelled it.
% Personality of the people around him

\par It's also useful to take into account the other people who tried out his discovery. Both Fry and Leela tried out the Smell-O-Scope. Leela's personality is often skeptical and logical, and this suggests that if she believed in the professor's discovery then the audience would be equally as likely to believe in his discovery. In addition, just by the speed Fry, Leela, and Bender run towards him suggests that they are usually excited to see his discoveries.

%% The extra scientific device they use.

\par Moreover, the fact that they used a computer and can map the trajectory of the giant ball of garbage does help validate the idea that the Smell-O-Scope is a reliable device. They used computational methods to validate their hypothesis, using satellites and other scientific radars, to guarantee that there is a moving object in space and it's coming towards either (regardless of its smell and what it's composed of). 

% Go into scientific understanding of what can be done and what can't be done.

\par The Smell-O-Scope has only been tested once, and we only know how to operate it and not how it works. In addition, since they are able to go into space, they could have validated their smells for other planets and see if they match, but haven't. To this point we haven't seen the `golden event', which Galison defines as ``the single picture of such clarity and distinctness that it commands acceptance" (Galison, 360), or any statistical evidence where we have valid evidence that the Smell-O-Scope works. Through Galison's framework we can't conclude anything since we have incomplete data to make a judgement.

\par However, if it hasn't been and done before it doesn't mean it can't happen. A lot of the disbelief in an invention comes from the ideology that if something hasn't been done before it's extremely likely that it is going to be possible now. There hasn't been a way to smell things from long distances, and the professor's discovery allows for this. Hooke's use of microscopes for scientific observations made ``the world was not what it seemed. [His observations] documented that (1) many 'invisible' things actually existed, and that (2) seeing objects microscopically disclosed radically new appearances" (Hardwood, 136). 

\par Cohen would argue that instruments don't give us the absolute truth. Our brains can still augment or interfere with what we smell, even if we use the Smell-O-Scope. Cohen argues that ``an explorer going into a new land tends to `see' what he expected to find, filling his reports with comparisons made between the new land and the lands with which the explorer was already familiar, either from personal experience or from reading and conversations" (Cohen, 459). In this situation, because Leela and Fry understood the function of the Smell-O-Scope they had a disposition to acknowledge that each planet had its own scent. 

\par In addition, instruments don't make our senses more reliable. They help us interpret the world in a better way, but we are still prone to misinterpreting the world even when using an instrument like the Smell-O-Scope. The instrument could have been misaligned or the smell might have been coming from a gas explosion on the International Space Station. Fry and Leela don't understand the mechanics of how the Smell-O-Scope works, and the professor doesn't even remember how we created it. None of them are able to reliably explain what happen between the smell coming into the device and when it enters their nose or even how the smell is traveling to the device itself. The smell can be altered, and Bender could potentially create a panic in New York because they weren't realizing their bias.

\par From this observation, a question arises - we've been able to extend our hearing and seeing, so why not smelling? Similar to sight, smell is a sense and it can be extended. Just like you can see things far using a telescope or close through a microscope. Galileo, and Hooke were the first individuals to discover that we're able to use these instruments to extend our sight, and it stands to reason that the professor could be the first individual to extend our sense of smell. Cohen writes ``Science no longer [depends] on the primary sense-impressions of sight, feel, sound, taste, or smell, but rather on the output of a scientific instrument that [is] a mediator of the direct impressions of nature (Cohen, 434).

\par In conclusion, 

\end{document}