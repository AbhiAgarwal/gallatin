\documentclass[11pt, oneside]{article}
\usepackage{geometry}
\geometry{letterpaper}
\usepackage{graphicx}
\usepackage{amssymb}
\linespread{2.0}

\title{Debate: Things newly seen}
\author{Abhi Agarwal}
\date{}

\begin{document}
\maketitle


% Side A: Professor Farnsworth?s discovery of the giant ball of garbage is reliable, the city is doomed, and we should begin looting immediately.
% Side B: Professor Farnsworth?s discovery of the giant ball of garbage is not reliable, the city has nothing to fear, and looting should be postponed slightly. 

%% ASSUMPTIONS
% 1000 years into the future

% Establish who this person is first.
% Talk about his personality

\par Professor Farnsworth has had many successes in the past. In the show, he has managed to create a clone of himself, the planet express ship (that runs at 300\% fuel capacity), invent a Paradox-Free Time Travel machine, is a tenured professor at the Mars University, and won a Fields Medal. In addition, he has also managed to save the world through his inventions in many ways in many different episodes. His experiences of building and changing the world in the past help validate the idea that it's possible that he managed to build a Smell-O-Scope. All of his inventions seem to have been impossible and beyond the realm of what was imaginable before they were invented. 

\par However, just by his appearance in the clip the professor doesn't seem like he's in the right mind. The professor forgot that he built the device a whole year ago, and is known to suffer from dementia. Naturally, it would be fairly difficult to trust a scientific breakthrough of that caliber if the individual was of that age and medical condition. 

% There were multiple people who were smelling at least "something different" every time they smelled it.
% Personality of the people around him

\par It's also useful to take into account the other people who tried out his discovery. Both Fry and Leela tried out the Smell-O-Scope. Leela's personality is often skeptical and logical, and this suggests that if she believed in the professor's discovery then the audience would be equally as likely to believe in his discovery (assuming they continually watch the show and know the character). In addition, just by the speed Fry, Leela, and Bender run towards him suggests that they are usually excited to see his discoveries.

\par The fact that they can map the trajectory and it's a moving object does help validate the idea that it can be a real device. 

% Go into scientific understanding of what can be done and what can't be done.

\par If it hasn't been and done before it doesn't mean it can't happen.

\par There hasn't been a way to smell things 

\par Smell is a sense, and it can be extended. Just like you can see things far using a telescope or close through a microscope - you might be able to smell things from far as well. We've been able to extend our hearing, and seeing - why not also smelling?

\par It has only been tested in that scenario once. They weren't able to validate their smells for other planets and see if they match. Since they are able to go into space. Here I believe that we're some what experiencing the golden event as Galison described it: ``'golden event': the single picture of such clarity and distinctness that it commands acceptance" (Galison, 360). There's no statistical evidence. They just started testing the Smell-O-Scope and then described their first 

\end{document}