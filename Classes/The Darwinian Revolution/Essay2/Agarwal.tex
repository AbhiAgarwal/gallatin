\documentclass[11pt, oneside]{article}   	
\usepackage{geometry}                		
\usepackage{graphicx}	
\usepackage{amssymb}
\usepackage{epigraph}
\usepackage[doublespacing]{setspace}
\usepackage{etoolbox}

\setlength\epigraphwidth{15cm}
\setlength\epigraphrule{0pt}
\geometry{letterpaper} 
\makeatletter
\patchcmd{\epigraph}{\@epitext{#1}}{\itshape\@epitext{#1}}{}{}
\makeatother

\title{Darwin's Theory of Evolution and Eugenics}
\author{Abhi Agarwal (abhia@nyu.edu)}
\date{}							

\begin{document}
\maketitle

% Introduction
\par 

% Background information
\par Eugenics was born out an interpretation of the theory of evolution. Francis Galton who was a half-cousin to Charles Darwin desired to apply Darwin's theory that explained the development of plants and animal species to humans. Galton found the idea of questioning natural selections in humans intriguing and coined the term eugenics, which tried to improve the quality of our society by applying the principles of Darwinism to the human population. 

% Sterilization law in USA
\par Darwin's influence on the elite shaped the way research, development, and progress was made in major nations such as the United States. The majority of individuals in the United States rejected Darwinism, however some of the leaders, intellectuals, scientists, and biologists believed in the theory and so it was applied in many scientific, economic, and intellectual discussions and proposals in the United States. 

% Galton coins the term after Darwin's Death.

\begin{thebibliography}{9}
\bibitem{1}
  Darwin, Charles, and W. F. Bynum.
  \emph{On the Origin of Species: By Means of Natural Selection or the Preservation of Favoured Races in the Struggle for Life}.
  London: Penguin Classics, 2009.
  Print.

\end{thebibliography}
\end{document}  