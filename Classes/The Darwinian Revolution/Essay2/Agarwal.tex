\documentclass[11pt, oneside]{article}   	
\usepackage{geometry}                		
\usepackage{graphicx}	
\usepackage{amssymb}
\usepackage{epigraph}
\usepackage[doublespacing]{setspace}
\usepackage{etoolbox}

\setlength\epigraphwidth{15cm}
\setlength\epigraphrule{0pt}
\geometry{letterpaper} 
\makeatletter
\patchcmd{\epigraph}{\@epitext{#1}}{\itshape\@epitext{#1}}{}{}
\makeatother

\title{Darwin's Theory of Evolution and Eugenics}
\author{Abhi Agarwal (abhia@nyu.edu)}
\date{}							

%Structure
%Introduction
%Charles Darwin, and Galton Introduction
%Definition of Eugenics by Galton
%Darwin's work and portions of Eugenics
%Interpretation from Darwin's Work

% Resources
% http://www.ianramseycentre.info/videos/whos-to-blame-for-eugenics.html
% https://highschoolbioethics.georgetown.edu/units/cases/unit4_note.html
% http://www.todayinsci.com/G/Galton_Francis/GaltonFrancis-Quotations.htm

\begin{document}
\maketitle

% Introduction
\par 

% Darwinism
\par We can understand Darwinism to be a theory 

% Background information
% Galton's Introduction
% Galton -> Father of Behavioral Statistics, 
% Will our children be born with more virtuous dispositions, if we ourselves have acquired virtuous habits? Or are we no more than passive transmitters of a nature we have received, and which we have no power to modify? (http://psychclassics.yorku.ca/Galton/talent.htm)
% Meaning of the word Eugenics
% http://www.eubios.info/EJ93/ej93e.htm
% science of improving stock-not only by judicious mating, but whatever tends to give the more suitable races or strains of blood a better chance of prevailing over the less suitable than they otherwise would have had.

\par 
% 1. Eugenics
Eugenics was born out an interpretation of the theory of evolution. Francis Galton, who was a half-cousin to Charles Darwin, desired to apply Darwin's theory that explained the development of plants and animals to humans. Darwin's work showed that evolution takes place through natural selection, and operates on a variety of traits and characteristics that influence ones survival and propagation of their species. 
% 2. Defining Eugenics, and his initial links to natural selection
Galton found the idea of questioning natural selection and variability in humans intriguing and in particular he was mostly interested in differences between individuals in their mental traits. He was interested in examining variation in mental traits, and mental ability or as Galton saw it their ``genius". Galton evaluated his hypothesis of mental abilities being inherited in his book Hereditary Genius, which provided evidence by looking at frequencies of ``genius" in families and concluded that mental abilities did in fact run in families. 
% 3. Galton
In his introductory paper, Inquiries into Human Faculty, he defines eugenics in a footnote as the ``science of improving stock, which is by no means confined to questions of judicious mating, but which, especially in the case of man, takes cognisance of all influences that tend in however remote a degree to give to the more suitable races or strains of blood a better chance of prevailing speedily over the less suitable than they otherwise would have had" ("Inquiries Into Human Faculty and Its Development", 24-25). In essence, Galton was interested in trying to improve the quality of our society by applying the principles of Darwinism to the human population. 

% Darwin's theory. Was it correctly derived? Was Darwin actually implying this or not.
% Was he just inspired, or did Darwin actually imply it in his work?
% Galton coins the term after Darwin's Death.
% Galton was not a biologist, but he was a mathematician so he saw it as a statistical problem rather than a mathematical problem. He wanted to improve the probability or the odds in general rather than focusing on individuals.
% Man was still evolving, Galton believed that a wide range of human characteristics were inherited, including mental, physical, and moral traits. 
\par Self-direction of human evolution. Darwinism could be seen as the identify of eugenics, and not the root of eugenics.
\par Man was still evolving
\par Darwin's actual words that could have influenced Eugenics. 
\par Darwin's passage in the Descent of Man: ``Thus the weak members of civilized societies propagate their kind. No one who has attended to the breeding of domestic animals will doubt that this must be highly injurious to the race of man", and ``hardly any one is so ignorant as to allow his worst animals to breed" ("Descent of Man", 159).
\par Darwin's view on superior and inferior races. 
\par Galton was not a biologist, but a statistician and so he looked at this idea or this problem from a mathematical point of view. 

% Sterilization law in USA
% Davenport
\par Darwin's influence on the elite shaped the way research, development, and progress was made in major nations such as the United States. The majority of individuals in the United States rejected Darwinism, however some of the leaders, intellectuals, scientists, and biologists believed in the theory and so it was applied in many scientific, economic, and intellectual discussions and proposals in the United States. 

% Conclusion
\par

\begin{thebibliography}{9}
\bibitem{1}
  Darwin, Charles, and W. F. Bynum.
  \emph{On the Origin of Species: By Means of Natural Selection or the Preservation of Favored Races in the Struggle for Life}.
  London: Penguin Classics, 2009.
  Print.
\bibitem{2}
  Galton, Francis.
  \emph{Eugenics: Its Definition, Scope, and Aims}.
  The American Journal of Sociology 10.1 (1904): 1-25.
  The University of Chicago Press. Web. 11 Nov. 2014.
\bibitem{1}
  Galton, Francis.
  \emph{Inquiries Into Human Faculty and Its Development}.
  Inquiries into Human Faculty and Its Development (1883): 24-25. 
  Web.
\end{thebibliography}
\end{document}  