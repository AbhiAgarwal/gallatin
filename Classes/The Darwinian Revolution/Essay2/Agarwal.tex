\documentclass[11pt, oneside]{article}   	
\usepackage{geometry}                		
\usepackage{graphicx}	
\usepackage{amssymb}
\usepackage{epigraph}
\usepackage[doublespacing]{setspace}
\usepackage{etoolbox}

\setlength\epigraphwidth{15cm}
\setlength\epigraphrule{0pt}
\geometry{letterpaper} 
\makeatletter
\patchcmd{\epigraph}{\@epitext{#1}}{\itshape\@epitext{#1}}{}{}
\makeatother

\title{Darwin's Theory of Evolution and Eugenics}
\author{Abhi Agarwal (abhia@nyu.edu)}
\date{}							

%Structure
%Introduction
%Charles Darwin, and Galton Introduction
%Definition of Eugenics by Galton
%Darwin's work and portions of Eugenics
%Interpretation from Darwin's Work

% Resources
% http://www.ianramseycentre.info/videos/whos-to-blame-for-eugenics.html
% https://highschoolbioethics.georgetown.edu/units/cases/unit4_note.html
% http://www.todayinsci.com/G/Galton_Francis/GaltonFrancis-Quotations.htm

\begin{document}
\maketitle

% Introduction
\par 

% Background information
% Galton's Introduction
% Galton -> Father of Behavioral Statistics, 
% Will our children be born with more virtuous dispositions, if we ourselves have acquired virtuous habits? Or are we no more than passive transmitters of a nature we have received, and which we have no power to modify? (http://psychclassics.yorku.ca/Galton/talent.htm)
% Meaning of the word Eugenics
% http://www.eubios.info/EJ93/ej93e.htm
% science of improving stock-not only by judicious mating, but whatever tends to give the more suitable races or strains of blood a better chance of prevailing over the less suitable than they otherwise would have had.

\par 
% 1. Eugenics
Eugenics was born out an interpretation of the theory of evolution. Francis Galton, who was a half-cousin to Charles Darwin, desired to apply Darwin's theory that explained the development of plants and animals to humans. Darwin's work showed that evolution takes place through natural selection, and operates on a variety of traits and characteristics that influence ones survival and propagation of their species. 
% 2. Defining Eugenics, and his initial links to natural selection
\par Galton found the idea of questioning natural selection and variability in humans intriguing and in particular he was mostly interested in differences between individuals in their mental traits. He was interested in examining variation in mental traits, and mental ability or as Galton saw it their ``genius". Galton evaluated his hypothesis of mental abilities being inherited in his book Hereditary Genius, which provided evidence by looking at frequencies of ``genius" in families and concluded that mental abilities did in fact run in families. 
% 3. Galton
In his introductory paper, Inquiries into Human Faculty, he defines eugenics in a footnote as the ``science of improving stock, which is by no means confined to questions of judicious mating, but which, especially in the case of man, takes cognisance of all influences that tend in however remote a degree to give to the more suitable races or strains of blood a better chance of prevailing speedily over the less suitable than they otherwise would have had" ("Inquiries Into Human Faculty and Its Development", 24-25). In essence, Galton was interested in trying to improve the quality of our society by applying the principles of Darwinism to the human population. 

% Darwin's theory. Was it correctly derived? Was Darwin actually implying this or not.
%\par Darwin's passage in the Descent of Man: ``Thus the weak members of civilized societies propagate their kind. No one who has attended to the breeding of domestic animals will doubt that this must be highly injurious to the race of man", and ``hardly any one is so ignorant as to allow his worst animals to breed", ``if we were intentionally to neglect the weak and helpless, it could only be for a contingent benefit, with a certain and great present evil." ("Descent of Man", 159).
% Was he just inspired, or did Darwin actually imply it in his work?

% Galton coins the term after Darwin's Death.
% Galton was not a biologist, but he was a mathematician so he saw it as a statistical problem rather than a mathematical problem. He wanted to improve the probability or the odds in general rather than focusing on individuals.
% Man was still evolving, Galton believed that a wide range of human characteristics were inherited, including mental, physical, and moral traits. 

\par Darwin's theory of evolution explains that species are altered by natural selection and it can be suggested that the same principles are applied to artificial selection, for example farmers select the best plants and the best animals to breed to get the best yield in the future. The natural thought leading that principle that Galton had is if artificial selection in plants and animals is being done then the same principles can be applied to humans. Eugenics in its purest form is improving the genetic quality of the human population by strictly allowing only a select portion of humans with the best traits to breed. 
\par A couple passages from The Descent of Man can be interpreted as Darwin's acknowledgement of concepts similar to eugenics. Darwin suggests that ``weak members of civilized societies propagate their kind" ("Descent of Man", 159), and it establishes that Darwin does believe in a society where the weaker members create weaker offspring. He follows with the idea that ``hardly any one is so ignorant as to allow his worst animals to breed" ("Descent of Man", 159), which amplify the view that selection is important and it's our responsibility to always improve traits of our animals and plants and by extension our own race. In addition, in my opinion the most relevant opinion that Darwin expresses is  ``if we were intentionally to neglect the weak and helpless, it could only be for a contingent benefit, with a certain and great present evil" ("Descent of Man", 159). This resonates with eugenics in a sense that the weak are neglected, and left childless in order to not pass on their genetic material, and there is opposition that occurs because of this decision which presents itself as the evil as well as the act of leaving someone to die. Additionally, Darwin could be referring to the human population when he makes this statement as previously he refers to a surgeon operating on a patient right before the quotation. I don't think that he's particularly referring to selectively breeding individuals in this statement, but he's referring to intentionally neglecting people who are diseased or are prisoned. People who are diseased (quarantine) or prisoned (prison) are generally removed from society until they fit into society again, which temporarily or permanently does not allow them to pass on their genetic material or if they have had children then further pass on their genetic material. 
\par Moreover, the three quotations above do not particularly suggest Darwin's opinions on eugenics itself, but the ideas he suggested could have been used to inspire the idea or could have been used to initially spark the eugenics mentality. There is a likelihood that this along with The Origin of Species inspired Galton, and the timeline also reflects that Galton would have read both works before he published his paper. Darwin's tone in the Origin of Species implies this sense of competition, and that competition is necessary in order for the human race to improve. It's a competition in a sense that there is a struggle to survive, and over time species that aren't able to adapt become extinct. One of the key takeaways from Darwin's theory in relation to eugenics is that man is still evolving, and there isn't an end to his evolution. In the Origin of Species he writes that ``[he is] fully convinced that species are not immutable" ("On the Origin of Species", 15). This is important in relation to competition as it implies that we are able to mould and adapt our species, and that we as a race are able to make these changes ourselves rather than leaving them to natural selection. Linking this to eugenics, it's clear from Galton's essay on Eugenics in 1904 that his views on eugenics were to speed up the process, and he writes ``[what] nature does blindly, slowly, and ruthlessly, man may do providently, quickly, and kindly" ("Inquiries Into Human Faculty and Its Development", 24-25). My opinion is that Galton could have thought about speeding up this process to improve the chances of survival, and eugenics is definitely a way that the human population

\par Self-direction of human evolution. Darwinism could be seen as the identify of eugenics, and not the root of eugenics.
\par Darwin's view on superior and inferior races. 
\par Galton was not a biologist, but a statistician and so he looked at this idea or this problem from a mathematical point of view. 

% Sterilization law in USA
% Davenport
\par Darwin's influence on the elite shaped the way research, development, and progress was made in major nations such as the United States. The majority of individuals in the United States rejected Darwinism, however some of the leaders, intellectuals, scientists, and biologists believed in the theory and so it was applied in many scientific, economic, and intellectual discussions and proposals in the United States. 

% Conclusion
\par

\begin{thebibliography}{9}
\bibitem{1}
  Darwin, Charles, and W. F. Bynum.
  \emph{On the Origin of Species: By Means of Natural Selection or the Preservation of Favored Races in the Struggle for Life}.
  London: Penguin Classics, 2009.
  Print.
\bibitem{2}
  Galton, Francis.
  \emph{Eugenics: Its Definition, Scope, and Aims}.
  The American Journal of Sociology 10.1 (1904): 1-25.
  The University of Chicago Press. Web. 11 Nov. 2014.
\bibitem{1}
  Galton, Francis.
  \emph{Inquiries Into Human Faculty and Its Development}.
  Inquiries into Human Faculty and Its Development (1883): 24-25. 
  Web.
\end{thebibliography}
\end{document}  