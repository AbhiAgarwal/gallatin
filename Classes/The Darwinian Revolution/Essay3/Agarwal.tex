\documentclass[11pt, oneside]{article}   	
\usepackage{geometry}                		
\usepackage{graphicx}	
\usepackage{amssymb}
\usepackage{epigraph}
\usepackage[doublespacing]{setspace}
\usepackage{etoolbox}

\setlength\epigraphwidth{15cm}
\setlength\epigraphrule{0pt}
\geometry{letterpaper} 
\makeatletter
\patchcmd{\epigraph}{\@epitext{#1}}{\itshape\@epitext{#1}}{}{}
\makeatother

\title{}
\author{Abhi Agarwal (abhia@nyu.edu)}
\date{}

\begin{document}
\maketitle

\par REVOLUTION? What exactly is revolutionary about the Darwinian revolution? Imagine that a friend
or relative who knows you are taking this course asked you this question. How would you answer? Is
it the positive contributions to knowledge? The dangerous philosophical implications suggested by
Gould and others? The cynical answer to the meaning of human existence suggested by Vonnegut?
Or by Dawkins? How would you articulate the dimensions of the Darwinian revolution?

\par Talk about how people applied their understanding/thoughts before Darwin's theories

\par Talk about defining and explaining Darwin's theory that could cause revolutions, and how it could create shifts in understanding.

\par Talk about people's thoughts and representations after his theory came out.

\par 

\begin{thebibliography}{9}
\bibitem{1}
	Darwin, Charles, and W. F. Bynum.
	\emph{On the Origin of Species: By Means of Natural Selection or the Preservation of Favored Races in the Struggle for Life}.
	London: Penguin Classics, 2009.
	Print.
\bibitem{2}
	Pagel, Mark D. 
	\emph{Wired for Culture: Origins of the Human Social Mind}. 
	N.p.: W. W. Norton \& Company, 2013. 
	Print. 
\bibitem{3}
	Slack, Gordy. 
	\emph{The Battle over the Meaning of Everything: Evolution, Intelligent Design, and a School Board in Dover, PA.}
	San Francisco: Jossey-Bass, 2007. 
	Print.
\end{thebibliography}
\end{document}  