\documentclass[11pt, oneside]{article}   	
\usepackage{geometry}                		
\usepackage{graphicx}	
\usepackage{amssymb}
\usepackage{epigraph}
\usepackage[doublespacing]{setspace}
\usepackage{etoolbox}

\setlength\epigraphwidth{15cm}
\setlength\epigraphrule{0pt}
\geometry{letterpaper} 
\makeatletter
\patchcmd{\epigraph}{\@epitext{#1}}{\itshape\@epitext{#1}}{}{}
\makeatother

\title{}
\author{Abhi Agarwal (abhia@nyu.edu)}
\date{}

\begin{document}
\maketitle

\par REVOLUTION? What exactly is revolutionary about the Darwinian revolution? Imagine that a friend
or relative who knows you are taking this course asked you this question. How would you answer? Is
it the positive contributions to knowledge? The dangerous philosophical implications suggested by
Gould and others? The cynical answer to the meaning of human existence suggested by Vonnegut?
Or by Dawkins? How would you articulate the dimensions of the Darwinian revolution?

% 1. EXPLAINING AND DEFINITION DARWIN'S THEORY OF EVOLUTION TO CREATE A BASE.

\par Talk about defining and explaining Darwin's theory that could cause revolutions, and how it could create shifts in understanding.

``Man selects only for his own good: Nature only for that of the being which she tends." ("The Origin of Species")

% 1. THEORIES PREVIOUS TO DARWIN'S THEORY OF EVOLUTION - BRIEF INTRODUCTION

\par Talk about how people applied their understanding/thoughts before Darwin's theories

% THESIS:
% THESIS THAT DARWIN'S THEORY CREATED A REVOLUTION BY ALLOWING US TO EXPLAIN HOW THINGS CHANGED OR ADAPTED, AND WHY THEY CHANGED OR ADAPTED. 
% There as no framework to explain how things such as culture before, but now there exists a theory that allows us t	o explain everything. 

% 1. PRESENT DEFINITIONS OF HOW SLACK & PAGEL VIEW HIS THEORY OF EVOLUTION
% 2. HOW THEY BOTH DEFINED EVOLUTION, OR NATURAL SELECTION.
% 

\par The many definitions that people have used to define his theories, and the different views people take on it. 

``Natural selection does not maximize happiness or even well-being, but rather long-term reproductive success" (Pagel, 24).

``[Although] evolution may seem, in some vague sense, `good thing', especially since we are a product of it, nothing actually `wants' to evolve. Evolution is something that happens, willy-nilly, in spite of all the effort of the replicators (and nowadays of the genes) to prevent it happening" (Dawkins, 19).

\par Talk about people's thoughts and representations after his theory came out.

\par Pagel looks at how languages evolve by applying the biological evolution and nature selection theory as a template. 

``what this debate was really about was neither evolution nor ID, but the worldviews they enabled" (Slack, 39-40).

``Evolution, he said, permits a relativistic, purposeless, Godless view of the world, in which self-aggrandizement and pleasure are sufficient ends in themselves, and the only objective measure of goodness is reproductive fitness." (Slack, 40-41)

``Our invention of culture around that time created an entirely new sphere of evolving entities. Humans had acquired the ability to learn from others, and to copy, imitate and improve upon their actions" (Pagel, 2).

``Our cultural survival vehicles were built not from coalitions of genes but from coalitions of ideas roped together by cultural evolution" (Page, 46).

\begin{thebibliography}{9}
\bibitem{1}
	Darwin, Charles, and W. F. Bynum.
	\emph{On the Origin of Species: By Means of Natural Selection or the Preservation of Favored Races in the Struggle for Life}.
	London: Penguin Classics, 2009.
	Print.
\bibitem{2}
	Pagel, Mark D. 
	\emph{Wired for Culture: Origins of the Human Social Mind}. 
	N.p.: W. W. Norton \& Company, 2013. 
	Kindle Edition. 
\bibitem{3}
	Slack, Gordy. 
	\emph{The Battle over the Meaning of Everything: Evolution, Intelligent Design, and a School Board in Dover, PA.}
	San Francisco: Jossey-Bass, 2007. 
	Kindle Edition.
\end{thebibliography}
\end{document}  