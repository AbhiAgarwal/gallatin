\documentclass[11pt, oneside]{article}   	
\usepackage{geometry}                		
\usepackage{graphicx}	
\usepackage{amssymb}
\usepackage{epigraph}
\usepackage[doublespacing]{setspace}
\usepackage{etoolbox}

\setlength\epigraphwidth{15cm}
\setlength\epigraphrule{0pt}
\geometry{letterpaper} 
\makeatletter
\patchcmd{\epigraph}{\@epitext{#1}}{\itshape\@epitext{#1}}{}{}
\makeatother

\title{}
\author{Abhi Agarwal (abhia@nyu.edu)}
\date{}

\begin{document}
\maketitle

\par REVOLUTION? What exactly is revolutionary about the Darwinian revolution? Imagine that a friend
or relative who knows you are taking this course asked you this question. How would you answer? Is
it the positive contributions to knowledge? The dangerous philosophical implications suggested by
Gould and others? The cynical answer to the meaning of human existence suggested by Vonnegut?
Or by Dawkins? How would you articulate the dimensions of the Darwinian revolution?

\begin{thebibliography}{9}
\bibitem{1}
  Darwin, Charles, and W. F. Bynum.
  \emph{On the Origin of Species: By Means of Natural Selection or the Preservation of Favored Races in the Struggle for Life}.
  London: Penguin Classics, 2009.
  Print.
\end{thebibliography}
\end{document}  