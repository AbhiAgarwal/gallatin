\documentclass[11pt, oneside]{article} 
\usepackage{geometry}                		
\geometry{letterpaper}
\usepackage{graphicx}				
\usepackage{amssymb}
\linespread{2}

\usepackage[sc]{mathpazo}
\usepackage{eulervm}

\makeatletter
\newcommand*{\rom}[1]{\expandafter\@slowromancap\romannumeral #1@}
\makeatother

\title{Exploring the shift in worldview due to Darwin's Theory of Evolution}
\author{Abhi Agarwal (abhia@nyu.edu)}
\date{}

\begin{document}
\maketitle

\par Darwin's theory of evolution caused a shift in how people perceived the world, and how they tried to understand their surroundings. Slack, in his book `The Battle over the Meaning of Everything', writes ``what [the dover trials were] really about was neither evolution nor ID, but the worldviews they enabled" (Slack, 39-40). Slack points out that Darwin's theory fundamentally altered the way individuals thought about the world, rather than simply adding on to their existing notions. It dramatically shifted people's view of different fields and concepts, and even introduced new ideas to fields that it was not directly related to, such as Computer Science, Design Theory, and Mathematics. 

\par To grasp this idea, we have to first look at defining and understanding Darwin's theory-only then will we be able to comprehend the impact it can have. Darwin's theory has five big concepts, and each of these concepts can be held or studied independently. The five big concepts are Evolution, Common Descent, Gradualism, Population Speciation, and Natural selection. 

\par To begin to understand Darwin's theory we should first explore the principle of natural selection. At the beginning of his chapter on natural selection, Darwin poses the question ``[can] the principle of selection, which we have seen is so potent in the hands of man, apply in nature?" (Darwin, Origin, 80). The principle of natural selection is the theory that stands to answer this question. Several professions, such as farming, have practiced the principle of selection by letting their best animals breed; can this same principle occur in nature, among all living creatures too? 

\par Darwin explains this best through the idea that ``[man] selects only for his own good: Nature only for that of the being which she tends" (Darwin, Origin, 468), which represents the idea that an individual selects only for her own good, but nature selects for the adaptation and progress of all beings. Darwin's ideology of natural selection stands on top of the idea that ``every variation, even the slightest; rejecting that which is bad, preserving and adding up all that is good" (Darin, Origin, 83). Natural selection is to preserve variations that leads to an advantage, and elimination variations that contribute to injury or death. 

\par Another significant part of Darwin's theory is the notion of a struggle for existence. Darwin's explanation of the principle for struggle for existence is important to note: ``As many more individuals of each species are born than can possibly survive; and as, consequently, there is a frequently recurring struggle for existence, it follows that any being, if it vary however slightly in any manner profitable to itself, under the complex and sometimes varying conditions of life, will have a better chance of surviving, and thus be naturally selected. From the strong principle of inheritance, any selected variety will tend to propagate its new and modified form" (Darwin, Origin, 14). Huxley's explanation is the idea that populations would have grown geometrically without the struggle for existence, but this struggle corrects for it. Contrastingly, Darwin's view is that the struggle for existence is a larger metaphor for the inner workings of a larger natural system, which in itself is a very powerful idea.

\par An interesting observation is the fact that people who have written or developed theories involving Darwin's theory have different views of defining natural selection or the theory of evolution. This is interesting as it depicts the ways in which these authors have thought about and interpreted these theories, and applied it in their own ways. The theory of evolution involves a multitude of different components to it, and in order to understand the worldview that their theories represent it is important to note how they portray it. Different individuals form different opinions on the theory given their personal assumptions beforehand, and also mould it to apply to their topic.

\par Pagel, in his book `Wired for Culture: Origins of the Human Social Mind', writes that ``[natural] selection does not maximize happiness or even well-being, but rather long-term reproductive success" (Pagel, 24). Pagel here expresses natural selection as concept for leaving more progeny in future generations. His view of natural selection represents a worldview that it always looks positively to improve the number of offspring there are in the future, but does not necessarily focus on how the process occurs or how we adapts. 

\par Similarly, Dawkins, in his book 'The Selfish Gene', describes evolution as a ```good thing', especially as we are a product of it, [and] that nothing actually `wants' to evolve. Evolution is something that happens, willy-nilly, in spite of all the effort of the replicators (and nowadays of the genes) to prevent it happening`` (Dawkins, 19). His view on evolution can be seen to be both positive and negative. It can be seen positively because it depicts the marvels of the theory of evolution, and that over a long period of time it has created  complex creatures such as ourselves. However, the negative side expresses the ideology that we have no control, and the idea that nothing `wants' to evolve, but goes through the process regardless, is a significant realization. It forces individuals to think about the greater power and the importance of a process such as evolution. 

\par The worldview of many individuals around the world was very different before Darwin's theory of evolution, and changed significantly in the 200 year period before and after Darwin, from 1750-1950. There were a few key theories that individuals developed, which were thought to be credible by scientists at the time. Firstly, there was an evolutionary theorist called Jean Baptiste Chevalier de Lamarck who started the discussion of biological change and started to develop scientific theories that questioned the process of how we are what we are. Lamarck was prominent to the theory that organisms arose into their natural forms by spontaneous generation, then underwent transmutation over a long period of time to become more complex. Lamarck described his theory as organisms moving up a ladder of progress. Lamarck was also known for his theory of Inheritance of Acquired Characteristics, which hypothesized that changes that occur in an organisms life may be transmitted to its offspring. Previous to Lamarck's work biologists used to believe that spontaneous generation would occur to form complex beings. 

\par After Lamarck, a French scientist, Georges Cuvier was the first to document proof for the idea of extinction. Cuvier did not believe in Inheritance of Acquired Characteristics, but believed that there was no connection between different species, and one's characteristics would not impact the future of its offspring. He believed in the theory of catastrophism, which explained how certain species were rendered extinct before a new species evolved. His observations stood from the standpoint of looking at layers of rock, and seeing each level of the rock formation to be extinctions. Another important theory was proposed by Charles Lyell, who made the argument that by studying the geological processes in the present day, we can begin to understand the history of the Earth. This is an important idea, and one that seems obvious to us as we read now. However, it was significant because it allowed Darwin to understand the importance of studying Geology, and sparked the idea that if there exists a connection between different species, then by studying their fossil record it would be possible to discover their origins or the connections between them. 

\par A plethora of work led to what we now know as the Theory of Evolution. It is clear that the ideas leading up to Darwin's theory shifted the perspective of how Darwin saw and perceived things around him. The introduction of the concept of extinction, the idea of acquiring characteristics from our parents, and several other ideas helped Darwin perceive a new worldview, which in turn, shifted our worldview.

\par Moreover, Slack's book, `The Battle Over the Meaning of Everything', was an eyewitness account of the Kitzmiller vs. Dover Areas School Board trial. The Dover trial required schools in the local area to teach intelligent design alongside evolution as an explanation for the origin of the Earth. For the book, Slack interviewed individuals who were key observers or participators in the trial. The views that a some of these individuals have reveal a range of diverse perspectives on Darwin's theory of evolution, as the theory evokes a range of emotions, from a sense of purposelessness to amazement and wonder.

\par During the trial Slack interviews an individual at the Berkeley Faculty Club named Phillip Johnson wrote the book `Darwin on Trial', which is regarded as one of the central texts around the intelligent design movement. In the interview Johnson describes evolution as a process that ``permits a relativistic, purposeless, Godless view of the world, in which self-aggrandizement and pleasure are sufficient ends in themselves, and the only objective measure of goodness is reproductive fitness" (Slack, 40-41). The focus of Johnson's thoughts is primarily religious, and concerns the fact that if evolution by natural selection did occur, then life would have no purpose and morals and ethics would have no foundation. Johnson does not seem to accept the idea that the theory of evolution and God can co-exist, and views the theory as atheistic philosophy. His ideology appears to stand from a fear or disbelief of the theory, as he projects that God is is the only source that gives life meaning and purpose. It also seems to stem from the idea that the theory was not written by a message from God, but an individual who claims to have created a theory to explain the meaning of our existence and why we came to be. In my opinion, it is difficult to shift your worldview, or be convinced, when you fully accept God to be the creator and find meaning in life through there being a God.

\par Moreover, Pagel looks at how languages and cultures evolve by utilizing his worldview of the biological evolution and natural selection theories. Culture is central to the way we view and experience the world, and Pagel uses his interpretation of Darwin's theories to try to understand these phenomenons. Pagel writes that ``[our] invention of culture around that time created an entirely new sphere of evolving entities. Humans had acquired the ability to learn from others, and to copy, imitate and improve upon their actions" (Pagel, 2). Darwin writes something very similar in the Descent of Man, and the idea that people and societies can be changed in a progressive yet drastic way seems to stem from a Darwinistic view of the world. Prior to the Darwinian revolution there were no theories that scientifically explained how man came to be, so there was not a universal understanding of how we were created. There were various religious theories that tried to explain how humans came into existence and how the universe began, but not a universally accepted theory. With this idea, I believe that the Darwinian revolution brought along an understanding that a civilization can progress together, and group dynamics play a role in their future advancement. In Descent of Man, Darwin comments on this issue by explaining that ``man advances in civilization, and small tribes are united into larger communities, the simplest reason would tell each individual that he ought to extend his social instincts and sympathies to all members of the same nation, though personally unknown to him" (Darwin, Descent, 147). 

\par In conclusion, I truly believe that the shift in worldview that Darwin has caused is revolutionary, and it is more important than the direct applications of his theory. Darwin's theory of evolution has allowed individuals to go one step beyond their basic understanding of the world, and to use their worldview to explore areas such as  the evolution of culture, and building Artificial Intelligence that utilizes these concepts of evolution. 

\begin{thebibliography}{9}
\bibitem{1}
	Darwin, Charles, and W. F. Bynum.
	\emph{On the Origin of Species: By Means of Natural Selection or the Preservation of Favored Races in the Struggle for Life}.
	London: Penguin Classics, 2009.
	Print.
\bibitem{2}
	Pagel, Mark D. 
	\emph{Wired for Culture: Origins of the Human Social Mind}. 
	N.p.: W. W. Norton \& Company, 2013. 
	Kindle Edition. 
\bibitem{3}
	Slack, Gordy. 
	\emph{The Battle over the Meaning of Everything: Evolution, Intelligent Design, and a School Board in Dover, PA}.
	San Francisco: Jossey-Bass, 2007. 
	Kindle Edition.
\bibitem{4}
	Dawkins, Richard. 
	\emph{The Selfish Gene}.
	Oxford; New York: Oxford UP, 2006. 
	Print.
\end{thebibliography}
\end{document}  