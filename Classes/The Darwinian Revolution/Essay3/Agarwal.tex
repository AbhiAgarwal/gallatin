\documentclass[11pt, oneside]{article} 
\usepackage{geometry}                		
\geometry{letterpaper}
\usepackage{graphicx}				
\usepackage{amssymb}
\linespread{2}

\makeatletter
\newcommand*{\rom}[1]{\expandafter\@slowromancap\romannumeral #1@}
\makeatother

\title{Exploring the shift in worldview post-Darwin's theory}
\author{Abhi Agarwal (abhia@nyu.edu)}
\date{}

\begin{document}
\maketitle

% REVOLUTION? What exactly is revolutionary about the Darwinian revolution? Imagine that a friend or relative who knows you are taking this course asked you this question. How would you answer? Is it the positive contributions to knowledge? The dangerous philosophical implications suggested by Gould and others? The cynical answer to the meaning of human existence suggested by Vonnegut? Or by Dawkins? How would you articulate the dimensions of the Darwinian revolution?
% http://www.sciencekids.co.nz/sciencefacts/scientists/charlesdarwin.html
% https://www.goodreads.com/work/quotes/481941-on-the-origin-of-species-by-means-of-natural-selection-or-the-preservat
% http://www.alternativereel.com/soc/display_article.php?id=0000000011
% http://martynmurray.wordpress.com/2011/06/09/my-favourite-quotations-from-the-origin-of-species/
% https://www.goodreads.com/book/show/334515.The_Battle_Over_the_Meaning_of_Everything
% http://en.wikipedia.org/wiki/History_of_evolutionary_thought
% https://www.goodreads.com/work/quotes/728232-the-descent-of-man
% http://www.nytimes.com/1982/04/20/science/100-years-after-darwin-s-death-his-theory-still-evolves.html

% Introduction
% 105 words

\par Darwin's theory of evolution caused a shift in how people perceived the world, and tried to understand their surroundings. Slack, in his book `The Battle over the Meaning of Everything', writes ``what [the dover trials were] really about was neither evolution nor ID, but the worldviews they enabled" (Slack, 39-40). Slack points out that Darwin's theory fundamentally altered the way individuals thought about the world, rather than simply adding on to their existing notions. It dramatically shifted people's view of different fields, different concepts and introduced new ideas to fields that it wouldn't directly be applicable to, such as Computer Science, Design Theory, and Mathematics. 

% 1. EXPLAINING AND DEFINITION DARWIN'S THEORY OF EVOLUTION TO CREATE A BASE.
% 389 words

\par To understand this idea, we have to first look at defining, and understanding Darwin's theory and understanding the impact it can create. Darwin's theory has five big concepts, and each of these concepts can be held or studied independently. The five big concepts are Evolution, Common Descent, Gradualism, Population Speciation, and Natural selection. 

\par To begin understanding Darwin's theory we should first explore the principle of natural selection. At the beginning of his chapter on natural selection Darwin poses the question ``[can] the principle of selection, which we have seen is so potent in the hands of man, apply in nature?" (Darwin, Origin, 80), and the principle of natural selection is the theory that stands to answer this question. Farmers and many other professions have practiced the principle of selection by letting their best animals breed and so can that same principle happen in nature, but occur in all living creatures instead? 

\par Darwin explains it best through the idea that ``[man] selects only for his own good: Nature only for that of the being which she tends" (Darwin, Origin, 468), which represents the idea that an individual selects only for her own good, but nature selects for the adaptation and progress of all beings. Darwin's ideology of natural selection stands on top of the idea that ``every variation, even the slightest; rejecting that which is bad, preserving and adding up all that is good" (Darin, Origin, 83). Natural selection is to preserve variations that leads to an advantage, and elimination variations that contribute to injury or death. 

\par In addition, another big part of Darwin's theory is the notion of a struggle for existence. Darwin's explanation of the principle for struggle for existence is important to note: ``As many more individuals of each species are born than can possibly survive; and as, consequently, there is a frequently recurring struggle for existence, it follows that any being, if it vary however slightly in any manner profitable to itself, under the complex and sometimes varying conditions of life, will have a better chance of surviving, and thus be naturally selected. From the strong principle of inheritance, any selected variety will tend to propagate its new and modified form" (Darwin, Origin, 14). Huxley's explanation is the idea that populations would have grown geometrically without the struggle for existence, but the struggle for existence corrects for it. Darwin's view is that the struggle for existence is a larger metaphor for the inner workings of a larger natural system, which in itself is a very powerful idea.

% OPINIONS ON DEFINITION OF EVOLUTION
% 1. PRESENT DEFINITIONS OF HOW SLACK & PAGEL VIEW HIS THEORY OF EVOLUTION
% 2. HOW THEY BOTH DEFINED EVOLUTION, OR NATURAL SELECTION.
% 343 words

\par Moreover, another interesting observation is that different people who have written about theories involving Darwin's theory have different views of defining natural selection or the theory of evolution. This is interesting to observe as it depicts the ways these authors have thought about these theories, and applied it in their own ways. The theory of evolution involves a lot of different components to it, and in order to understand the worldview that their theories represent it is important to note how they portray it. Different individuals form different opinions on the theory given their personal assumptions beforehand, and also mould it to apply to their topic.

\par Pagel, in his book `Wired for Culture: Origins of the Human Social Mind', writes ``[natural] selection does not maximize happiness or even well-being, but rather long-term reproductive success" (Pagel, 24). Pagel here expresses natural selection as concept for leaving more progeny in future generations. His view of natural selection represents a worldview that it always looks positively to improve the number of offspring we have in the future, but doesn't necessarily care about how the process occurs or how it adapts us. Similarly, Dawkins, in his book 'The Selfish Gene', describes evolution as a ```good thing', especially since we are a product of it, [and] nothing actually `wants' to evolve. Evolution is something that happens, willy-nilly, in spite of all the effort of the replicators (and nowadays of the genes) to prevent it happening" (Dawkins, 19). Dawkins' view on evolution can be seen both positively and negatively. It can be seen positively because it depicts the marvels of the theory of evolution, and how over a long period of time it has created such complex creatures such as ourselves. However, the negative side expresses the ideology that we have no control, and the idea that nothing `wants' to evolve, but goes through the process anyways is a huge realization to undergo. The idea of having no control is important because it forces individuals to think about the broader power and importance of a process such as evolution. 

%%%%%%%%%%%% 850 MAX ABOVE %%%%%%%%%%

% 1. THEORIES PREVIOUS TO DARWIN'S THEORY OF EVOLUTION - BRIEF INTRODUCTION
% 429 words

\par The worldview of many individuals around the world was very different before Darwin's theory of evolution, and changed a lot in the 200 year period before and after Darwin, 1750-1950. There were a few key theories that individuals developed that scientists at that time found to be credible. Firstly, there was evolutionary theorist called Jean Baptiste Chevalier de Lamarck who started the discussion of biological change and started to develop and build scientific theories that started questioning the process of how we are what we are. Lamarck was prominent to the theory that organisms arose into their natural forms by spontaneous generation, then were transmuted over a long period of time to becoming more complex, and Lamarck described his theory the organisms moving up a ladder of progress. Lamarck was also known for his theory of Inheritance of Acquired Characteristics, which hypothesized that changes that occur in an organisms life may be transmitted to its offspring. Previous to Lamarck's work biologists used to believe that spontaneous generation would occur to form complex beings. 

\par After Lamarck, a French scientist, Georges Cuvier was the first to document proof for the idea of extinction. Cuvier did not believe in Inheritance of Acquired Characteristics, but believed that there was no connection between different species, and one's characteristics would not impact the future of its offspring. He believed in the theory of catastrophism, which explained how certain species were killed off before a new species came. His observations stood from the standpoint of looking at layers of rock, and seeing each level of the rock formation to be extinctions. Another important theory was proposed by Charles Lyell. Lyell made the argument that by studying the geological processes in the present day we can begin to understand the history of the earth. This is an important idea, and one that seems obvious to us as we read now, but is  important because it allowed Darwin to understand the significance of studying Geology, and sparked the idea that if there exists a connection between different species then by studying their fossil record we can help discover their origins or the connections between them. 

\par A plethora of work led to what we now know as the Theory of Evolution. It's clear to see that the ideas leading up to Darwin's theory shifted the perspective of how Darwin saw and perceived things around him. The introduction of the concept of extinction, the idea of acquiring characteristics from our parents, and much more helped Darwin perceive a new worldview in the same way our worldview shifted by Darwin's theory. 

%%%%%%%%%%%% 1300 MAX ABOVE %%%%%%%%%%

% THESIS:
% THESIS THAT DARWIN'S THEORY CREATED A REVOLUTION BY ALLOWING US TO EXPLAIN HOW THINGS CHANGED OR ADAPTED, AND WHY THEY CHANGED OR ADAPTED. 
% There as no framework to explain how things such as culture before, but now there exists a theory that allows us t	o explain everything. 
% 1000 words
% Talk about people's thoughts and representations after his theory came out.

\par Slack's book, `The Battle Over the Meaning of Everything', was a eye-witness account of the Kitzmiller vs. Dover Areas School Board trial. The Dover trial would require schools in the local area to teach intelligent design alongside evolution as an explanation for the origin of the earth. For the book, Slack interviewed individuals who were key observers or participators in the trial. The views that a couple of these individuals have show a diverse range of views on Darwin's theory of evolution as they depict a range of emotions, such as a sense of purposeless. 

\par During the trial Slack interviews an individual at the Berkeley Faculty Club called Phillip Johnson. Johnson wrote the book `Darwin on Trial', which is regarded as one of the central texts around the intelligent design movement. In the interview Johnson describes evolution as a process that ``permits a relativistic, purposeless, Godless view of the world, in which self-aggrandizement and pleasure are sufficient ends in themselves, and the only objective measure of goodness is reproductive fitness" (Slack, 40-41). The focus of Johnson's thoughts is primarily religious and is concerned that if evolution by natural selection did occur then life would have no purpose as well as moral and ethics would have no foundation. Johnson doesn't seem like he accept the idea that the theory of evolution and God can co-exist, and views the theory as atheistic philosophy. His ideology seems like it stands from a fear or disbelief of the theory, and the idea that God is is the only source that gives life meaning and without God people don't have any reason to live. It also seems to stem from the idea that the theory wasn't written by a message from God, but an individual who claims to have create a theory to explain the meaning of our existence and why we came to be. In my opinion, it's hard to shift your worldview, or be convinced, when you fully accept God to be your creator and find meaning in life through there being a God.

\par Moreover, Pagel looks at how languages and cultures evolve by utilizing his worldview of the biological evolution and natural selection theories. Cultures are central to the way we view, and experience the world and Pagel utilizes his understanding of Darwin's theories to try and understand these phenomenons. Pagel writes that ``[our] invention of culture around that time created an entirely new sphere of evolving entities. Humans had acquired the ability to learn from others, and to copy, imitate and improve upon their actions" (Pagel, 2). Darwin writes something very similar in the Descent of Man, and the idea that people and societies can be changed in a progressive yet drastic way seems to stem from a Darwinistic view of the world. Previous to the Darwinian revolution there weren't theories that explained scientifically how man came to be, and a universal understanding of how we were created in the same way. There were many separate religious theories that explained how the humans came into existence and how the universe began, but not a universal theory. From that idea, I believe that the Darwinian revolution brought along an understanding that  a civilization can progress together, and group dynamics play an effect on the future. In Descent of Man Darwin comments on this similar issue by explaining that ``man advances in civilization, and small tribes are united into larger communities, the simplest reason would tell each individual that he ought to extend his social instincts and sympathies to all members of the same nation, though personally unknown to him" (Darwin, Descent, 147). 

%%%%%%%%%%%% 2500 MAX ABOVE %%%%%%%%%%

% 100 words
\par In conclusion, I truly believe that the shift in worldview that Darwin has caused is revolutionary, and it is more important than the direct applications of Darwin's theory. Darwin's theory of evolution has allowed individuals to go one step beyond our basic understanding of the world, and to use their worldview to explore areas like explaining the evolution of culture, and building Artificial Intelligence that utilize these concepts of evolution. 

%%%%%%%%%%%% 2700 MAX ABOVE %%%%%%%%%%

\begin{thebibliography}{9}
\bibitem{1}
	Darwin, Charles, and W. F. Bynum.
	\emph{On the Origin of Species: By Means of Natural Selection or the Preservation of Favored Races in the Struggle for Life}.
	London: Penguin Classics, 2009.
	Print.
\bibitem{2}
	Pagel, Mark D. 
	\emph{Wired for Culture: Origins of the Human Social Mind}. 
	N.p.: W. W. Norton \& Company, 2013. 
	Kindle Edition. 
\bibitem{3}
	Slack, Gordy. 
	\emph{The Battle over the Meaning of Everything: Evolution, Intelligent Design, and a School Board in Dover, PA}.
	San Francisco: Jossey-Bass, 2007. 
	Kindle Edition.
\bibitem{4}
	Dawkins, Richard. 
	\emph{The Selfish Gene}.
	Oxford; New York: Oxford UP, 2006. 
	Print.
\end{thebibliography}
\end{document}  