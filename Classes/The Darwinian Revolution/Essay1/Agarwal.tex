\documentclass[11pt, oneside]{article}
\usepackage{geometry}
\geometry{letterpaper}
\usepackage{graphicx}
\usepackage{amssymb}
\usepackage{setspace}
\doublespacing
\usepackage{epigraph}

% \epigraphsize{\small}% Default
\setlength\epigraphwidth{15cm}
\setlength\epigraphrule{0pt}

\usepackage{etoolbox}

\makeatletter
\patchcmd{\epigraph}{\@epitext{#1}}{\itshape\@epitext{#1}}{}{}
\makeatother

\title{Personal development of Charles Darwin during The Voyage of the Beagle}
\author{Abhi Agarwal}
\date{}

\begin{document}
\maketitle

% Introduction

\par Darwin developed both as a person and as an academic during The Voyage of the Beagle. His view of the world, and his view on God changed during the trip, and he also solidified his interest to become an academic in science and his interest to contribute to science. But there were parts of his personality that stayed the same, such as his curiosity and child-like passion for exploration, which, in my opinion, along with his development ultimately allowed him to explore the parts of science he did and push the theories forward.

% God

\par Darwin was raised as a Nonconformist, attended Shrewsbury School, a Church of England school, then Edinburgh University before moving to Cambridge University. Historically at that time Nonconformists sent their sent their children to Scottish universities, like Edinburgh University, because they had a reputation for the hard sciences and medicine. In contrast to University of Oxford and Cambridge, which had mandatory religious reading specifically on the Anglicanism tradition. At Edinburgh University Darwin became interested in natural history, and most of the professors and reading for natural history, at the time, depicted animals and planets to be creations of God. Darwin was interested in natural history, and he wasn't interested in doing medicine and so he left Edinburgh University. After leaving Edinburgh University he took his fathers suggestion to pursue a Bachelor of Arts, at Cambridge University, to become a parson of the Anglican Church, a clergyman.

\epigraph{``The old argument of design in nature, as given by Paley, which formerly seemed to me so conclusive, fails, now that the law of natural selection has been discovered. We can no longer argue that, for instance, the beautiful hinge of a bivalve shell must have been made by an intelligent being, like the hinge of a door by man."}{--- \textup{Autobiographies page 50, Charles Darwin}}

\par At Cambridge he was taken by the natural theology by William Paley, and his argument for the existence of an intelligent creator for the intelligent design of the universe, which Darwin, as shown by the quote below, starts criticizing only two years after the Voyage of the Beagle. In my opinion, during the voyage Darwin must have changed quite a lot as he went from an individual who was actively working towards becoming a clergyman to questioning the existence of a creator and questioning Paley's ideology and reasoning, which was what inspired him so much during his years at Cambridge.

\par On the voyage one of our first encounters of his belief is when he questions planktons. Darwin ponders why they were created with such beauty if they were not to be noticed, and why the creator would spend time designing such a creature. It was particularly interesting to me how observant Darwin was at this stage; Darwin was only roughly 23 years old, and was pondering questions about the existence of particular creatures especially when it was the first place they visited during the voyage. It shows the types of questions Darwin was interested in, and also the types of questions Darwin was puzzled by. Darwin was interested in understanding everything about each creature, including where the particular creature came from and its purpose, but he was also influenced by Paley's theory at that time which didn't allow him to see past there being a greater creator and a designer for the creatures.

\epigraph{``I remember being heartily laughed at by several of the officers for quoting the Bible as an unanswerable authority on some point of morality... I had gradually come, by this time, to see that the Old Testament from its manifestly false history of the world.. and from its attributing to God the feelings of a revengeful tyrant, was no more to be trusted than the sacred books of the Hindoos, or the beliefs of any barbarian."}{--- \textup{Autobiographies page 49, Charles Darwin}}

\par This becomes a little more apparent when reading his views in his autobiography. Darwin starts becoming a little more aggressive about his views, and at this point questions the Bible, and similar holy books. To me it seems like he's questioning the premise of the different holy books, and the fact that they can't all be the right. He seems like he's confused at this point, and I believe these thoughts are during roughly the middle of the voyage. It's hard for him to process his thoughts, and I believe it's hard for him because he's learning new things about the age of the universe and examining different creatures and finding that the evidence he is finding doesn't fit with the Bible. It's hard as a college student to try and have beliefs coming into college, and then learning new things that are counter to those beliefs. It takes a while to start adapting to those beliefs and trying to fit them into a new model, and Darwin was facing these issues but at a much larger scale as he was learning so many new things at the same time. 

\par Darwin further questions the creator when he's exploring the Galapagos Islands. Darwin thought it was unusual that the creatures were specific to their regions, and very different to each other. For example, on the Galapagos Islands the mockingbirds and tortoise were different on each island, and had characteristics that defined which island they came from. He contrasted these to the kangaroos and platypus that he found in Australia, and thought about how different they were, as if there were different creators who had designed them. I believe this is what sparked and later helped him realize how specific creatures develop depending on their surroundings; it seemed to me that he started to believe that there wasn't a way for a creator to have designed these many creatures that were specific to their environment. 

% Curiosity

\par Even after exploring the world, and things he hadn't experience before, such as slavery, Darwin still had a child-like curiosity and will to explore. He had a different way of exploring things, and he wasn't afraid of new experiences and new possibilities. To me it was interesting to see Darwin getting more open to ideas as he got older, and I think this can be see through both his views on religion and his behavior. Intuition would suggest that a person would become more closed off and sceptic after experiencing slavery, and encounters with different civilizations, but Darwin didn't. 

\epigraph{``One day, as I was amusing myself by galloping and whirling the balls round my head"}{--- \textup{The Voyage of the Beagle page 76, Charles Darwin}}

\epigraph{``The earthquake, however, must be to every one a most impressive event: the earth, considered from our earliest childhood as the type of solidity, has oscillated like a thin crust beneath our feet; and in seeing the laboured works of man in a moment overthrown, we feel the insignificance of his boasted power."}{--- \textup{The Voyage of the Beagle page 375, Charles Darwin}}

\par The two quotes above, in my opinion, depict the curiosity that Darwin had during the voyage. The first quote shows his child-like personality, and how he's willing to try and experience different things. He wants to understand things himself and question everything, questions such as how does this behave, and why do people do things a certain way. The second quote shows how child-like his curiosity is; he's able to visualize events and express them in a powerful way while also being able to dream of their power and beauty. I believe his curiosity was one of the key things that helped him succeed in his research during the voyage as he was able to perceive things differently by approaching them in different aspects. As an individual who is interested in science and technology it is quite obvious to me that a curiosity like his would help; it allows individuals to imagine different scenarios of that topic, and to derive to questions and be able to answer them just using their imaginations.

% My personal opinions

\par Darwin was going through the same issues that an average twenty-something would go through at this point of his life. He was discovering his passion, religious beliefs, and trying to understand what he wanted to do with his life. I think understanding this part allowed me to relate to Darwin was an individual, and also discover that life can change pretty easily when you're given the opportunity. You can discover your path by taking different opportunities like the ones Darwin had. He was higher than average student who had an interest in discovery and exploration, and given the opportunity he used those interests and explored the world. Looking at it now it seems fairly common as many individuals do go around the world searching for themselves, and trying to understand their passions and their beliefs. Darwin definitely didn't give up, and was clearly passionate - he was always getting sea-sick, but he still remained on the voyage until the end, which shows his determination and this also sets him apart from other individuals. 

\epigraph{``In conclusion, it appears that nothing can be more improving to a young naturalist, than a journey in distant countries."}{--- \textup{The Voyage of the Beagle page 375, Charles Darwin}}

\par Darwin realized that he wanted to do research after the voyage as he was already working on writing books and papers, and sending samples back to London. The quote above summarizes his experience as a naturalist, but I think it can still be applied to everything else Darwin experienced. He was able to get first hand opinions of important issues such as slavery, geology, and much more.

\par In conclusion, it seems evident that this trip changed his personality towards having an academic focus, and help him discover his passion. Darwin's voyage of discovery had succeeded more than he had imagined and gave him a chance to understand the world, and also his feelings and views on different issues. For a twenty-something year old there can't be a better experience than by seeing the world, and trying to understand what types of issues you're interested in and finding your passion and that's what I think Darwin got most of out it. He understood what he wanted to research, and what field he wanted to go into. 

\end{document}  