\documentclass[11pt, oneside]{article}
\usepackage{geometry}
\geometry{letterpaper}
\usepackage{graphicx}
\usepackage{amssymb}
\usepackage{setspace}
\doublespacing
\usepackage{epigraph}

% \epigraphsize{\small}% Default
\setlength\epigraphwidth{15cm}
\setlength\epigraphrule{0pt}

\usepackage{etoolbox}

\makeatletter
\patchcmd{\epigraph}{\@epitext{#1}}{\itshape\@epitext{#1}}{}{}
\makeatother

\title{Personal development of Charles Darwin during The Voyage of the Beagle}
\author{Abhi Agarwal}
\date{}

\begin{document}
\maketitle

% Introduction

\par Darwin developed both as a person and as an academic during The Voyage of the Beagle. His view of the world, and his view on God changed during the trip, and he also solidified his interest to become an academic in science and his interest to contribute to science. But there were parts of his personality that stayed the same, such as his curiosity and child-like passion for exploration, which, in my opinion, along with his development ultimately allowed him to explore the parts of science he did and push the theories forward.

% God

\par Darwin was raised as a Nonconformist, attended Shrewsbury School, a Church of England school, then Edinburgh University before moving to Cambridge University. Historically at that time Nonconformists sent their sent their children to Scottish universities, like Edinburgh University, because they had a reputation for the hard sciences and medicine. In contrast to University of Oxford and Cambridge, which had mandatory religious reading specifically on the Anglicanism tradition. At Edinburgh University Darwin became interested in natural history, and most of the professors and reading for natural history, at the time, depicted animals and planets to be creations of God. Darwin was interested in natural history, and so he took his fathers suggestion to pursue a Bacholer of Arts to become a parson of the Anglican Church, a clergyman.

Before the Voyage of the Beagle he was taken by the natural theology by William Paley, and his argument for the existence of an intelligent Creator for the intelligent design of the universe. 

\par Although I did not think much about the existence of a personal God until a considerably later period of my life, I will here give the vague conclusions to which I have been driven.

\epigraph{``The old argument of design in nature, as given by Paley, which formerly seemed to me so conclusive, fails, now that the law of natural selection has been discovered. We can no longer argue that, for instance, the beautiful hinge of a bivalve shell must have been made by an intelligent being, like the hinge of a door by man. There seems to be no more design in the variability of organic beings and in the action of natural selection, than in the course which the wind blows. Everything in nature is the result of fixed laws."}{--- \textup{The Autobiography of Charles Darwin}}

% World View

``Among the scenes which are deeply impressed on my mind, none exceed in sublimity the primeval forests undefaced by the hand of man; whether those of Brazil, where the powers of Life are predominant, or those of Tierra del Fuego, where Death and decay prevail. Both are temples filled with the varied productions of the God of Nature: -- no one can stand in these solitudes unmoved, and not feel that there is more in man than the mere breath of his body.? 

``The map of the world ceases to be a blank; it becomes a picture full of the most varied and animated figures. Each part assumes its proper dimensions: continents are not looked at in the light of islands, or islands considered as mere specks, which are, in truth, larger than many kingdoms of Europe. Africa, or North and South America, are well-sounding names, and easily pronounced; but it is not until having sailed for weeks along small portions of their shores, that one is thoroughly convinced what vast spaces on our immense world these names imply."

% Curiosity

``One day, as I was amusing myself by galloping and whirling the balls round my head" (76)

``The earthquake, however, must be to every one a most impressive event: the earth, considered from our earliest childhood as the type of solidity, has oscillated like a thin crust beneath our feet; and in seeing the laboured works of man in a moment overthrown, we feel the insignificance of his boasted power." (375)

% My personal opinions

Personally, looking through his work, to me, it's quite evident why science-minded individuals were interested in him. 

``In conclusion, it appears that nothing can be more improving to a young naturalist, than a journey in distant countries.? 

\par In conclusion, it seems evident that this trip changed his personality towards having an academic focus, and help him discover his passion. 

\end{document}  