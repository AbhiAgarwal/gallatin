\documentclass[11pt, oneside]{article} 
\usepackage{geometry}                		
\geometry{letterpaper}
\usepackage{graphicx}				
\usepackage{amssymb}
\linespread{1.5}

\makeatletter
\newcommand*{\rom}[1]{\expandafter\@slowromancap\romannumeral #1@}
\makeatother

\title{Response Paper 8}
\author{Abhi Agarwal}
\date{}							

\begin{document}
\maketitle

\par Firstly, I very much enjoyed the approach Slack had to introduce the topic itself before the chapters. He explained is stance on the issue, and that he would believe in a naturalistic explanation or approach even if evolution was proved to be untrue. I thought this was necessary for me to understand his views, and the perspectives he would bring to this trial. He also suggested that he would look forward to finding an naturalistic explanation that hasn't been explored if evolution was disproved. We know that the conclusion was that intelligent design was declared to not be a form of creationism, and knowing his view coming into the book is important as we know which side he is on. 

\par This book showed me a new perspective of what people feel about these issues, and it is something I felt I was fundamentally missing in my knowledge. It seems from this reading that people are against teaching certain things, but aren't comfortable or able to explain why they disagree about those things in detail. Most of the testimonies of this particular trial seem to be individuals who don't have a grasp on the issue or an understanding of their debate. Intelligent design and creationism take opposite approaches and start from opposite sides to reach a conclusion, and I wasn't sure how individuals wanted to argue that one was similar to the other. Intelligent design starts with evidence in nature and seeks to find conclusions it can draw from that evidence. While creationism usually starts with some religious text and tries to understand how findings within nature fit into it. 

\par Moreover, I also really enjoyed his view of religious activities - it resonated with me very much. His view that religious truths change much more than scientific truths help understand his argument much more in depth. He proposes his example by proposing a scenario where you stone people, and that at one point it is okay to stone people, and at other times it is not. 

\par I also wasn't too sure about this particular argument, but I saw the methodology they use for intelligent design similar to scientific experimentation. You start off with an observation, you make a hypothesis, you do some experiments, and then you make a conclusion. In intelligent design they look at nature, make hypothesizes about it, do experimentation to validate it, and then try and relate it back to the creator.  

\end{document}  