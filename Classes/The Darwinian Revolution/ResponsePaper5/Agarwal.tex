\documentclass[11pt, oneside]{article} 
\usepackage{geometry}                		
\geometry{letterpaper}
\usepackage{graphicx}				
\usepackage{amssymb}
\linespread{1.2}

\makeatletter
\newcommand*{\rom}[1]{\expandafter\@slowromancap\romannumeral #1@}
\makeatother

\title{Response Paper 5}
\author{Abhi Agarwal}
\date{}							

\begin{document}
\maketitle

\par The first part of chapter 14 shows how much Darwin believed in his theory as he points out that people who weren't agreeing with his theory were stubborn or ignorant as they weren't even allowing themselves to attempt to believing it, and arguing against it because of their unwillingness to believe that descent with modification is possible. I interpreted it as a way of Darwin saying that he didn't really quite care about what they thought, but he knew his theory was right and that they would agree at some point. I understand Darwin's religious beliefs and dedication to his theory, but I thought Darwin was being insensitive about other people's belief - I didn't see Darwin understanding that it was difficult for some people to change their view of the world so drastically. 

\par Looking at the first section closer it is clear that Darwin fully understood the implications of his theory, and the potential advancements it could cause. It has explained a lot of things in evolutionary biology, and using his theory people have been able to increase their understanding of classification, development of our society and basically the progression of the world. Moreover, for a long time it didn't quite make sense to me why people couldn't believe in evolution and religion in the same way Darwin described. My view is very similar - God created the world, and everything after that is self-sustaining and events that have occurred through random and evolutionary development over millions of years. In addition, it was quite strange to read Darwin's opinions being that there was a Creator, but it also seems scientifically challenging to come up with a theory for it.  The theory of the Big Bang wasn't developed at that time, and we hadn't reached a period where the scientific revolution of how our universe began had happened. It would be interesting to see Darwin's views on the Big Bang model, and if he would agree with it over the first life being created by a Creator.

\par While reading the material it also felt quite strange how an individual is able to change our understanding of the world so drastically. Darwin gave us an understanding of where our species came from, and some understanding of the magic behind the process that has occurred for us to develop the way we are. It also seems fascinating how the well his theory links to theories in biology today - notably genetic variation, and mutation theory. To me, it's a curious question to see how Darwin would be thought about genetics, and its implications if it had been developed at the time. I definitely think that it would have helped him in developing his theory further, and solidifying it by showing clear examples.

\par Lastly it was also quite exciting to see Darwin reference evolution in The Origin of Species. I enjoyed his final words being "evolved". He uses the last paragraph to depict the intensively imagery of evolution and of the theory we've just read about. I felt powerful, and enthralled by his view of the world, and how much work has gone into creating us and I think he uses this to let individuals imagine the progression using his theory by using phrases like "breathed into", "planet has gone cycling", "endless forms", and "beautiful and most wonderful", which all represent our progression, or surroundings, and us.

\end{document}  