\documentclass[11pt, oneside]{article} 
\usepackage{geometry}                		
\geometry{letterpaper}
\usepackage{graphicx}				
\usepackage{amssymb}
\linespread{1.3}

\makeatletter
\newcommand*{\rom}[1]{\expandafter\@slowromancap\romannumeral #1@}
\makeatother

\title{Response Paper 6}
\author{Abhi Agarwal}
\date{}							

\begin{document}
\maketitle

\par In chapter IV it's clear to see how much Darwin benefited from the Voyage, and his ability to understand different kinds of societies. He was able to take his understanding of different societies that he encountered on the Voyage and apply it in some parts of the chapter, and my belief was that people who have tried to explain social behavior weren't able to as they didn't have as much first-hand knowledge as Darwin did about the world. Darwin himself had encountered what he described savage beings, and was able to see the group dynamics and social aspects within tribes. His explanation of `savage' could only be understood by experience, and understanding why individuals behave the way they do. I understood his explanation as: people can be attributed as `savages' when they have a certain attribute that your society doesn't or that attribute doesn't extend further then their tribe, and you can't reason with that attribute. He adds to this in chapter V where he comments 

\par I found some points Darwin made in chapter V to be his continuance to comment on societies. Darwin explains that stronger members of society have to care for the weaker members of society so they can reproduce while the strong men go to war, and it's something that has occurred in history, and in civilizations that have went to war. But, I think he doesn't quite explain it well, and leaves it as an observation that he's made. I think this can be linked to his point about praise, and its link to establishing social behavior. I found it fascinating to compare these two as the strongest men go to war, and when they come back they are treated with praise, which establishes their social behavior and then they becoming desiring of praise and so they go to war again. This whole chapter is very interesting to try and put together as Darwin makes a lot of points about survival of humans, and the type of people who survive. His point about the more creative and intelligent individuals surviving at the beginning of the chapter confuses me, though. In exploring this relationship I got a little confused between survival and evolution as Darwin talks about survival here not strictly evolution.

\end{document}  