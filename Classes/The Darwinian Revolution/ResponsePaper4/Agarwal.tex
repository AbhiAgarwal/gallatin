\documentclass[11pt, oneside]{article} 
\usepackage{geometry}                		
\geometry{letterpaper}
\usepackage{graphicx}				
\usepackage{amssymb}
\linespread{1.5}

\makeatletter
\newcommand*{\rom}[1]{\expandafter\@slowromancap\romannumeral #1@}
\makeatother

\title{Response Paper 4}
\author{Abhi Agarwal}
\date{}							

\begin{document}
\maketitle

\par I always enjoyed the depth of thought that Darwin put into his theories, and chapter \rom{6} was a perfect example for it. I didn't expect coming into the reading that he would explicitly have a chapter trying to find the edge cases in his theory, but I think it really shows his character and his understanding of what he studies. To me his character transitioned from an individual who isn't sure what he wanted to do, to an individual who wants to question a fundamental theory that he holds true, and try and explore every aspect behind it. Chapter 6 inspired me because not many individuals think about things that wouldn't necessarily fit with their theory, and try and understand how they would explain that. In science I was taught to try and find a theory that we assume to be true and find evidence that proves it, but if you find evidence that is even somewhat against it then your theory is incorrect and I would stop working on it. I think Darwin's dedication to what he believed in allowed him to try and understand those confusions, in a way, and try find answers. 

\par In addition, I'm also very surprised at the number of examples Darwin uses to defend his theories, and how useful those examples are to what he tries to depict. He is able to explain theories about flying squirrels, flying lemurs, bats, ants, bees, and much more. I'm surprised even after knowing Darwin's past - how he's able to study the complex structures of all these species, and use them in his examples. To me, individuals of the 19th century were definitely not as well connected as individuals are now, and most academic papers now don't have as many examples as Darwin does even with the power of the Internet. It makes me wonder how large Darwin's book would be if we were to allow him to use Wikipedia or any web-based encyclopedia today. 

\par Some of the questions he proposed puzzled me a little as his explanations were quite interesting, and so were their implications. I was very curious of what Darwin would think about how we see the world as it is at a given point. When I look at different species right now I think they are suited to their surroundings, and I feel like at a given point in time a lot of species that you observe are suited to their surroundings. In my opinion, Darwin's answer would be to say that species that aren't adapted to their surroundings become extinct or change to suit their surroundings. But, I want to know what he says happens in the middle, and this links in to his second point transitional forms. What is happening to the species when they haven't adapted, and aren't fit for their surroundings - the process of evolution is over a long period of time.

\end{document}  