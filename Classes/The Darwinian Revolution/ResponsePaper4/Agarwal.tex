\documentclass[11pt, oneside]{article} 
\usepackage{geometry}                		
\geometry{letterpaper}
\usepackage{graphicx}				
\usepackage{amssymb}
\linespread{1.4}

\makeatletter
\newcommand*{\rom}[1]{\expandafter\@slowromancap\romannumeral #1@}
\makeatother

\title{Response Paper 4}
\author{Abhi Agarwal}
\date{}							

\begin{document}
\maketitle

\par I always enjoyed the depth of thought that Darwin put into his theories, and chapter \rom{6} was a perfect example for it. I didn't expect coming into the reading that he would explicitly have a chapter trying to find the edge cases in his theory, but I think it really shows his character and his understanding of what he studies. To me his character transitioned from an individual who isn't sure what he wanted to do, to an individual who wants to question a fundamental theory that he holds true, and try and explore every aspect behind it. Chapter 6 inspired me because not many individuals think about things that wouldn't necessarily fit with their theory, and try and understand how they would explain that. In science I was taught to try and find a theory that we assume to be true and find evidence that proves it, but if you find evidence that is even somewhat against it then your theory is incorrect and I would stop working on it. I think Darwin's dedication to what he believed in allowed him to try and understand those confusions, in a way, and try find answers.

\par 

\end{document}  