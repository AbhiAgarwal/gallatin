\documentclass[11pt, oneside]{article} 
\usepackage{geometry}                		
\geometry{letterpaper}
\usepackage{graphicx}				
\usepackage{amssymb}
\linespread{1.3}

\makeatletter
\newcommand*{\rom}[1]{\expandafter\@slowromancap\romannumeral #1@}
\makeatother

\title{Response Paper 7}
\author{Abhi Agarwal}
\date{}							

\begin{document}
\maketitle

\par I think one the most interesting things about the Geography of Man in Relations to Eugenics paper was the point about language and its influence on finding a mate. It's something that I understood well, but this paper demonstrated different purposes of it, and I was really impressed by the explanation (We are also reading parts of the Bell Curve for my Quantification class!). I've never taken into consideration the mating of persons with similar defects, such as deaf people. It's interesting to think about them from a eugenics perspective as he claims `it thus causes the intermarriage of dead mutes and the propagation of their physical defect" (293), which makes sense to me on first thought but deaf parents are more likely to have hearing children than the latter. I think this paper looks at it from a grand scheme, and to be fair to him the published statistics is that deaf-parents are 90\% likely to have hearing children and so with the large society there is still an increase. While a single hearing parent, and a deaf parent have a higher chance of having a hearing child. 

\par It's interesting to think about eugenics in a way of inclusion and exclusion of individuals, and I thought it was very important for me to think about it in this way. I imagine deaf people getting married to deaf people because they were excluded away from that group of non-deaf people and came together. I think this principle is central to what they try to describe, and something I realized after thinking about the examples and what they all have similarly. In this sense, I think this is something that is self-correcting because individuals who are in the excluded society are within a smaller set of individuals, and thus would have a smaller set of offspring. Most of those offspring might return back to the society (two-deaf parents have a hearing child), and the excluded society would have a limit to how large it would be.

\par I also understand how it is to have a bias in some way. I here assume that being deaf does not improve our genetic quality of the human population, and I question this because of our Galapagos readings.

\end{document}  