\documentclass[11pt, oneside]{article}   	% use "amsart" instead of "article" for AMSLaTeX format
\usepackage{geometry}                		% See geometry.pdf to learn the layout options. There are lots.
\geometry{a4paper}                   		% ... or a4paper or a5paper or ... 
\usepackage{graphicx}				% Use pdf, png, jpg, or eps§ with pdflatex; use eps in DVI mode
\usepackage{amssymb}

\title{Response Paper on Gal\'apagos}
\author{Abhi Agarwal}
\date{}

\begin{document}
\maketitle

\par Gal\'apagos is based upon theories developed by Charles Darwin, and focuses on what I think are the three main issues that represent the book. First is independence, and the fate of the human race, second is which parts of our existence make us human, and last is the importance of intelligence, and the merit of the human brain. 

\par The majority of the book is set after a financial crisis followed by a disease that makes most humans unable to reproduce. However, a select group of people escape Ecuador, and settle in the fictional island of Santa Rosalia in the Gal\'apagos. After a million years, they become seal-like creatures as they adapt to their environment, and their need to swim for survival. Their story is told by a dead narrator, Leon trout, who is a ghost looking at the island and watching this ``magic" occur.

\par The importance of intelligence, and the merit of the human brain is one of the more important themes of this book. The idea of us having a ``big brain", and it causing us problems is what led to the financial crisis and our need to quantify everything (Quantification in terms of evaluating assets even at the time of collapse). In the book it has said to have led to: distracting us from the most important issues - life and death, confusing us with too much information and the rise of our need to use statistics in everything, in lying, and in the fact that we aren't able to switch it off. The author also gives drinking alcohol as a way to reduce our brain to a smaller size temporarily. We had to adapt in order for survival on that island as the surroundings didn't need for a ``big brain", but required our head being streamlined to allow us to swim.

\par Moreover, there is also focus on the fate of the human race. I know that eventually our fate on that island was to adapt to it and change depending on what we needed over the years. I was a little puzzled by if we evolved wrongly into having big brains because that eventually led to the end of our species (except people who went to the island). I don't really know what the author was implying was the reason for us to die out, but I felt like he was trying to say that we evolved incorrectly, which I don't really know if is possible. I don't think he could be implying that we evolved wrongly.

\par One of the interesting things I found was the asterisks next to character names. I think that was quite strange, and also quite amazing at the same time. In my opinion, they were put next to characters who would die later in the story, and it puts the narrator in his place. It seems like the narrator knows the fate of what's going to happen. I'm not too sure what the asterisks means, but as the characters he puts it on die later in the story therefore I think it must be quite significant. It shows about the narrator and how he reflects on the events that are happening - I still have a feeling that the narrator has some significant role in the Darwinian theory that is at play, but I couldn't quite understand it.

\par Interestingly there has also been a strange theme of dependence on fish. There's a lot of references such as some of the characters being fishermen, or the narrators last name being trout, and also them eventually becoming seal-like creatures. I think these themes were to help foreshadow what was going to be their evolutionary state, and what they would adapt to. There were a couple more interesting links to fishes, and I think it's quite significant, but I also think that I'm reading into it too much. 

\end{document}  