\documentclass[11pt, oneside]{article} 
\usepackage{geometry}                		
\geometry{letterpaper}
\usepackage{graphicx}				
\usepackage{amssymb}
\linespread{1.2}

\title{Response Paper 3}
\author{Abhi Agarwal}
\date{}							

\begin{document}
\maketitle

\par The most fascinating topic was the Darwin's remarks on the distinction between species and variety. He says that the distinction is arbitrary, and his argument being that a ``well marked variety may be justly called an incipient species" seems quite obvious to us, but looking at it from a perspective of his generation it's quite revolutionary. It introduces a level of continuity for each well-marked variety can become its own species at some point, and this sets the scene for his readers as this small sentence depicts, to me, his arguments later on. It gives people an understanding of what Darwin is trying to argue, but I think it only makes sense to me because I know what he's trying to get at.

\par His argument for the ubiquity of variation in nature is quite interesting to interpret without introducing natural selection. He claims that variation exists everywhere, and he makes the claim with selective breeding. It's interesting to see a new perspective because here he's starting small and explaining how animals, and plants do change because we selectively breed and therefore the result is that there is some change. This introduces and plants it into the readers mind that there is some change that occurs over generations, and we know now that there's also a randomness involved because we can't select which genes we pass on.

\par It's also quite clear that Darwin cared about every small change that occurred for each species. He wanted to consider every way a specie could change in the long term. Darwin was strangely good at looking and considering long term effects of each little change, and it becomes quite obvious here because he's able to show examples across long periods of time.

\par We see a different Darwin here. It didn't come across how good Darwin was at deduction and linking topics together until we read this book. It was always noticeable how exception he was, but here we actually see that he was truly exception and how logical and well read he was. His arguments flow perfectly, and it's quite clear how they progress throughout the chapters. He starts with a small argument, and he's able to bring in arguments one thinks are un-relatable and shows link between them. He manages to link things like variation to distinctions in species and variety, and after the chapter it makes sense and I understood the flow of the argument.

\end{document}  