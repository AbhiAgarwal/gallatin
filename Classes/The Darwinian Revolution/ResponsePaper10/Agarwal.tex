\documentclass[11pt, oneside]{article} 
\usepackage{geometry}                		
\geometry{letterpaper}
\usepackage{graphicx}				
\usepackage{amssymb}
\linespread{1.5}

\makeatletter
\newcommand*{\rom}[1]{\expandafter\@slowromancap\romannumeral #1@}
\makeatother

\title{Response Paper 10}
\author{Abhi Agarwal}
\date{}							

\begin{document}
\maketitle

\par In the first page of the Prologue Pagel starts off very strong with the statement: ``culture has worked by coming to exercise a form of mind control over us. We willingly accept and even embrace this mind control, and probably without even knowing it, in return for the protection and prosperity our cultures provide". I believe this is powerful because it encapsulates the fact that social learning gave us culture, but it also has fallouts and tradeoffs. I felt a little strange reading culture and mind control in the same sentence at first. It seemed a little bizarre to switch perspective and look it the way he sees culture to be. 

\par Some of this reading is quite amazing because they have their own interpretation of Evolution - in a sense that they define it a little differently to the others. This is incredible when trying to understand how people perceive Darwin's theory of evolution, and it allows me to try and understand how people thought about it at different times. Pagel writes it as ``Natural selection does not maximize happiness or even well-being, but rather long-term reproductive success" (24). I like the way he proposes this definition because it sets him up to being more open ended like he is later on. It's fascinating, but I didn't quite understand his definition of success yet, which I think is important for me to understand his opinions. 

\par ``Our cultural survival vehicles were built not from coalitions of genes but from coalitions of ideas roped together by cultural evolution" (46). This is I think the foundation of this particular chapter. It's the encapsulation of his argument using evolutionary biology, our neurology, and he describes his vision, which is that culture has evolved with the principles of natural selection in the same way as genes have evolved. 

\par Pagel through this seems like an individual who thinks about grand ideas with his central philosophy, which I think is that the theory of evolution can be applied to everything, and define everything. Moreover, I also feel that the analysis he does becomes less and less strong as he writes further - where strong would mean how convincing it is. 

\end{document}  