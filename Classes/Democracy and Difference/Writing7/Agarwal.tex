\documentclass[11pt, oneside]{article}
\usepackage{geometry}
\geometry{letterpaper} 
\usepackage{graphicx}
\usepackage{amssymb}
\linespread{2.0}	

\makeatletter
\newcommand*{\rom}[1]{\expandafter\@slowromancap\romannumeral #1@}
\makeatother

\title{Essay 7: Young}
\author{Abhi Agarwal}
\date{}

\begin{document}
\maketitle

\par Young proposes a minimal definition of democracy which I think is extremely powerful - ``democratic politics entails a rule of law, promotion of civil and political liberties, free and fair election of lawmakers." (5). I think she's extremely clever here - the minimal nature of the definition allows for her to bring the aspects of inclusion into democracy that she proposes in the next chapter. I missed the definition when I attempted to understand her work at first, but I think her definition of democracy establishes all her principles and encapsulates it. 

\par Alongside her definition of democracy her views on the inclusive democratic practice, in my opinion, expects a lot. Inclusive democratic thinking requires people to be``open to having their own opinions and understandings of their interests change in the process" (6). It's hard to me to realize a society which follows these principles, and it seems like her inclusive principles very quickly break down for me after this sentence. It seems like the people who are unwilling or not able to open to have new opinions are excluded from her view of society, and aren't technically able to contribute to this society (in accordance to her definitions on page 5). I feel like she's also being very clever here to setup her theories and framework. This exclusion would exclude people who are most likely to have problems with her theories - in a sense that people who aren't open to have their ideas change are excluded. This is quite a big majority of individuals - most people are fixed on their principles. It makes sense that she's creating an ideal theory. In addition, this reminds me a lot of Rousseau's work. I'm not too familiar with Young's work and her influencers, but Rousseau seems like his work on democracy and social contract would definitely have inspired her definition and vision of democracy. 

\par My favorite way she expresses her inclusion principle is through establishing how one identifies oneself. The way I understand it, she believes identity to be how one expresses herself and also how one is expressed through being a member of a society. However, I didn't quite understand her feelings towards communities within democracies though. Her inclusion principle was intriguing, but what would she feel about people being excluded from communities within? I think the central questions she tries to answer and ask questions on revolve around this issue. I just didn't think we read enough chapters for me to get a gist of what her consensus was on this issue. Her idea of inclusion scares me a little because I found the principles to be a little vague. 

\par Moreover, something else I noticed that was quite interesting to me, was how she separated individuals and groups. I really enjoyed the idea that a group represented many voices joined together when it pursues justice. The definition of how groups perceive justice in comparison to individuals. 

\end{document}