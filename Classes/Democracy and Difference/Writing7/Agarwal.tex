\documentclass[11pt, oneside]{article}
\usepackage{geometry}
\geometry{letterpaper} 
\usepackage{graphicx}
\usepackage{amssymb}
\linespread{2.0}	

\makeatletter
\newcommand*{\rom}[1]{\expandafter\@slowromancap\romannumeral #1@}
\makeatother

\title{Essay 7: Young}
\author{Abhi Agarwal}
\date{}

\begin{document}
\maketitle

\par Firstly, just looking at the structure of the first couple chapters I found some of the ideas to be scattered at different places and I found it hard to relate some of the things together. However, I also note that we've only read the first couple chapters of the book - so I don't think she manages to form her central thesis yet.

\par One of my favorite quotation in this reading was on page 32. Young writes ``Self-determination, the second aspect of justice as I understand it, consists in being able to participate in determining one?s action and the condition of one?s action; its contrary is domination" (32). 

\par I didn't quite understand her feelings towards communities within democracies though. Her inclusion principle was intriguing, but what would she feel about people being excluded from communities within? I think the central questions she tries to answer and ask questions on revolve around this issue. I just didn't think we read enough chapters for me to get a gist of what her consensus was on this issue. 

\end{document}