\documentclass[11pt, oneside]{article}
\usepackage{geometry}
\geometry{letterpaper} 
\usepackage{graphicx}
\usepackage{amssymb}
\usepackage{setspace}
\doublespacing

\title{Essay 1: Locke, Mills, Kant, and Foucault}
\author{Abhi Agarwal}
\date{}

\begin{document}
\maketitle

\par In `Second Treatise of Government' Locke writes ``[everyone] is born a subject to his father, or his king, and is therefore perpetually a subject who owes allegiance to someone" (Locke, 37), and in the next paragraph writes ``there couldn?t possibly have been so many little kingdoms" (Locke, 37). The series of points here remind me of joint-families in India. Joint-families are very large families who all can be seen as little kingdoms because they usually follow a set of rules that have been passed down through many generations, and I think it's a valid example for this situation as they usually see the father and mother of the house as the government and the set of rules that are proposed are taken as seriously as laws. I would imagine that Locke would find this fascinating as the situation is that there are smaller sets of kingdoms within a large government structure. 

\par In the last paragraph I wasn't able to follow the logic that Kant was trying to make. Kant writes ``a lesser degree of civil freedom gives intellectual freedom enough room to expand to its fullest extent" (Kant, 3). Kant began by writing ``A high degree of civil freedom seems advantageous to a people's intellectual freedom" (Kant, 3) and then added ``yet it also sets up insuperable barriers to it" (Kant, 3). I couldn't follow for whom he was talking about here: was he talking about the governments who had to set up these barriers? Or the individuals to achieve this status? In addition, the next point seemed very different to what, I believe, he was trying to say. It seems like Kant was implying that opposition or some form of resistance was important in order to gain intellectual freedom. This argument seems a little bizarre to me as, theoretically, intellectual freedom would probably be maximized when there is complete civil freedom. I wasn't convinced that any reduction in civil freedom would improve intellectual freedom, and so my question becomes: can Kant make that argument? If so, how does his logic allow him to make that statement?

\par Moreover, after reading Foucault's `What is Enlightenment?' I began to start questioning the purpose of these essays. Enlightenment, as my understanding of Kant's writing, seems to be a decision that is to be made and not something you reach at some point. Is it a choice we make? a decision? Do we have to choose? In the last section of the essay I noticed some Socrates influence when he writes ``The critical ontology of ourselves" (Foucault, 14). It seems at Socrates was also an influence in this particular paper. In addition, I was a little puzzled by what ``faith in Enlightenment" (Foucault, 15) meant?

\par From another class: In the 1970s Foucault also made statements regarding French Prisons. It was a movement ``aimed to allow speaking inmates and mobilization of intellectuals and professionals involved in the prison system" (translated from http://fr.wikipedia.org/\\wiki/Groupe\_d'information\_sur\_les\_prisons). It's very interesting to read his ideology behind public execution and torture as they seem to represent what he truly believed in and tried to change in France. I wasn't aware that he had written and expressed his opinions on Prisons earlier. His point of view that prisons are deeply imbedded within a city rather than being an outsider to the city really show his character and his opinion on the importance on prisons. 

\par Lastly, I wasn't quite sure of the central point Foucault was making in the two chapters we read. They seemed a little disjoint to me and I understand the flow of the information, but I don't understand the central idea or claim he's making. It makes sense that he's talking about a transformation or a shift in the way we punish, and then surveillance at the end. However, it was very enjoyable to see his transition from talking about a single person or a criminal who is being disciplined (tortured) to expanding that out to saying that we're all being disciplined, controlled, and watched in some way.

\end{document}  