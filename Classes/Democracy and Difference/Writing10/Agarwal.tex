\documentclass[11pt, oneside]{article}
\usepackage{geometry}
\geometry{letterpaper} 
\usepackage{graphicx}
\usepackage{amssymb}
\linespread{1.8}	

\makeatletter
\newcommand*{\rom}[1]{\expandafter\@slowromancap\romannumeral #1@}
\makeatother

\title{Essay 10: Butler}
\author{Abhi Agarwal}
\date{}

\begin{document}
\maketitle

\par My favorite part of this book is the first page. Over the last two classes I was thinking about the value of lives to American war-planners or army-men. Specifically the value of lives outside of America to Americans. I think Butler really encapsulates what I was thinking in this quotation: ``certain lives do not qualify as lives or are, from the start, not conceivable as lives within certain epistemological frames, then these lives are never lived nor lost in the full sense" (Butler, 1). The simplicity of the point she makes in accordance to her idea 	of `frames' is powerful and beautiful. I enjoy the idea that the word frame can have so many meanings, and she has utilized it throughout the book in such unique ways. Frame could be a portion of time, or a perspective people take, or being the victim/attacker, and more. 

\par I've studied these perspectives in Artificial Intelligence for a while, but never applied it to our society. In AI we explore the idea where if we get people emotionally connected to a robot then they are more likely to be upset when it's turned off. When they see it get turned off without any emotional attachment then they see it like a machine. It seems very strange that we treat people who aren't like us as machines, but in my perspective - we really do. In the movie `Ex-Machina', there was a female humanoid robot called Ava. Ava was an AI that was designed to clone a human being in every way. The only difference was the external appearance - you can see parts of the internals of the robot). In the movie, she is kept in an underground room and never let out. Next, an individual comes to test her `humanness` and ends up falling in love with her. That individual makes a promise to help her escape. (Spoilers) At the end she betrays him - she has used him as a means of escape and she leaves him at the facility. This helped me realize that people can humanize things that aren't human, and people dehumanize humans. I'm not sure if this part is connected, but it really helped me understand her point. 

\par Do people get genuinely upset that lives have been lost? Or do people get upset at the fact that something bad has been done to people who are similar to them? The terrorist killings in France recently show that it's the latter, which was quite a big shock to me. However, we can't make a fair assessment on the larger audience as it's what the media prioritizes and chooses. 

\par A great point about portrayal of lives is in her `Torture and the Ethics of Photography: Thinking with Sontag' chapter. It's obvious to most people that censorship exists in some form in most countries, but to me it seems like the American people don't believe it exists in the American media. When censorship mixes with the idea that people think there is freedom of speech everywhere, censorship becomes much more powerful. Powerful because people don't even think about the fact that they might not be getting the whole story. In other countries people are aware that censorship exists so they try formulate their stories by looking at outside sources. The specific example is not allowing the pictures of the American troop's coffins to be displayed, but allowing the coffins of the people in the Iraqi War to be displayed. From my personal experience, I think that most people justify the censored pictures by establishing the idea that at least something came out (even if it was censored). Again, the idea of frames is used in a different way here. 

\par 

\end{document}