\documentclass[11pt, oneside]{article}
\usepackage{geometry}
\geometry{letterpaper} 
\usepackage{graphicx}
\usepackage{amssymb}
\usepackage{setspace}
\doublespacing


\makeatletter
\newcommand*{\rom}[1]{\expandafter\@slowromancap\romannumeral #1@}
\makeatother

\title{Essay 4: Wolin, Habermas, Benhabib, Fraser, Mouffe}
\author{Abhi Agarwal}
\date{}

\begin{document}
\maketitle

\par I was quite taken with Wolins definition of `political', `politics', and `democracy'. It's one of the clear definitions that I believe also explains what it encapsulates. His definition of `political' reminded me of the first week when we concentrated on Locke, Mills, Kant, and Foucault, and also the idea of there being a state of nature. I haven't been exposed to a lot of politic theory before this particular course so basic definitions and clear ideas, such as the ones presented here, help me understand the buildup or timeline that democracy or political theory could have followed. It's quite clear to me now that the `political` and the `politics` guides what type of government or politics they form. It's the situation the free society is in it that dictates why they come together collectively and decides on the politics. The politics then decides what type of politics follows, and what type of government follows. It's a little confusing, but through these readings I've began to believe that it's a process - as if each of these particular steps that society goes through. I started establishing this after the class on Locke, Mills, Kant, and Foucault. But I'm not too sure if this is always the case or if I'm right.

\par Habermas presents three models that were very useful in forming an understanding of what type of politics follow after the political (even though it is strictly concerning democracy). Habermas names the three models to be liberal, republican and deliberative politics. The interesting thing for me is to think about is how different societies will look, perceive, and understand the different kinds of politics, and how different societies will adapt ones of these politics. 

\par The one question I am puzzled about is if each system that is setup in the society can be viewed from all three perspectives or is each system setup to follow one of these perspectives? They are fundamentally different perspectives, but even in economics markets are very large and there are many theories to describe them - different people simultaneous use at each of these theories to describe the market. Does this particular state of mind work in political theory when we look at a system and describe its politics? It's kind of confusing because i some of the works I've read it's explicitly mentioned that these theories can be practiced by people in different types of democracies, but does that mean that people practice them together or not?

\par Moreover, I found the difference between Habermas and Mouffe, and their theories to be intriguing and challenging. From the reading I think both Habermas Mouffe would agree that the most difficult question in their views of democracy is how to manage how much power is given to individuals while also taking into accounts the central aims of democracy. Moreover, the other key difference is that Habermas's theory requires there to be a consensus among everyone - a society-wide consensus, which doesn't seem very practical. Habermas's approach to handling the amount of power is just giving everyone equal power. The difference to me seems like it's conflict. The most interesting distinction is how each of their public spheres would differ - one would have an established power to overlook that discourse, and the other wouldn't.

\par I'm not too sure if this is correct, but Mouffe's theory seemed to have an influence from Arendt. His ideas on the public sphere on 247 reminded me of some of the theories that Arendt had on public spheres in The Human Condition. I haven't read enough of Arendt or Mouffe to make this judgement, but I'm curious what the connections are and if he actually was influenced by her work. Their ideology on how the public sphere should be laid out are very similar to me.

\end{document}