\documentclass[11pt, oneside]{article}
\usepackage{geometry}
\geometry{letterpaper} 
\usepackage{graphicx}
\usepackage{amssymb}
\usepackage{setspace}
\doublespacing


\makeatletter
\newcommand*{\rom}[1]{\expandafter\@slowromancap\romannumeral #1@}
\makeatother

\title{Essay 3: Arendt, Beltr\'an, Forment}
\author{Abhi Agarwal}
\date{}

\begin{document}
\maketitle

\par The first chapter of `The Human Condition` by Hannah Arendt was very powerful for me in establishing her understanding of human nature vs human condition. Human condition is something that changes depending on how we adapt to the environment around us, and how we evolve to the place we live in. On the other hand, our human nature is unchanging and something thats within us - something innate. This is very clear in comparison to some of the writing I have read by different authors in which they use human nature and the human condition very interchangeably, and also in works that use the human condition as a philosophical problem. 

\par Moreover, her insights into the shift in the way we think are, for me, revolutionary. Revolutionary in the sense that it shifts my perspective of how I look at the world as the way she reaches the argument is very interesting. She reasons that we've moved from thinking about areas regarding the human condition to more general and abstract ideas. We've stopped talking about the different philosophical aspects of our human condition, and we've started talking about more general ideas such as Mathematics in order to formalize the world. I might be stretching too far, but I got the sense that there was a comment about our human nature, and how we need to formalize things. I was a little confused about making this claim: I don't think this could be true as our human nature doesn't change, but a shift has occurred as we've transitioned over to the modern era. Is it the human condition that has changed, and not the human nature?

\par I'd argue that there wasn't a decline of the public sphere, but a shift in what a public sphere was. People now discuss different things then they used - that is clear, but to my understanding there are just many different types of public spheres now. I believe that because of the increase in specialization people started to think about their specific field, which led to a decline in people discussing the greater ideas. However, people didn't stop discussing ideas - they just started discussing ideas regarding their own specialization and they discussed them in smaller public spheres. They discussed their ideas with individuals who were trained in the same `job' they were trained in. Previously, everyone was trained in Logic, Mathematics, Literature, and the Sciences, and so everyone had a common thread of ideas to discuss in. However, individuals like Newton were specialized in Mathematics and Physics so their public sphere were individuals who were accomplished and trained in those fields, while the public sphere of the individuals who studied Literature was separate. 

\par I was incorrect in the past where I assumed that Arendt was in support of the idea that things have changed. It seems like Arendt prefers the public sphere we used to have where we would discuss ideas such as the purpose of life, and how is one's life satisfied. She's quite disappointed in the way our society has moved and where we are now, and very deeply questions where we are moving, and why we define progress in such a way. To be honest, I disagree with her. For me, we are still discussing what it means to be human, but in a different way. Earlier we should have conversations about the different the meaning of our existence, but now we've used those conversations to try find answers - answers that lie in science, literature and reasoning, and through innovation we're trying to figure out the meaning of what it means to be human. In addition, I also feel like we are progressing given that in the last 200 years we've been able to come up with the Theory of Evolution, Genetic theory, and much more. 

\par Moreover, I do feel like my paper was about a very general idea she presented in the book. The historical perspective she gave was quite amazing, but some of the ideas she presented I strongly disagreed with.

\end{document}