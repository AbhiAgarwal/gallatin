\documentclass[11pt, oneside]{article}
\usepackage{geometry}
\geometry{letterpaper} 
\usepackage{graphicx}
\usepackage{amssymb}
\linespread{2.0}	

\makeatletter
\newcommand*{\rom}[1]{\expandafter\@slowromancap\romannumeral #1@}
\makeatother

\title{Essay 8: Ahmed}
\author{Abhi Agarwal}
\date{}

\begin{document}
\maketitle

\par This book surprised me in a lot of ways. There were a couple things that I thought were positive remarks, but I realized that they were completely wrong and manipulative. The example that stuck with me most was: when parents use the phrase ``I just want you to be happy" to their gay son/daughter - that actually is a warning from their parent as it suggests that their current way of life is not what happiness should be for them. This actually applies to most examples where there exists a difference between the individuals, and I've personally used this phrase many times. It seems like it would make an individual feel better, but it does the opposite and it's hard to realize that. In these situations, you look towards their vision of happiness and seek it which actually creates unhappiness. 

\par Ahmed suggests that there is an expectation from the society or people around us, who have outlined happiness for us, to live up to their view of happiness. Happiness is defined for us in advance, and is defined by the views of the people who define it for us. Any changes we make to these views they have, in their perception, hurts our happiness. It's more of an obligation to follow their view of happiness rather than an expectation. In this way, parents sometimes relate ``I just want you to be happy" to unhappiness because their offspring will have to go through hardships. This creates to a relation in their mind that their offspring's happiness is related to not being gay. To an individual who is still mentally developing, this means that they have to suppress it in order for their parents to be happy and proud of their offspring.

\par I very much enjoyed her premise or idea about objects. I loved her view on an object being just anything you can have a two or one way relationship with (or some kind of affection towards?). The most intriguing view that she proposes is that we filter our objects we dislike and objects we do like to maximize our happiness, and so the view we have of the world are objects that either make us happy or don't make us happy (by extension we should have removed the objects that don't make us happy). 

\end{document}