\documentclass[11pt, oneside]{article}
\usepackage{geometry}
\geometry{letterpaper} 
\usepackage{graphicx}
\usepackage{amssymb}
\linespread{2.1}	

\makeatletter
\newcommand*{\rom}[1]{\expandafter\@slowromancap\romannumeral #1@}
\makeatother

\title{Essay 9: Asad}
\author{Abhi Agarwal}
\date{}

\begin{document}
\maketitle

\par To me, the central point of this book was the idea that suicide bombing is seen very differently than conventional war, and people are much more afraid of suicide bombing than any other type of warfare or terrorism. I think the following quotations really summarizes this: ``My argument, however, is directed against thinking of terrorism simply as an illegal and immoral form of violence and advocates an examination of what the discourse of terror - and the perpetration of error - does in the world of power" (26). In this regard he defines a terrorist as ``someone who creates a sense of fear and insecurity among a civilian population for political purpose" (26). I agreed with how accurate this definition of terrorist was. 

\par Linking this with ``In the long perspective of human history, massacres are not new. But there is something special about the fact that the West, having set up international law, then finds reasons why it cannot be followed in particular circumstances" (94), and ``there is no moral difference between the horror inflicted by state armies ... and the horror inflicted by insurgents" (94). With these four quotations I think he portrays the idea that suicide bombing is not different, but people treat it differently as it seems like a self-sacrifice and has emotion behind it rather than emotionless machines. 

\par I found this book a little bit too much like a lecture. I needed to look up a lot of the content he was referring to during the book. Some arguments he made were a little problematic and I had some disagreements with them. However, overall I definitely thought this book was extremely thoughtful. I really like the way he said some things - notably the idea of the Islamic/Muslim culture being the culture of death, and it being the biggest topic of discourse on terrorism (in our century). 

\end{document}