\documentclass[11pt, oneside]{article}
\usepackage{geometry}
\geometry{letterpaper} 
\usepackage{graphicx}
\usepackage{amssymb}
\linespread{1.75}	

\makeatletter
\newcommand*{\rom}[1]{\expandafter\@slowromancap\romannumeral #1@}
\makeatother

\title{Essay 9: Asad}
\author{Abhi Agarwal}
\date{}

\begin{document}
\maketitle

\par To me, the central point of this book was the idea that suicide bombing is seen very differently than conventional war, and people are much more afraid of suicide bombing than any other type of warfare or terrorism. I think the following quotations really summarizes this: ``My argument, however, is directed against thinking of terrorism simply as an illegal and immoral form of violence and advocates an examination of what the discourse of terror - and the perpetration of error - does in the world of power" (26). In this regard he defines a terrorist as ``someone who creates a sense of fear and insecurity among a civilian population for political purpose" (26). I agreed with how accurate this definition of terrorist was. 

\par The difference between terrorism and war, to me, would be the absence of a country to back it. The aims and methodology could be very similar, but it seems like the difference boils down to the representation. It makes sense that people differentiate between state armies and non-state armies. I don't think it's justified to go to war because it's a `potential threat'. It would be justified, however you justify killing people to protect people, to go to war if it was a direct threat. Terrorist organizations act because they think there is a potential threat to their society or religion, and so it becomes less justified for them to perform killings. Countries do the same! They go into war because of a potential threat that may or may not be a threat, and people somehow buy that. Why do people buy this? Well people within that country want to be protected, but why is the average person outside that country okay with it? 

\par Linking this with ``In the long perspective of human history, massacres are not new. But there is something special about the fact that the West, having set up international law, then finds reasons why it cannot be followed in particular circumstances" (94), and ``there is no moral difference between the horror inflicted by state armies ... and the horror inflicted by insurgents" (94). With these four quotations I think he portrays the idea that suicide bombing is not different, but people treat it differently as it seems like a self-sacrifice and has emotion behind it rather than emotionless machines. 

\par I found this book a little bit too much like a lecture. I needed to look up a lot of the content he was referring to during the book. Some arguments he made were a little problematic and I had some disagreements with them. However, overall I definitely thought this book was extremely thoughtful. I really like the way he said some things - notably the idea of the Islamic/Muslim culture being the culture of death, and it being the biggest topic of discourse on terrorism (in our century). 

\par I still don't think there is a connection to terrorism and the Islamic/Muslim culture. I do, however, feel like for some reason there are just a higher frequency of muslim/islamic people who are performing these actions (at least to American media). I think, we, inherently believe statistics very easily, and we now most Americans only know terrorists who are Muslim/Islamic so it does make sense for them to automatically assume that they did it because they were Muslim/Islamic. This is extremely wrong and this is something that should not happen, but it's something we do with everything. Moreover, it seems very bizarre that Islamic-American citizens are very likely to do such actions - reading comments from people who knew the Boston bombings individuals - they were extremely normal when they first came into the U.S. Does the US suppress or anger people? Seems extremely strange!

\end{document}