\documentclass[11pt, oneside]{article}
\usepackage{geometry}
\geometry{letterpaper} 
\usepackage{graphicx}
\usepackage{amssymb}
\linespread{2.0}	

\makeatletter
\newcommand*{\rom}[1]{\expandafter\@slowromancap\romannumeral #1@}
\makeatother

\title{Essay 6: Alexander}
\author{Abhi Agarwal}
\date{}

\begin{document}
\maketitle

\par Prior to reading this I had no clue what the Jim Crow laws were. This is quite a huge gap in my historical knowledge - mostly because I did not ever study the American history. I was very surprised that I just heard about this - the population of people in prison now is actually very close to the population of people in prison during the Soviets. The Soviets had about 850 people in prison per 100,000 people while the United States has 800 people in prison per 100,000. This is quite bizarre to me as the Soviets had a large amount of prisoners because they were constantly under war. 
\par The metric I was shocked most by was the fact that African Americans constitute of 14\% of the American population, but 40\% of the prison population. She mentioned a couple factors that leads to this, but the biggest factor that I believe is most important is the fact that they rely on public defenders over highly rated defense attorneys. I definitely do agree with this in most cases. It seems to me that money is the single driving force in America, and African Americans are in a huge disadvantage because of racism and financial causes. Every other race is not at a huge disadvantage because either they are financial stable or people aren't racist against them in the same degree. 
\par The system in American is kind of a shock to me. In India we have discrimination within our race. India has a caste system, which I can explain with an example. A type of caste in India is the sweepers, which are basically people who are maids, chimney sweepers, and people who sweep the road. In my mother's parents household in India, the maids have to shower before they enter the kitchen after they have swept something. This is because these people are said to be impure, and need to physically be cleaned before they enter the kitchen (Not the case with any other caste). I bring up this example because it's horrible that we're doing this against them and its incredibly wrong, but they are brought up in this way and understand how to proceed in their own society. The difference is that they aren't bashed by anyone for being different, and for being in that role. They have a clear and established role in their society. The issues in India are much much worse, but at least the police don't get involved in killing individuals just because they are different or doing something wrong. 
\par In Thailand there is isn't much racism because most of the cities in Thailand are high tourist spots. Racism does exist, but it doesn't exist with people who are high up in authority and/or people high up in authority don't act upon racism. It seems like tourist-destinations are more unlikely to have a lower-amount of racism. Moreover, most asian countries are also becoming less racist because the other races are becoming equally as wealthy as their own. For example, my parents used to work in Hong Kong about 25 years ago, and then people used to be racist to Indians because they weren't wealth - they were racist by not letting Indians enter their shops. As Indians grew businesses and got wealthier, people started to get less and less racist against them. I make the conjecture that if African Americas had the same standard of living and wealth as the average white family then over a period of time the racism would decrease. This is just looking at my own experiences. 
\par In conclusion, the rest of the world is very scared of the prison system in the United States and now I understand why. One thing that this book has made really clear is that people (both upper and middle class) seek to make a profit from racism. Most of my writings are just me reflecting back on my ideas after reading this book. 

\end{document}