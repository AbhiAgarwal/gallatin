\documentclass[11pt, oneside]{article}
\usepackage{geometry}
\geometry{letterpaper} 
\usepackage{graphicx}
\usepackage{amssymb}
\usepackage{setspace}
\doublespacing

\makeatletter
\newcommand*{\rom}[1]{\expandafter\@slowromancap\romannumeral #1@}
\makeatother

\title{Essay 2: Rousseau}
\author{Abhi Agarwal}
\date{}

\begin{document}
\maketitle

\par Rousseau differentiates between the different types of inequalities - natural and moral. It makes perfect sense to me that a natural inequality would lead to create of moral inequalities, but I wasn't quite sure if he believed that a moral inequality would only be created if there was a natural inequality. Would there still be a moral inequality if we did not have any natural inequality? My understanding is that when there is a natural inequality the more intellectually or physically gifted individuals have superiority in some way thereby when they perform an action people are more likely to follow them. This follows Rousseau's principle that an individual claimed to own this particular part of the land, and because she was naturally superior to everyone else people followed. However, if this was not the case and there was no natural inequality then I believe that there wouldn't be any moral inequality either because people wouldn't be able to influence each other. My question becomes that is physical inequality the reason we have moral inequality? 

\par In addition, for a while I was pondering the idea that following another individual because of his superiority seems to be reasoning, but I also realized that there following or being with a group of individuals seems to be an instinct and something I believe now to come with natural laws. I might also be completely wrong here, but does being territorial come into being an instinct or within the natural law or am I misunderstanding the concept? The conditions I set in the last paragraph for there being no natural inequality were me assuming there was no concept of being territorial. Rousseau spends a lot his time in the Introduction and in the Preface talking about humans and their link to animals. Are we territorial in the same way that animals are? If so then wouldn't that mean that regardless of our natural inequalities it would have led to there being moral inequalities? My reasoning here is because we have instincts to mark, protect, and be territorial about our land and also to return to the same place over and over again. 

\par Rousseau discusses Hobbes' theories in Part \rom{1} of the book. My understanding was that Rousseau's only issue with Hobbes was his idea of conflict between individuals. Rousseau's philosophy was that people were neither good nor bad, and he takes a less cynical view than Hobbes does on the state of nature. I haven't read Hobbes before, but is Hobbes saying that people were irrational? Or were they still rational and just were at war and feared each other? This led me to think about my earlier point. Wouldn't this imply that individuals who were more superior and gifted in terms (there is natural inequality) would be dominant because they would come out on top? This, in my opinion, would also ultimately create moral inequality. I resonated with Hobbes' explanation of the state of nature more because, to me, it would definitely lead to the creation of a government. 

\par Moreover, just going through this particular thought process - I've also started beginning to understand the more fundamentals of why government institutions could have been originally formed. I think it was wrong of me to compare both Hobbes and Rousseau in this particular way because now I believe each of their work could apply to a different civilization and could be the right way to view their circumstance. Different governments get formed because of different actions that occur, and in one society there could be more conflict between individuals, and in one none. These lead to different government structures as you're uniting for different reasons, one for purely protection and one to bring people into harmony. 

\par A lot of my analysis lies within trying to understand the link between physical inequality and moral inequality. I was fascinated by a lot of his other points, but trying to understand this particular topic intrigued me. I wanted to see the connection of how one could have led to the other. I was mostly intrigued because I wanted to figure out if we would inevitably have moral inequality in our society. I also find it quite hard to try and explore this argument as the `state of nature' is a world without reason.

\end{document}