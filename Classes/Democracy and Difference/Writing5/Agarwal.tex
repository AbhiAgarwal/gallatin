\documentclass[11pt, oneside]{article}
\usepackage{geometry}
\geometry{letterpaper} 
\usepackage{graphicx}
\usepackage{amssymb}
\linespread{2.0}


\makeatletter
\newcommand*{\rom}[1]{\expandafter\@slowromancap\romannumeral #1@}
\makeatother

\title{Essay 5: Norton}
\author{Abhi Agarwal}
\date{}

\begin{document}
\maketitle

\par My favorite quote and example from Norton in her book was on page 21. Norton writes ``Americans in Disneyland do not mistake it for reality. Rather, recognizing it as a representation of desire, they celebrate their collective capacity to produce a world more rational and more rewarding than that which Providence supplies them. In their play, as in their politics, they know themselves as the creators of a new world order". A central point Norton made was that we express our freedom and the idea that we have choice through how we consume things and it impacts our consumption patterns. I felt like this was one of the key examples that came out to be regarding this point. I have some issues with this statement (discussing later), but having choice and freedom allows Americans to mould and adapt their own world as they become leaders of what is to be consumed and drive that market force. I felt like Norton was trying to say something deeper with ``Americans in Disneyland do not mistake it for reality", but I couldn't figure out what she meant. 

\par I enjoyed the idea that people use consumption as signs to express themselves and by extension their freedom. This is a very capitalistic idea and is a bedrock in modern economics theory. In addition, I also enjoyed the idea of every consumption and every activity is important in the development of society. This point really hit me when I started to realize that the biggest concern most people had with Soviet Union was their consumption patterns, and how people have a scarcity in their choices. People are fixated and happier because they have a large amount of choices and people are also more politically active because politics gives them access to this freedom. 

\par I loved the content in this book, but I disliked the writing style as I felt Norton was a little too assertive and general in making her claims. All of her claims about societies considered the whole society and not a subset of society or a selection. For example, when she's describing ``Americans in Disneyland" she talks about all Americans that go to disneyland and not a couple or some of them. This writing style kind of threw me as I found it hard to believe that some of the assertions she made applied to everyone. It's understandable because she's writing to generalize signs in society itself, but some of her statements such as the Disneyland statement does not apply to every American. To me it seems hard to believe that all Americans resonate and think about things in very similar ways - just by observing polls and the divisions in thought, to me it seems unlikely. 

\par It's also interesting to kind of understand how her experiences shaped her writing. Just by doing background research, her family moved around the world a lot as her father was in the U.S. Navy. She was also born in the 1950s, and so this suggests that she moved around during the Cold war. This kind of explains why she is so passionate and fixated on consumer cultures in America, and also on consumer cultures in general. She seemed to have experience with a wide variety of consumer cultures, and so this might also explain why she was confident in asserting some of her points (direct experiences). 

\par In conclusion, I really enjoyed this reading and after this reading I begun to see to world a little differently. I constantly use some of the analysis she does to view the world, and it's very interesting to use her ideas in everyday life. I couldn't quite figure out how she looks at the world in this particular way. I was very fixated in trying to understand her past to understand her work better, but I still couldn't figure it out.

\end{document}