\documentclass[11pt, oneside]{article}   	
\usepackage{geometry}                		
\geometry{letterpaper}
\usepackage{graphicx}	
\usepackage{amssymb}
\usepackage{epigraph}

% \epigraphsize{\small}% Default
\setlength\epigraphwidth{15cm}
\setlength\epigraphrule{0pt}

\usepackage{etoolbox}

\makeatletter
\patchcmd{\epigraph}{\@epitext{#1}}{\itshape\@epitext{#1}}{}{}
\makeatother

\title{Essay 3 Proposal}
\author{Abhi Agarwal}
\date{}							

\begin{document}
\maketitle


\par I wanted to concentrate on developing three ideas in my essay, and then evaluating these three ideas. The first is looking at an explanation of how individuals have defined intelligence and learning, or the acquisition of knowledge, in the past, and looking at work from a philosophical perspective. Then to explore how Jeff Hawkins defines and looks at intelligence, and the models he defines. The final step would be to look at traditional approaches of how intelligence or learning has been done in Computer Science, and questioning the approaches they take.

\par Scope of evidence/sources: 
\par For background on neural network approaches: Jeff Hawkins - Intelligence
\par For background approaches other than neural networks: Pattern Recognition and Machine Learning, Machine Learning: A Probabilistic Perspective, and Foundations of Machine Learning
\par To define Intelligence: Artificial Intelligence: The Modern Approach

\par Argument/thesis:
\par Which approach to intelligence, in Computer Science, fits how we have defined intelligence? Is it the approach of using mathematical models or is it the approach of using a neuroscience approach? It seems like the neuroscience approach should be more key because it's the same way we learn, but are the approaches that the one-learning algorithm takes fit our definition of intelligence?  
\par My argument is that both these algorithms will represent pseudo-intelligence or artificial intelligence, but not true-intelligence, which is the goal of its developers. 

\par How you plan to support this argument with your sources:

\par 1. Looking at an explanation of how people have defined intelligence.
\par 2. Looking at how Jeff Hawkins defines and looks at intelligence, and modeling it.
\par 3. Looking at the traditional approaches to learning, and how people did it before Jeff Hawkins approach.

\par

\end{document}  