\documentclass[11pt, oneside]{article}   	
\usepackage{geometry}                		
\usepackage{graphicx}	
\usepackage{amssymb}
\usepackage{epigraph}
\usepackage[doublespacing]{setspace}
\usepackage{etoolbox}

\setlength\epigraphwidth{15cm}
\setlength\epigraphrule{0pt}
\geometry{letterpaper} 
\makeatletter
\patchcmd{\epigraph}{\@epitext{#1}}{\itshape\@epitext{#1}}{}{}
\makeatother

\title{Approaches to intelligence in machine learning}
\author{Abhi Agarwal (abhia@nyu.edu)}
\date{}							

\begin{document}
\maketitle

% Introduction
\par Intelligence and why create a framework for intelligence?

\par ``Artificial Intelligence (AI) was the first large scientific community, established already in the mid 1950s, working on problems that require intelligence to be solved" (W\l{}odzis\l{}aw, 1).

\par Hawkins is an electrical engineer, and hasn't had any professional experience in neuroscience. His framework approaches the problem from an engineer's perspective as well as his personal study of the research done on the cerebral cortex to formulate his framework.

\par Jeff Hawkin's Intelligence

\par Intelligence in the Machine Learning community with just a mathematical training

\begin{thebibliography}{9}
\bibitem{1}
	Hawkins, Jeff, and Sandra Blakeslee,
	\emph{On Intelligence}.
	New York: Henry Holt, 2005.
	Print.
\bibitem{2}
	Duch, W\l{}odzis\l{}aw.
	"What Is Computational Intelligence and What Could It Become?" Challenges for Computational Intelligence.
	2007: n. pag. Print.
\end{thebibliography}
\end{document}  