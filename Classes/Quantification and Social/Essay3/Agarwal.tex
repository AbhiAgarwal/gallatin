\documentclass[11pt, oneside]{article}
\usepackage{geometry}
\usepackage{graphicx}
\usepackage{amssymb}
\usepackage{epigraph}
\usepackage[doublespacing]{setspace}
\usepackage{etoolbox}
\geometry{letterpaper}

\setlength\epigraphwidth{15cm}
\setlength\epigraphrule{0pt}
\makeatletter
\patchcmd{\epigraph}{\@epitext{#1}}{\itshape\@epitext{#1}}{}{}
\makeatother

% Exploring approaches to define intelligence, and intelligence frameworks in machine learning

\title{Quantifying intelligence using aspects of machine learning}
\author{Abhi Agarwal (abhia@nyu.edu)}
\date{}							

\begin{document}
\maketitle

% ~ 4000 words
% Introduction
\par 

% DEFINING INTELLIGENCE

% Intelligence build-up
\par 
% What is intelligence? 
Intelligence has been defined as many different things, and each of these definitions have allowed us to quantify or understand intelligence in a different way. Intelligence for a particular individual, in the past, has been defined to be the knowledge gained, accomplishments, ability to understand other individuals, your community/society,  an ability to reason, and much more. 
% Quantifying intelligence - what it is and how we can do it
The idea of trying to quantify intelligence or measuring intelligence was first introduced by the field of phrenology, and Franz Joseph Gall who is known to be the founder of the field. Franz Joseph Gall wanted to try study the localization of the mental functions in the brain by observing skull sizes, and facial features of people. Even though the field of phrenology wasn't ultimately success, some of the concepts such as attempting to quantify intelligence intrigued the scientific community. 

\par Intelligence and why create a framework for intelligence?

% DEFINING APPLICATIONS OF INTELLIGENCE

% Defining/explaining/understanding intelligence
\par How people have quantified intelligence in the past - IQ tests. Done by powerful individuals for gains (help recruitment for military). 
\par ``Artificial Intelligence (AI) was the first large scientific community, established already in the mid 1950s, working on problems that require intelligence to be solved" (W\l{}odzis\l{}aw, 1).
\par Quantifying and defining intelligence and frameworks around it became important when 


% DEFINING INTELLIGENCE FRAMEWORKS
% ASPECTS/CHARACTERISTICS OF INTELLIGENCE AND HOW THEY ARE APPLIED BY EACH FRAMEWORK
% i.e.: HOW IS REASONING DONE IN EACH? IN THE MATHEMATICAL EXPLANATION REASONING IS DONE BY BOOLEAN SATISFIABILITY PROBLEMS.

% One way of defining Intelligence in Machine Learning
\par Hawkins is an electrical engineer, and hasn't had any professional experience in neuroscience. His framework approaches the problem from an engineer's perspective as well as his personal study of the research done on the cerebral cortex to formulate his framework.
\par Jeff Hawkin's Intelligence

% Next way of defining intelligence in Machine Learning
\par Intelligence in the Machine Learning community with just a mathematical training

\begin{thebibliography}{9}
\bibitem{1}
	Hawkins, Jeff, and Sandra Blakeslee,
	\emph{On Intelligence}.
	New York: Henry Holt, 2005.
	Print.
\bibitem{2}
	Duch, W\l{}odzis\l{}aw.
	"What Is Computational Intelligence and What Could It Become?" Challenges for Computational Intelligence.
	2007: n. pag. Print.
\end{thebibliography}
\end{document}  