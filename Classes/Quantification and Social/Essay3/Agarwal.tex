\documentclass[11pt, oneside]{article}
\usepackage{geometry}
\usepackage{graphicx}
\usepackage{amssymb}
\usepackage{epigraph}
\usepackage[doublespacing]{setspace}
\usepackage{etoolbox}
\geometry{letterpaper}

\setlength\epigraphwidth{15cm}
\setlength\epigraphrule{0pt}
\makeatletter
\patchcmd{\epigraph}{\@epitext{#1}}{\itshape\@epitext{#1}}{}{}
\makeatother

% Exploring approaches to define intelligence, and intelligence frameworks in machine learning

\title{Exploring methods of quantifying intelligence using theories in Machine Learning}
\author{Abhi Agarwal (abhia@nyu.edu)}
\date{}							

\begin{document}
\maketitle

%% OUTLINE %%
% Define intelligence
% Show that individuals do it
% Show where it came from
% Show why academics wanted to do it - AI community, machine learning
% Show why the intelligence community wanted to quantify intelligence. 
% Trying to show that everyone quantifies intelligence for some reason

% Look at Machine Learning aspects 
% Look at Jeff Hawkins approach
% Look at traditional approaches to machine learning

% Look at the aims and problems that Jeff Hawkins and traditional approaches are trying to solve
% Are the aims different? 

% What are their approaches?
% Do they try do the same thing?

% Is it better to Quantify intelligence either way?
% Why are we as humans the intelligence standard they look up to? Are we what they want intelligence to be?
% They are trying to replicate our understanding and how we learn, and how our brain does intelligence - is that right? Here Jeff Hawkins is in the argument
% Others are trying to find an objective definition of intelligence, and trying to approach intelligence from a perspective of mathematics and trying to look at all the variables that make up intelligence.  Looking at features "what they call it in machine learning", and trying to simulate these

% ~ 4000 words
% Introduction
\par 

% DEFINING INTELLIGENCE

% Intelligence build-up
% What is intelligence? 
% Define intelligence
\par Intelligence has been defined as many different things, and each of these definitions have allowed us to quantify or understand intelligence in a different way. The simple act of telling people how intelligent she is, is the most basic act of quantifying intelligence that we do day-to-day. By stating that a particular individual is intelligent we could mean a couple things. For example, we could be commenting on the ability to gain knowledge at speed, their accomplishments, their society/community group, their an ability to reason, and much more. 
% Show that individuals do it
These are certain characteristics that we think about when making a judgement about an individual's intelligence, and the majority of society does this. The most intelligent individuals part-take in this by accepting awards that claim they have a high IQ such as high IQ society awards or genius grants, and less intelligent individuals part-take in this as well by observing this phenomenon and participating by discussing it. Therefore, there is an inherit part of our society and the way we perceive of the world that needs to compare or judge intelligence.
% Quantifying intelligence - what it is and how we can do it
% Show where it came from
\par The idea of trying to quantify intelligence or measuring intelligence was first introduced by the field of Phrenology, and Franz Joseph Gall who is known to be the founder of the field. Gall wanted to try study the localization of the mental functions in the brain by observing skull sizes, and facial features of people. Even though the field of Phrenology wasn't ultimately success, some of the concepts such as attempting to quantify intelligence intrigued the scientific community. Moreover, another individual we studied in class that tried to attempt to define intellectual ability was Samuel George Morton. Morton, similarly, in his work Crania Americana claimed in his paper that you could measure the intellectual ability of a race by their skull capacity. 

% Defining/explaining/understanding intelligence
\par This idea become more popular when Alfred Binet and Theodore Simon, in France, designed the first wide-used intelligence test known as the Binet-Simon Scale. Binet personally believed that intelligence is too broad of a concept to quantify with a single numerical value. However, he did agree that intelligence is influenced by a number of factors, and can be compared if broken down into its parts. In 1916, the Binet-Simon Scale was brought to the U.S. to Stanford University and researchers adapted it to become the Intelligent Quotient or IQ. The reason for the U.S. to create an intelligent test, as we read in class, was to screen army recruits during World War 1. In addition, IQ tests were also used to screen immigrants as they arrived at Ellis island, and became increasingly more useful to governments as the century progressed. 

\par Throughout the 20th century the IQ test was used to filter out individuals in different ways. It has been used to filter individuals for prizes and grants, for military and government recruitment, in medicine, job interviews, universities, and much more. We're depended on ways to quantify intelligence in order to pick individuals out, and narrow down our search fields in the same way we have used strength or speed. It's becoming increasingly more important as we move from jobs being in the primary and secondary sector to the tertiary sector. 

% Show why academics wanted to do it - AI community, machine learning
\par In the same way in Computer Science there has been a big push from giant organizations to make computers increasingly more intelligent to help reduce costs. This push, as well as a goal for Computer Scientists to create artificial life, has driven Computer Scientists to define intelligence. Since the development of computers there has been a vision of creating an intelligent agent, and the community of individuals studying Artificial Intelligence have had a vision and different approaches to solving this problem. An intelligent agent is an autonomous entity that is goal driven and uses previous knowledge or learns to reach its goal.

\par In 2007 W\l{}odzis\l{}aw published a paper on computational intelligence, and wrote ``Artificial Intelligence (AI) was the first large scientific community, established already in the mid 1950s, working on problems that require intelligence to be solved" (W\l{}odzis\l{}aw, 1). Intelligence is important to the field because it's an inherent part of building artificial life, and to build systems that mimic human life. Most of the Artificial Intelligence community is still debating on ways to solve this problem as building an intelligent agent inherently requires a mathematical or computational definition of intelligence. It's required in order for the intelligent agent to learn, and be able to make intelligent decisions on its own. Building an intelligent agent requires for us to define a program algorithmically that would allow it to act autonomously, and in order to do that we need some methodology or steps that it could follow to learn.

\par Given the limitations in terms of computational power in pre-2000, research in building an intelligent agent branched of into two perspectives. The first was looking at this problem from a mathematical perspective, and the second was exploring neuroscience and the human anatomy to apply the principles of our intelligence to building an intelligent agent. Building an intelligent agent from a mathematical perspective is looking at features, or aspects of intelligence and trying to model them mathematical in order to later bring them together into one coherent model. The latter looks at how our cerebral cortex was formed and borrows the architecture and the way neurons make connections between things we learn to apply the same methodology to learning as our brain does. Since we are trying to mimic human beings in creating intelligent agents the theory of learning from our brains became a desirable theory in the community. 

\par During the next decade individuals worked on advancing these fields theoretically until they reached a point where processing power would catch up with their research. The decade gave both the branches time to grow their communities and to improve traction around their work, and this created a separation in ideology and the way individuals dealt with the aspects of quantifying intelligence. 

\par Intelligence and why create a framework for intelligence?
\par Quantifying and defining intelligence and frameworks around it became important when


% DEFINING APPLICATIONS OF INTELLIGENCE
% DEFINING INTELLIGENCE FRAMEWORKS
% ASPECTS/CHARACTERISTICS OF INTELLIGENCE AND HOW THEY ARE APPLIED BY EACH FRAMEWORK
% i.e.: HOW IS REASONING DONE IN EACH? IN THE MATHEMATICAL EXPLANATION REASONING IS DONE BY BOOLEAN SATISFIABILITY PROBLEMS.

% One way of defining Intelligence in Machine Learning
\par Hawkins is an electrical engineer, and hasn't had any professional experience in neuroscience. His framework approaches the problem from an engineer's perspective as well as his personal study of the research done on the cerebral cortex to formulate his framework.
\par Jeff Hawkin's Intelligence

% Next way of defining intelligence in Machine Learning
\par Intelligence in the Machine Learning community with just a mathematical training

\begin{thebibliography}{9}
\bibitem{1}
	Hawkins, Jeff, and Sandra Blakeslee,
	\emph{On Intelligence}.
	New York: Henry Holt, 2005.
	Print.
\bibitem{2}
	Duch, W\l{}odzis\l{}aw.
	"What Is Computational Intelligence and What Could It Become?" Challenges for Computational Intelligence.
	2007: n. pag. Print.
\end{thebibliography}
\end{document}  