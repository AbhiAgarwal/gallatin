\documentclass[11pt, oneside]{article}   	
\usepackage{geometry}                		
\usepackage{graphicx}	
\usepackage{amssymb}
\usepackage{epigraph}
\usepackage[doublespacing]{setspace}
\usepackage{etoolbox}

\setlength\epigraphwidth{15cm}
\setlength\epigraphrule{0pt}
\geometry{letterpaper} 
\makeatletter
\patchcmd{\epigraph}{\@epitext{#1}}{\itshape\@epitext{#1}}{}{}
\makeatother

\doublespacing
\title{One learning algorithm}
\author{Abhi Agarwal (abhia@nyu.edu)}
\date{}							

\begin{document}
\maketitle

\par The one learning algorithm is a concept in machine learning and neuroscience that has been studied extensively by Jeff Hawkins. In his book, On Intelligence, he laid out a memory-prediction framework depicting what he thinks the one learning algorithm would consist of. His framework is based upon the work on columnar organization of the cerebral cortex done by Vernon Mountcase who shows that "all regions of the cortex [perform] the same operation" (Hawkins, 51), and that ``there is a common function, a common algorithm, that is performed by all the cortical regions" (Hawkins, 51). In essence, the one learning algorithm accurately predicts outcomes of future input sequences based on patterns that this algorithm learns from regardless of what type of patterns they are (i.e.: can be images, videos, etc.).

\par Hawkins is an electrical engineer, and hasn't had any professional experience in neuroscience. His framework approaches the problem from an engineer's perspective as well as his personal study of the research done on the cerebral cortex to formulate his framework. He treats the network of neurons in our brain as an encoding problem, and he quotes ``All your brain knows is patterns" (Hawkins, 56), which is truly how a mathematician would see it as patterns imply that some formulation could be found to connect the pattern together. He explains that our understanding of the world is based upon patterns and ``[correct] predictions result in understanding. Incorrect predictions result in confusion and prompt you to pay attention' (Hawkins, 89). 

\par Hawkins key aim is to create an algorithm that mirrors how we learn, and form what he calls `true intelligence' rather than forming `artificial intelligence'. He defines intelligence as a ``[measure of the] capacity to remember and predict patterns in the world, including language, mathematics, physical properties of objects, and social situations" (Hawkins, 97). Hawkins' view of how our brain recognizes patterns and learns is that our ``brain doesn't ``compute" the answers to problems; it retrieves the answers from memory ... entire cortex is a memory system. it isn't a computer at all'' (Hawkins, 68). Thus, Hawkins sees the brain as being able to predict the answer or the best outcome of a certain task by having a very efficient prediction system that utilizes the connection of neurons in our brain to find the best prediction. This is in contrast to how most individuals think about learning and how the neurons in the brain responses to actions. Most individuals believe that the reactions to actions are, in a sense, computed at that particular moment and our brains are built on logic, assumptions, and prior knowledge. Is learning something we can model as patterns? Can how we think be quantified as a series of steps or is it more complex?

\par Approaching this algorithm from an engineering perspective, a mathematical model of learning would ignore a lot of things that we believe contribute to learning, and our definition of learning. We can define learning to be ``process of acquiring modifications in existing knowledge, skills, habits, or tendencies through experience, practice, or exercise" (Merriam-Webster, ``Learning"). 

\par The thesis I'm deriving to is that can learning by formualized? This book clearly approaches learning from a perspective that all of 

\par The audience , just looking at the feedback.

\epigraph{``All the information that enters your mind comes in as spatial and temporal patterns on the axons.''}{(Hawkins, 57)} 

\epigraph{``The brain quickly learns to interpret the patterns correctly.''}{(Hawkins, 61)} 

\epigraph{``Patterns are the fundamental currency of intelligence that leads to some interesting philosophical questions.''}{(Hawkins, 63)} 

\epigraph{``Neurons are quite slow compared to the transistors in a computer.''}{(Hawkins, 66)} 

\epigraph{``The brain is a parallel computer.''}{(Hawkins, 66)} 

\epigraph{``The memory of how to catch a ball was not programmed into your brain; it was learned over years of repetitive practice, and it is stored, not calculated, in your neurons.''}{(Hawkins, 69)} 


\begin{thebibliography}{9}
\bibitem{1}
  Hawkins, Jeff, and Sandra Blakeslee,
  \emph{On Intelligence}.
  New York: Henry Holt, 2005.
  Print.
\bibitem{2}
  Merriam-Webster,
  \emph{Learning}.
  Merriam-Webster, n.d. Web. 05 Nov. 2014.
  Electronic.
\end{thebibliography}
\end{document}  