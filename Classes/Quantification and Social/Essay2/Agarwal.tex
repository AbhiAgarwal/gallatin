\documentclass[11pt, oneside]{article}   	
\usepackage{geometry}                		
\geometry{letterpaper}
\usepackage{graphicx}	
\usepackage{amssymb}
\usepackage{epigraph}

% \epigraphsize{\small}% Default
\setlength\epigraphwidth{15cm}
\setlength\epigraphrule{0pt}

\usepackage{etoolbox}

\makeatletter
\patchcmd{\epigraph}{\@epitext{#1}}{\itshape\@epitext{#1}}{}{}
\makeatother

\title{One learning algorithm}
\author{Abhi Agarwal}
\date{}							

\begin{document}
\maketitle

\par The one learning algorithm is a concept in Machine Learning that has been studied extensively by Jeff Hawkins, and in his book On Intelligence he laid out a memory-prediction framework echoing what he thinks the one learning algorithm would consist of. His framework is based upon the work done by Vernon Mountcase who shows that "all regions of the cortex are performing the same operation" (Hawkins, 51), and that ``there is a common function, a common algorithm, that is performed by all the cortical regions" (Hawkins, 51). 

\par Hawkins is an electrical engineer, and he hasn't had a professional degree in neuroscience. His framework approaches the problem from engineering concepts as well as a study of the research done behind the human brain to formulate his framework. He treats the network of neurons in our brain as an encoding problem, and he quotes ``All your brain knows is patterns" (Hawkins, 56).

\epigraph{``All the information that enters your mind comes in as spatial and temporal patterns on the axons.''}{(Hawkins, 57)} 

\epigraph{``The brain quickly learns to interpret the patterns correctly.''}{(Hawkins, 61)} 

\epigraph{``Patterns are the fundamental currency of intelligence that leads to some interesting philosophical questions.''}{(Hawkins, 63)} 

\epigraph{``Neurons are quite slow compared to the transistors in a computer.''}{(Hawkins, 66)} 

\epigraph{``The brain is a parallel computer.''}{(Hawkins, 66)} 

\epigraph{``The entire cortex is a memory system. It isn't a computer at all.''}{(Hawkins, 68)} 

\epigraph{``Correct predictions result in understanding. Incorrect predictions result in confusion and prompt you to pay attention.''}{(Hawkins, 89)} 

\epigraph{``The memory of how to catch a ball was not programmed into your brain; it was learned over years of repetitive practice, and it is stored, not calculated, in your neurons.''}{(Hawkins, 69)} 

\epigraph{``Intelligence is measured by the capacity to remember and predict patterns in the world, including language, mathematics, physical properties of objects, and social situations.''}{(Hawkins, 97)} 


\begin{thebibliography}{9}
\bibitem{lamport94}
  Hawkins, Jeff, and Sandra Blakeslee,
  \emph{On Intelligence}.
  New York: Henry Holt, 2005.
  Print.
\end{thebibliography}

\end{document}  