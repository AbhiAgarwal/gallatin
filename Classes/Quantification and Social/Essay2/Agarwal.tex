\documentclass[11pt, oneside]{article}   	
\usepackage{geometry}                		
\usepackage{graphicx}	
\usepackage{amssymb}
\usepackage{epigraph}
\usepackage[doublespacing]{setspace}
\usepackage{etoolbox}

\setlength\epigraphwidth{15cm}
\setlength\epigraphrule{0pt}
\geometry{letterpaper} 
\makeatletter
\patchcmd{\epigraph}{\@epitext{#1}}{\itshape\@epitext{#1}}{}{}
\makeatother

\title{One learning algorithm}
\author{Abhi Agarwal (abhia@nyu.edu)}
\date{}							

\begin{document}
\maketitle

% Introduction
\par The one learning algorithm is a concept in machine learning and neuroscience that has been studied extensively by Jeff Hawkins. In his book, On Intelligence, he laid out a memory-prediction framework depicting what he thinks the one learning algorithm would consist of. His framework is based upon the work on columnar organization of the cerebral cortex done by Vernon Mountcase who shows that ``all regions of the cortex [perform] the same operation" (Hawkins, 51), and that ``there is a common function, a common algorithm, that is performed by all the cortical regions" (Hawkins, 51). In essence, the one learning algorithm accurately predicts outcomes of future input sequences based on patterns that this algorithm learns from regardless of what type of patterns they are (patterns can be images, videos, smell, etc.).

% Author
\par Hawkins is an electrical engineer, and hasn't had any professional experience in neuroscience. His framework approaches the problem from an engineer's perspective as well as his personal study of the research done on the cerebral cortex to formulate his framework. He treats the network of neurons in our brain as an encoding problem, and he quotes ``All your brain knows is patterns" (Hawkins, 56), which is truly how a mathematician would see it as patterns imply that some formulation could be found to connect the pattern together. He explains that our understanding of the world is based upon patterns and ``[correct] predictions result in understanding. Incorrect predictions result in confusion and prompt you to pay attention' (Hawkins, 89). 

% Defining the algorithm, introduces true intelligence, and defines intelligence. 
\par Hawkins key aim is to create an algorithm that mirrors how we learn, and form what he calls `true intelligence' rather than forming `artificial intelligence'. He defines true intelligence as a ``[measure of the] capacity to remember and predict patterns in the world, including language, mathematics, physical properties of objects, and social situations" (Hawkins, 97). Hawkins' view of how our brain recognizes patterns and learns is that our ``brain doesn't ``compute" the answers to problems; it retrieves the answers from memory ... entire cortex is a memory system. It isn't a computer at all'' (Hawkins, 68). Thus, Hawkins sees the brain as being able to predict the answer or the best outcome of a certain task by having a very efficient prediction system that utilizes the connection of neurons in our brain to find the best prediction. This is in contrast to how most individuals think about learning and how the neurons in the brain responses to actions. Most individuals believe that the reactions to actions are, in a sense, computed at that particular moment and our brain is built on logic, assumptions, and prior knowledge. This contrast arises the question is learning something we can model as patterns? Can how we think be quantified and put into a series of steps or is it more complex? Can the one learning algorithm be used to represent `true intelligence', and the process of learning in a similar way that our brains do?

% Defining Learning, and trying to understand existing learning.
\par 
Approaching this algorithm from an mathematical perspective, a mathematical model of learning would ignore a lot of things that we believe contribute to learning, and our definition of learning. Lets take an objective definition of learning to be the ``process of acquiring modifications in existing knowledge, skills, habits, or tendencies through experience, practice, or exercise" (Merriam-Webster, ``Learning"). 
Knowledge to us is an understanding of pieces of information through experience or learning, and we usually recall these pieces of information using languages that we know, or visual descriptions that we collect. However, to a computer knowledge is represented as a series of numerical values and things a computer learns are converted, through some measure, into numerical values to be processed through the algorithm and then stored. 
Without representing them as numerical values there is no way a computer can understand their meanings, as these algorithms require some medium to compare and relate two pieces of information and to classify this new piece of information. 
Once we input a phrase into this particular learning algorithm it becomes arbitrary. It gets converted into a series of values that the computer performs actions upon to match the particular pattern, and to try to classify it. 
In contrast, our brains understand this information as it has complex biological traits that allow it to interpret and break down these words, and processes them through its existing knowledge. The question arises that can a learning algorithm react to situations using true intelligence? 

% Step up from it where we think about experience, and building on experience, and the definition of languages - how can Experience be formed with a mathematical approach?

% True intelligence, and Hawkins intelligence
\par If we define true intelligence to be ``the ability to learn or understand things or to deal with new or difficult situations" (Merriam-Webster, ``Intelligence") then we can begin to look at Hawkins definition of intelligence, and if that be used to represent true intelligence. 

%\par What is the approach of us practicing something? Is a computer really practicing or just computing it over and over again given an equation? Is it truly practicing? Is this important?

% Experience
\par In addition, in the definition of learning there was an aspect of experience. What we perceive as experience and what the one learning algorithm would define to be experience would be very different. 

% Bring in the single learning algorithm
\par A single learning algorithm will represent and be able to make predictions, using experience and existing knowledge, about anything and would be able to learn things regardless of their medium. 

\par In conclusion, the intelligence or learning capabilities that a computer will have could never will truly intelligent. They are being developed by quantifying learning and intelligence and using those arbitrary values to make predictions about the future


%\epigraph{``All the information that enters your mind comes in as spatial and temporal patterns on the axons.''}{(Hawkins, 57)} 

%\epigraph{``The brain quickly learns to interpret the patterns correctly.''}{(Hawkins, 61)} 

%\epigraph{``Patterns are the fundamental currency of intelligence that leads to some interesting philosophical questions.''}{(Hawkins, 63)} 

%\epigraph{``Neurons are quite slow compared to the transistors in a computer.''}{(Hawkins, 66)} 

%\epigraph{``The brain is a parallel computer.''}{(Hawkins, 66)} 

%\epigraph{``The memory of how to catch a ball was not programmed into your brain; it was learned over years of repetitive practice, and it is stored, not calculated, in your neurons.''}{(Hawkins, 69)} 


\begin{thebibliography}{9}
\bibitem{1}
  Hawkins, Jeff, and Sandra Blakeslee,
  \emph{On Intelligence}.
  New York: Henry Holt, 2005.
  Print.
\bibitem{2}
  Merriam-Webster,
  \emph{Learning}.
  Merriam-Webster, n.d. Web. 05 Nov. 2014.
  Electronic.
\bibitem{3}
  Merriam-Webster,
  \emph{Intelligence}.
  Merriam-Webster, n.d. Web. 05 Nov. 2014.
  Electronic.
\end{thebibliography}
\end{document}  