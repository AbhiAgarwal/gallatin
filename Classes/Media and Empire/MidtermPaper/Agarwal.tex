\documentclass[12pt, oneside]{article}
\usepackage[a4paper]{geometry}
\geometry{letterpaper}
\usepackage{amssymb}
\linespread{2.2}
\usepackage[affil-it]{authblk}
\usepackage{etoolbox}
\usepackage{lmodern}
\usepackage{ifpdf}
\usepackage{mla}
\usepackage{blindtext}
\usepackage[english]{babel}
\usepackage{fancyhdr}

\renewcommand\Authfont{\fontsize{12}{14.4}\selectfont}
\renewcommand\Affilfont{\fontsize{9}{10.8}\itshape}

\setlength{\voffset}{-0.7in}
\setlength{\headsep}{6pt}

\title{Media technologies and the modern empire\vspace{-0.4cm}}
\author{Abhi Agarwal\vspace{-1cm}}
\date{}

\begin{document}

\maketitle

\par Experimentation with media technologies helped shape and develop many of the methods that are used in today?s society by governmental organizations, and also helped build the colonial powers between the 19th and early 20th century. Media infrastructure and technology were used not just as a form of control over another society, but also as a measure of the value and progress that a society has made; as Adas writes ?European observers came to view science and especially technology as the most objective and unassailable measures of their own civilization's past achievement and present worth? (Adas, 134). There was an inherent belief that valuing a society based on their technology was objective, and introducing societies to technologies could help bring progress in those societies.
\par Having better technology meant that the colonial power had dominance over other countries, as they saw it as an indication of superiority. The colonized countries were not able to compete against the military power of the colonizer. This was primarily the case in initially gaining colonies, as the European countries had better technology to begin with. In his book, ?Policing America?s Empire?, McCoy writes ?the creation of sophisticated modern policing was crucial to the U.S. pacification of the Philippines? (McCoy, 16). The development of technologies such as the telegraph, telephone, and other modern media technologies allowed for communication between the different colonies that the colonizer had. 
\par Colonial power gets stronger at home and abroad as it learns from its military and political experimentation. McCoy discusses in depth the advantage that the colonialist power obtains from experimenting abroad, specifically in Philippines. He writes, one way that America uses its colonization over Philippines is as ?a social experiment in the use of police as an instrument of power? (McCoy, 16), and ?war transforms technology and industry, so colonialism plays a comparable role for government, producing innovations, particularly in the use of coercive controls, with a profound impact on its bureaucracies both home and abroad? (McCoy, 37). The colonial power is able to try many approaches in that state without facing repercussions in its other states or in their home country. This is an important distinction from experimenting on its own land, as the downside of the gamble that it makes on different tactics is not very high, and in most circumstances, does not affect its soldiers or citizens at home. It is able to keep iterating on these military or political techniques in its many colonies, and can keep improving its tactics on both keeping control of its existing colonies and gaining new ones. The countries being colonized by these nations become social and political laboratories for the colonizer.
\par Experimentation helps the colonial power learn about what works and what does not in that state, and what has worked in that state in the past. McCoy discussed the use of the national police and how post-independence it ?remained a key instrument of both legal and extralegal presidential power? (McCoy, 16). The control of both the national police and the media allowed them to selectively govern the political situation in the Philippines. The colonial power could control both the wires and the content. It was able to control the personnel who came into power in the Philippines and experiment with the different methods that were used to create and keep that control over the political situation. Through the experimentation of different tactics, they are able to gain more military and political strength, thereby increasing their power within the country, on the politicians and people in it. In addition, gaining more political support from the national police increases their strength even from abroad, as they start developing control over the security apparatuses of the state.
\par The traits of being able to experiment in colonized countries helps to initially develop, and then later iterate on media technologies more easily, which is necessary for innovation and for the growth of the empire. The state benefits significantly from having full freedom over the tactics it develops in nations it is aiming to colonize, and McCoy also believes that experimentation along with the improvement in technology helped the United States strengthen its colonization in the Philippines.
\par Progress in media technologies acts as a way to temporarily impress the colony with the technological and political power that the colonizer has, and hand over their nation in hopes of also technologically developing. Larkin writes, the "[grand] openings of infrastructure projects like the Kano Water and Electric Light Works are both a visual and spectacle and political ritual... where the public display of colonial authority is made manifest" (Larkin, 19). In Nigeria the British were able to enforce their colonial regime by celebrating the development of Nigeria and reminding the Nigerians how much the British could ?help? their country. Larkin describes this as colonial sublime. The "intent in using infrastructure technologies in colonial rule was to provoke feelings of the sublime not through the grandeur nature but through the work of humankind" (Larkin, 36). Technology is here used as a way to invoke the sublime as the colonizer uses these bridges and electricity projects to raise this idea that a more developed culture is bringing progress. They create a sense of both surprise and beauty, and temporarily shows them the power of the colonial power and the level of progress it could bring.
By developing media infrastructure, the colonizers can spread political information to the citizens through means of mass media. Shohat and Stam, in their book ?Unthinking Eurocentrism?, write ?[newspapers] - like TV news today - made people aware of the simultaneity and interconnectedness of events in different places, while novels provided a sense of the purposeful movement through time of fictional entities bound together in a narrative whole? (Shohat and Stam, 102), and storytelling through different mediums was incredibly useful ?to relay the projected narratives of nations and empires? (Shohat and Stam, 101). 
\par Having fairs and cinemas allowed ?millions of fairgoers [be introduced to] evolutionary ideas about race in an atmosphere of communal good cheer? (Shohat and Stam, 27). This was used to shift focus away from the arguments against colonization and towards the colonizer partnership being a mutual gain for both countries and sometimes advertising it as a gain only for the country being colonised. Cinema and video are especially important as they serve two purposes. The initially purpose is that bringing cinema or the ability to watch video to a country brings development, and there is a sense of sublimity in seeing these videos with a group of people. Second, ?[if] cinema partly inherited the function of the novel, it also transformed it. Whereas literature plays itself out within a virtual lexical space, the cinematic chronotope is literal, splayed out concretely across the screen and unfolding in the literal time of twenty-four frames per second. In this sense, the cinema can all the more efficiently mobilize desire in ways responsive to nationalized and imperialised notions of time, plot and history? (Shohat and Stam, 103). Cinema is very powerful in depicting a message as it develops a story along with visual effects that the view can be immersed into, which is much more powerful on a larger group than text could be. The development of cinema was a powerful tool both within the country that was colonizing and also the country which was being colonized.
\par Similarly, techniques that developed in advertising and in media allowed the colonizers to influence and manipulate the habits of the consumers in those colonies. Anne McClintock writes about this in her book ?Soft-Soaping Empire: Commodity Racism and Imperial Advertising?. The discussion McClintock brings up in her book is relevant to the development of media infrastructure, specifically through radio and fairs and its link to racial superiority. The narrative that prefaced her theory is the idea that the colonized, from the perspective of the colonizers, should be like the Europeans, and set their aims to bring about progress in technology and social structures to mirror the advancements and culture of western countries. McClintock writes about how soap and household/daily products were a shift from this idea and its significance. The introduction of soap (and other household products) created the ability for individuals in these colonies to be European, and not just like the Europeans, and to be able to use the same daily products as they did, and this brought about a new shift in the way they could advertise. 
\par People in the colonies would want to look like the Europeans, be clean like the Europeans, and most importantly, be white like the Europeans. The advertising and branding in those countries reflected these values, and it changed the discourse as people began buying utilities and daily products that made them feel like the Europeans. From the perspective of the colonizer, this is incredibly successful as it embeds these products into their daily lives and the country begins to increasingly rely on imports. McClintock writes, that the ?mystique of [a] commodity fetish lies not in its use value but in [the] exchange value and its potency as a sign? (McClintock, 220). The importance of her work here is to show that the product has a significance that is greater than its utility. It represents a meaning of racial supremacy, which is not realized by the colonized countries as it is advertised to be almost a subconscious.
\par Media technologies brought numerous benefits for the colonizer, but also brought a lot of opposition from the colonized in the long run. Larkin writes, "[the] coming of electricity effected a split in Nigeria between electrified and modern towns and those that remained without electric power" (Larkin, 18). In this situation technology creates a division within the country, and is problematic as it creates a separation between the developed cities and the rural cities. Larkin discusses the idea that the colonial sublime only lasts so long, and the idea that the colonizers did not predict that the colonial sublime was so fragile and that it could be short lived. People in these communities start to question why these countries are bringing them these technologies and this progress. 
\par In addition, the separation that is created between the different parts of the country becomes dangerous to the colonial power as it could lead to a revolution against its power. This is primarily because it begins to increase the inequality between two parts of the country. Technology, in most cases, adds to the inequality as not all individuals in the country are uplifted by the additions that the power brings, and so the media technology and this improvement in infrastructure can also be disadvantageous to the colonial power. In Nigeria, the British colonial power brought technology to the capital cities, which were more educated, wealthy, and developed in comparison to the rest of Nigeria. This increase in inequality between the major cities and the rest of Nigeria was not only dangerous for the British empire, but also for Nigeria itself. It createsed a social, economical, and religious separation in the country which impacts the country in the long run, and the impacts of these effects are still present in Nigeria. Larkin writes, Nigeria is constantly faced with structural conflicts between ?progress and preservation, identity and difference ... isolation and connection? (Larkin, 32) because of the development of these technologies.
\par By gaining more territory, the colonizers themselves gain access to an increased amount and a more diverse set of resources, and with these resources they are able to increase the means of production. This is one of the aims for most colonial powers; all of the large empires are interested in developing new colonies as a way to increase production and innovation in their main countries. By gaining newer colonies they are able to utilize newer resources to produce and further their technical infrastructure, and colonies are able to maintain and improve control over these countries by making innovations. This creates a cycle for the colonies as unless some event happens in opposition, they are able to continue developing and innovating and influencing more of the colonized country through their media technologies.
\par Overall, the development of media technologies were vital in the development of the modern Empire. It helped accelerate communication between the colonizer and its colonies, allowed for experimentation, and the development of film and news infrastructure helped influence the thoughts of the general public in each of the colonies. 

\end{document}