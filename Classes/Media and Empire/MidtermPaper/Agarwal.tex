\documentclass[12pt, oneside]{article}
\usepackage[a4paper]{geometry}
\geometry{letterpaper}
\usepackage{amssymb}
\linespread{2.1}
\usepackage[affil-it]{authblk}
\usepackage{etoolbox}
\usepackage{lmodern}
\usepackage{ifpdf}
\usepackage{mla}
\usepackage{blindtext}
\usepackage[english]{babel}
\usepackage{fancyhdr}

\renewcommand\Authfont{\fontsize{12}{14.4}\selectfont}
\renewcommand\Affilfont{\fontsize{9}{10.8}\itshape}

\setlength{\voffset}{-0.7in}
\setlength{\headsep}{6pt}

\title{Media technologies and the modern empire\vspace{-0.4cm}}
\author{Abhi Agarwal\vspace{-1cm}}
\date{}

\begin{document}

\maketitle

% Material as well as symbolic dimensions that have shaped "Territorial or military control"

% Material as well as symbolic dimensions that have shaped "Racial and cultural supremacy"

% Discuss how anti-colonial thinkers address the power of media infrastructure and culture in their writings


% Colonialist gets more powerful as it does experimentation
\par Colonial power gets stronger at home as it learns from experimentation.
\par ``war transforms technology and industry, so colonialism plays a comparable role for government, producing innovations, particularly in the use of coercive controls, with a profound impact on its bureaucracies both home and abroad ... experiments whose lessons were later repatriated through policies and personnel" (McCoy, 37).
\par ``a social experiment in the use of police as an instrument of power" (McCoy, 16)

% Experimentation not only helps the colonist get stronger at home, but also in that country as it recognizes what works
\par Experimentation helps the colonialist power learn about what works and what does not in that state.
\par Give example of America learning about filipino control of media and of rigging elections.

% Having better technology means that it has dominance over other countries. This is because the colonialist powers saw this as being more superior.

\noindent

\begin{workscited}
\bibent \\
\bibent Hawkins, Jeff, and Sandra Blakeslee, On Intelligence. New York: Henry Holt, 2005. Print. \\
\end{workscited}
\end{document}