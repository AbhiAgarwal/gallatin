\documentclass[11pt, oneside]{article}
\usepackage{geometry}
\geometry{letterpaper}
\usepackage{graphicx}
\usepackage{amssymb}

\title{Exploring moral behavior of thinking machines}
\author{Abhi Agarwal}
\date{}
\begin{document}
\maketitle

% Plan
% 	Start with de waal on evolution of primates 
% 	Introduced further evolution step to machines 
% 	We have unique traits compared to primates in morality and we evolved to have those
% 	Can machines evolve and have their own unique distinctions in morality 
% 	Machines can have perfect knowledge of each other. This would mean that they would be perfectly in sync 
% 	Would that mean they are still moral or 
% 	Aristotle framework for morality would be accomplished

\par Did morality evolve? Frans de Waal centralizes this idea and furthers it by asking how morality evolved. Opposing de Wall on the idea are a small group of biologists. They argue that morality is a construct or an idea that is unique to human beings, and the purpose of morality is to minimize our animalistic instincts. de Wall gives accounts for the idea that morality has evolved continuously and has evolved from Chimpanzees, Bonobos, and Great Apes.

%\par There is a unique aspect to our morality. We have managed to define a set language which we use to communicate. 
%\par The question that arises from this is - what 
%\par My proposal is to begin by defining what a moral agent and an artificial moral agent is - the general idea of Moral Machines.
%\par What does it mean for morality to evolve?
%\par Is morality learned? Does evolution of morality mean we take principles we have learnt of morality and teach/embed that %within computers?
%\par Since we could be able to program a computer to make certain deliberations, what should the deliberations be?
%\par What goals would we program a computer to fulfill?

\begin{thebibliography}{9}

\bibitem{Superintelligence}
  Bostrom, Nick. 
  \emph{Superintelligence: The Coming Machine Intelligence Revolution}.
  Oxford: Oxford UP, 2013. 
  Print.

\end{thebibliography}
\end{document}  