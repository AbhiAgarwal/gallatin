\documentclass[11pt, oneside]{article}
\usepackage{geometry}
\geometry{letterpaper}
\usepackage{graphicx}
\usepackage{amssymb}
\linespread{2.0}	

\title{Exploring moral behavior of thinking machines}
\author{Abhi Agarwal}
\date{}
\begin{document}
\maketitle

\par Did morality evolve? Frans de Waal centralizes this idea and furthers it by asking how morality evolved. Opposing de Waal on the idea are a small group of biologists. They argue that morality is a construct or an idea that is unique to human beings, and the purpose of morality is to minimize our animalistic instincts. De Waal gives accounts for the idea that morality has evolved continuously and has evolved from Chimpanzees, Bonobos, and Great Apes. He examines different species of animals and points out that primates close to us in the ladder of evolution display a large number of moral traits. Traits such as peacemaking, empathy, sympathy, and many more.

\par The smaller group of biologists consists of individuals like Thomas Huxley and George Williams. De Wall, in his book, coined the term Veneer theory to encapsulate their arguments. Veneer theory ``assumes that deep down we are not truly moral. It views morality as a cultural overlay, a thin veneer hiding an otherwise selfish and brutish nature" (De Wall, 6). 

\par There are aspects of morality that are unique to human beings. De Wall notes that the ability to weigh, reason, and judge two separate moral decisions and choose an outcome is one of them. The other is the ability to be impartial and spectate a situation. 

\begin{thebibliography}{9}

\bibitem{Superintelligence}
  Bostrom, Nick. 
  \emph{Superintelligence: The Coming Machine Intelligence Revolution}.
  Oxford: Oxford UP, 2013. 
  Print.

\end{thebibliography}
\end{document}  