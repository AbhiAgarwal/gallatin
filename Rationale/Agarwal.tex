%!TEX TS-program = xelatex
%!TEX encoding = UTF-8 Unicode
\documentclass[11pt]{article}
\usepackage{geometry}
\geometry{letterpaper}
\usepackage{graphicx}
\usepackage{amssymb}
\usepackage{amsthm}
\usepackage{fontspec,xltxtra,xunicode}
\defaultfontfeatures{Mapping=tex-text}
\setromanfont[Mapping=tex-text]{Hoefler Text}
\setsansfont[Scale=MatchLowercase,Mapping=tex-text]{Gill Sans}
\setmonofont[Scale=MatchLowercase]{Andale Mono}
\newfontfamily{\J}[Scale=0.85]{Osaka}

% \newtheorem*{nhp-definition}{The new-human problem}

%\title{Gallatin Undergraduate Rationale: The new-human problem}
\title{Gallatin Undergraduate Rationale: The creation of Artificial Intelligence}
\author{Abhi Agarwal}
\date{}

\begin{document}
\maketitle

% Abstract
%\begin{nhp-definition}
%Suppose a new-human appeared on an island. The new-human is as perfectly capable of learning, %reasoning, and viewing the world as we are. The new-human problem is to determine the obscurities in our society that exist that the new-human would have to learn to fit into our society.
%\end{nhp-definition}

% 1. Portrayal & development of AI in books, media, and research
{\large What is AI? How is it Potrayed \& developed of AI in books, media, and research?\par}

\par Artificial Intelligence has been a hotly debated topic in the last decade, and even more debated in the media in the last couple years. 

\par What is intelligence? How do we define it?\\

% 2. Evaluating intelligence. Intelligence testing & factors of intelligence for intelligent machines
% - Methods of evaluating intelligence in machines
% - Exploring methods of evaluating intelligence in machines
{\large Intelligence testing \& factors of intelligence for intelligent machines\par}

% 3. Conscience, rationality, intuition, common sense, and their relation to intelligence. How that ipacts building strong AI or Artificial General Intelligence.
{\large Conscience, desire, rationality, intuition, common sense, and their relation to intelligence\par}

% 4. Methods of quantification for a thinking or intelligence machines. How to perceive the world if you think about everything in numbers.
{\large Methods of quantification for a thinking or intelligence machines\par}

% 5. Superintelligence vs intelligence
{\large Superintelligence vs intelligence\par}

% 6. Embedding of AI into our social structure.
{\large Embedding of AI into our social structure\par}

% Impacts of the creation of AI
{\large Impacts of the creation\par}

\newpage

{\Large Booklist\\\par}

\par \textbf{Ancient, Medieval and Renaissance Classics}
\par At least seven works produced before the mid-1600s.
\begin{itemize}
	\item Sidereus Nuncius ­-- Galileo Galilei
	\item The Discourse on Method -- Rene Descartes
	\item The Republic -- Plato
	\item Bhagavad Gita -- Sage Vyasa
	\item Utopia -- Thomas More
\end{itemize}

\par \textbf{Modernity—The Humanities}
\par At least four works, produced after the mid-1600s, in Humanities disciplines such as Literature, Philosophy, History, the Arts, Critical Theory and Religion.
\begin{itemize}
	\item Treatise of Human Nature -- David Hume
\end{itemize}

\par \textbf{Modernity—The Social and Natural Sciences}
\par At least four nonfiction works, produced after the mid-1600s, in the Natural Sciences and Social Science disciplines such as Political Science, Economics, Psychology, Anthropology, and Sociology.
\begin{itemize}
	\item An Enquiry of Human Understanding -- David Hume
	\item On the Origins of Species -- Charles Darwin
	\item The Descent of Man -- Charles Darwin
	\item On Natural Selection -- Charles Darwin
	\item Computing Machinery and Intelligence -- A. M. Turing
	\item Hackers -- Steven Levy
	\item The Bell Curve -- Richard J. Herrnstein and Charles Murray
\end{itemize}

\par \textbf{Area of Concentration}
\par At least five additional works representing the student's area or areas of concentration; students whose area of concentration already appears among the above categories may simply choose five additional works from these categories.
\begin{itemize}
	\item Godel, Escher, Bach -- Douglas R. Hofstadter
	\item The Human Brain -- Susan Greenfield
	\item On Intelligence -- Jeff Hawkins
	\item Superintelligence: Paths, Dangers, and Strategies -- Nick Bostrom
	\item Artificial Intelligence: A Modern Approach -- Stuart Russell and Peter Norvig
	\item The Big Test: The Secret History of the American Meritocracy -- Nicholas Lemann
	\item Dataclysm: Who We Are -- Christian Rudder
	\item The Singularity Is Near -- Ray Kurzweil
	\item Darwin's Devices -- John Long
\end{itemize}

\par \textbf{Optional influences}
\par These are books I have not fully read or embedded into my rationale, but I have used parts of these books to influence my arguments.
\begin{itemize}
	\item The Structure of Scientific Revolutions -- Thomas S. Kuhn
	\item Flatland: A romance of many dimensions -- Edwin A. Abbott
	\item I am a strange loop -- Douglas R. Hofstadter
\end{itemize}
\end{document}