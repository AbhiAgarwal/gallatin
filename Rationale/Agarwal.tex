%!TEX TS-program = xelatex
%!TEX encoding = UTF-8 Unicode
\documentclass[12pt]{article}
\usepackage{geometry}
\geometry{letterpaper}
\usepackage{graphicx}
\usepackage{amssymb}
\usepackage{amsthm}
\usepackage{fontspec,xltxtra,xunicode}
\defaultfontfeatures{Mapping=tex-text}
\setromanfont[Mapping=tex-text]{Hoefler Text}
\setsansfont[Scale=MatchLowercase,Mapping=tex-text]{Gill Sans}
\setmonofont[Scale=MatchLowercase]{Andale Mono}
\newfontfamily{\J}[Scale=0.85]{Osaka}

\newtheorem*{nhp-definition}{The new-human problem}

\title{Gallatin Undergraduate Rationale: The new-human problem}
\author{Abhi Agarwal}
\date{}

\begin{document}
\maketitle

% Abstract
\begin{nhp-definition}
Suppose a new-human appeared on an island. The new-human is as perfectly capable of learning, reasoning, and viewing the world as we are. The new-human problem is to determine the obscurities in our society that exist that the new-human would have to learn to fit into our society.
\end{nhp-definition}

% Introducing AI and problems around it
\par It's been pointed out by many researchers in the field of Artificial Intelligence that machines with general intelligence will arise within the next century. The aim of some researchers in the field is to create a machine that perfectly mimics a human-being, and the aim of other researchers in the field is to create a machine that intelligently assists human-beings in their day to day life.

\par In the book `Machines Who Think`, Pamela McCorduck notes that AI was ``an ancient wish to forge the gods".

% Pre-formulation of AI
\par Since way before the field was created there have been written works that depict intelligent mechanical men. They first appeared in Greek myths, such as the golden robots of Hephaestus and Pygmalion's Galatea. 

% Post-formulation of AI
\par 

% 

\end{document}