%!TEX TS-program = xelatex
%!TEX encoding = UTF-8 Unicode
\documentclass[12pt]{article}
\usepackage{geometry}
\geometry{letterpaper}
\usepackage{graphicx}
\usepackage{amssymb}
\usepackage{amsthm}
\usepackage{fontspec,xltxtra,xunicode}
\defaultfontfeatures{Mapping=tex-text}
\setromanfont[Mapping=tex-text]{Hoefler Text}
\setsansfont[Scale=MatchLowercase,Mapping=tex-text]{Gill Sans}
\setmonofont[Scale=MatchLowercase]{Andale Mono}
\newfontfamily{\J}[Scale=0.85]{Osaka}

\newtheorem*{nhp-definition}{The new-human problem}

\title{Gallatin Undergraduate Rationale: The new-human problem}
\author{Abhi Agarwal}
\date{}

\begin{document}
\maketitle

% Abstract
\begin{nhp-definition}
Suppose a new-human appeared on an island. The new-human is as perfectly capable of learning, reasoning, and viewing the world as we are. The new-human problem is to determine the obscurities in our society that exist that the new-human would have to learn to fit into our society.
\end{nhp-definition}

% 1. Portrayal & development of AI in books, media, and research

% 2. Evaluating intelligence. Intelligence testing & factors of intelligence for intelligent machines
% - Methods of evaluating intelligence in machines
% - Exploring methods of evaluating intelligence in machines

% 3. Methods of quantification for a thinking or intelligence machines. How to perceive the world if you think about everything in numbers.

% 4. Superintelligence vs intelligence

% 5. Embedding of AI into our social structure.

% 6. Conscience, rationality, intuition, common sense, and their relation to intelligence. How that ipacts building strong AI or Artificial General Intelligence.

% Impacts of the creation of AI

\end{document}