%!TEX TS-program = xelatex
%!TEX encoding = UTF-8 Unicode
\documentclass[12pt]{article}
\usepackage{geometry}
\geometry{letterpaper}
\usepackage{graphicx}
\usepackage{amssymb}
\usepackage{amsthm}
\usepackage{fontspec,xltxtra,xunicode}
\defaultfontfeatures{Mapping=tex-text}
\setromanfont[Mapping=tex-text]{Hoefler Text}
\setsansfont[Scale=MatchLowercase,Mapping=tex-text]{Gill Sans}
\setmonofont[Scale=MatchLowercase]{Andale Mono}
\newfontfamily{\J}[Scale=0.85]{Osaka}

\newtheorem*{nhp-definition}{The new-human problem}

\title{Gallatin Undergraduate Rationale: The new-human problem}
\author{Abhi Agarwal}
\date{}

\begin{document}
\maketitle

\begin{nhp-definition}
Suppose a new-human appeared on an island. The new-human is as perfectly capable of learning, reasoning, and viewing the world as we are. The new-human problem is to 
\end{nhp-definition}

\par It's been pointed out by many researchers in the field of Artificial Intelligence that machines with general intelligence will `something` within this century. The aim of some researchers in the field is to create a machine that perfectly mimics a human-being, and in order to study their usefulness we have to begin to understand the process that they will learn things 
\par 

\end{document}